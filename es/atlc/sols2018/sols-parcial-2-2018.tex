\documentclass[spanish]{article}

\usepackage{color}
\definecolor{darkblue}{RGB}{49,130,189}

% Access modes
\newcommand{\acmo}[1]{\underline{\texttt{#1}}}

% Print today's date in ISO format (YYYY-MM-DD).
\def\isodate{\leavevmode\hbox{\the\year-\twodigits\month-\twodigits\day}}
\def\twodigits#1{\ifnum#1<10 0\fi\the#1}

\usepackage[utf8]{inputenc}
\usepackage[es-tabla]{babel}
\usepackage{booktabs}

\usepackage{mathtools}
\usepackage{tikz}
\usetikzlibrary{automata, positioning,arrows.meta}
\usetikzlibrary{babel}
\usetikzlibrary{decorations.text}
\usetikzlibrary{decorations.pathreplacing}

\usepackage{xpatch}
\DeclareMathVersion{ttmath}
\SetSymbolFont{letters}{ttmath}{OT1}{\ttdefault}{m}{n}
\xapptocmd{\ttfamily}{\mathversion{ttmath}}{}{}

\newcommand{\emptystr}{\lambda}

\usetikzlibrary{automata, positioning, arrows,
fit, % for the dashed boxes on Thompson's construction
calc
}
\tikzset{%
  node distance=3cm, % specifies the minimum distance between two nodes. Change if necessary.
  every state/.style={thick}, % sets the properties for each ’state’ node
  double distance=2.5pt,
  shorten >= 2pt, shorten <= 2pt,
  initial text=\( \),
  every edge/.style={%
    draw,->, >=stealth, auto, semithick
  }
}

\usepackage{hyperref}
\hypersetup{pdfauthor={Cristian Adrián Ontivero}}
\usetikzlibrary{shapes}
\graphicspath{{imgs/}}

\newlength\tindent
\setlength{\tindent}{\parindent}
\setlength{\parindent}{0pt}
\renewcommand{\indent}{\hspace*{\tindent}}

\renewcommand*{\tableautorefname}{Tabla}
\renewcommand*{\figureautorefname}{Figura}
\newcommand{\smallunderbracket}[1]{\underbracket[0.7pt][2pt]{#1}}

%These tell TeX which packages to use.
\usepackage{array,epsfig}
\usepackage{amsmath}
\usepackage{amsfonts}
\usepackage{amssymb}
\usepackage{amsxtra}
\usepackage{amsthm}
\usepackage{mathrsfs}

\begin{document}
\centerline{\LARGE{Autómatas, Teoría de Lenguajes y Compiladores (72.39)}}
\vspace{6pt}
\centerline{\Large{Soluciones del 2\textsuperscript{do} Parcial 2018}}

\subsubsection*{Ejercicio 1}
Dado el alfabeto \(\Sigma = \{{\tt 0, 1}\}\), y el lenguaje \(L = \{\omega {\tt a} \omega^r~/~\omega \in \Sigma^*, {\tt a} \in \Sigma \}\),
  construir una gramática libre de contexto que genere \(L\), y construir en base a
  ella un autómata de pila que reconozca \(L\).

\paragraph{Solución}
\subsubsection*{Ejercicio 2}
Demostrar que el lenguaje $L=\{{\tt 0^n \# 0^{2n} \# 0^{3n} }~/~{\tt n}
  \geq 0 \}$ \textbf{no es libre de contexto}.

\paragraph{Solución}

Lo probamos por contradicción. Supongamos que \(L$ es un CFL.\@ Entonces $L\) debe
cumplir el lema de bombeo para CFLs. Es decir, existe un entero \(p \geq 1\), tal que para todo
\(w \in L$ de longitud igual o mayor que $p$, $w$ puede ser escrito como $uvxyz\),
satisfaciendo las siguientes condiciones:
\begin{enumerate}
  \item \(|vxy| \geq p\)
  \item \(|vx| \geq 1\)
  \item \(\forall i\geq 0~(uv^ixy^iz \in L) \)
\end{enumerate}

En particular, sea \(w = {\tt 0^{p} \# 0^{2p} \# 0^{3p}}$. Mostramos que $w\) no puede ser
bombeado. Ni \(v$ ni $y$ pueden contener $\#$, de lo contrario $uv^2xy^2z\) tendría
más de dos \(\#$. Por lo tanto, si dividimos $w$ en tres segmentos: $\tt 0^p\),
\(\tt 0^{2p}$ y $\tt 0^{3p}\), al menos uno de los segmentos no está contenido en
ni en \(v$ ni en $y$. En consecuencia $uv^2xy^z \not\in L\) porque la proporción
\(1:2:3\) de la longitud de los segmentos no se mantiene.

Supusimos que el lenguaje era libre de contexto y llegamos a una
contradicción con la tercera condición del lema. Por lo tanto, el lenguaje no
puede ser libre de contexto.

\[
\begin{tikzpicture} [node distance=2.4cm, every state/.style={thick, minimum
  size=31pt}, every edge/.style={draw,->, >=stealth, auto, semithick,font=\tt}]
  \node[state, initial] (i) {\(Q_0\)};
  \node[state, right of=i] (q24) {\(Q_{24}\)};
  \node[state, below of=q24] (q34) {\(Q_{34}\)};
  \node[state, right of=q24] (q5) {\(Q_5\)};
  \node[state, above of=q24] (q145) {\(Q_{145}\)};

  \node[state, right of=q5] (q6) {\(Q_6\)};
  \node[state, right of=q6, accepting] (q7) {\(Q_7\)};

  \draw (i) edge[out=90, in=180] node[sloped] {a} (q145);
  \draw (i) edge node {b} (q24);
  \draw (i) edge[out=270, in=180] node[sloped, below] {c} (q34);

  \draw[loop above] (q145) edge node {a} (q145);
  \draw (q145) edge[out=0, in=90] node[sloped] {a} (q6);

  \draw (q24) edge node {a} (q5);
  \draw[loop above] (q24) edge node {b} (q24);

  \draw (q34) edge[out=0, in=270] node[sloped, below] {a} (q5);
  \draw[loop above] (q34) edge node {c} (q34);

  \draw (q5) edge node {a} (q6);
  \draw (q6) edge node {b} (q7);
\end{tikzpicture}
\]

\subsubsection*{Ejercicio 3}
Describir de forma completa una máquina de Turing que reconozca el lenguaje del ejercicio anterior.

\begin{itemize}
  \item \textbf{Explicar claramente el funcionamiento de la máquina}.
  \item Mostrar \textbf{formalmente} que acepta \(\tt 0\#00\#000\)  pero no acepta \(\tt 0\#000\#0\).
\end{itemize}


\end{document}

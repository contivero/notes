\documentclass[spanish]{article}

\usepackage{color}
\definecolor{darkblue}{RGB}{49,130,189}

% Access modes
\newcommand{\acmo}[1]{\underline{\texttt{#1}}}

% Print today's date in ISO format (YYYY-MM-DD).
\def\isodate{\leavevmode\hbox{\the\year-\twodigits\month-\twodigits\day}}
\def\twodigits#1{\ifnum#1<10 0\fi\the#1}


\usepackage[utf8]{inputenc}
\usepackage[es-tabla]{babel}
\usepackage{booktabs}

\usepackage{mathtools}
\usepackage{tikz}
\usetikzlibrary{automata, positioning,arrows.meta}
\usetikzlibrary{babel}
\usetikzlibrary{decorations.text}
\usetikzlibrary{decorations.pathreplacing}

\usepackage{xpatch}
\DeclareMathVersion{ttmath}
\SetSymbolFont{letters}{ttmath}{OT1}{\ttdefault}{m}{n}
\xapptocmd{\ttfamily}{\mathversion{ttmath}}{}{}

\newcommand{\emptystr}{\lambda}

\usetikzlibrary{automata, positioning, arrows,
fit, % for the dashed boxes on Thompson's construction
calc
}
\tikzset{%
  node distance=3cm, % specifies the minimum distance between two nodes. Change if necessary.
  every state/.style={thick}, % sets the properties for each ’state’ node
  double distance=2.5pt,
  shorten >= 2pt, shorten <= 2pt,
  initial text=$ $,
  every edge/.style={%
    draw,->, >=stealth, auto, semithick
  }
}

\usepackage{hyperref}
\hypersetup{pdfauthor={Cristian Adrián Ontivero}}
\usetikzlibrary{shapes}
\graphicspath{{imgs/}}

\newlength\tindent
\setlength{\tindent}{\parindent}
\setlength{\parindent}{0pt}
\renewcommand{\indent}{\hspace*{\tindent}}

\renewcommand*{\tableautorefname}{Tabla}
\renewcommand*{\figureautorefname}{Figura}
\newcommand{\smallunderbracket}[1]{\underbracket[0.7pt][2pt]{#1}}

%These tell TeX which packages to use.
\usepackage{array,epsfig}
\usepackage{amsmath}
\usepackage{amsfonts}
\usepackage{amssymb}
\usepackage{amsxtra}
\usepackage{amsthm}
\usepackage{mathrsfs}

\begin{document}
\centerline{\LARGE{Autómatas, Teoría de Lenguajes y Compiladores (72.39)}}
\vspace{6pt}
\centerline{\Large{Soluciones del 1\textsuperscript{er} Parcial 2018}}
\paragraph{Ejercicio 1} Demostrar que el lenguaje $L = \{a^i b^j\mid i, j~\text{primos}\}$ no es regular.

\paragraph{Solución} Lo probamos por contradicción. Supongamos que $L$ es regular. Entonces $L$ debe cumplir el lema de bombeo para
lenguajes regulares. Es decir, existe un entero $p \geq 1$, tal que para todo
$w \in L$ de longitud igual o mayor que $p$, $w$ puede ser escrito como $xyz$,
satisfaciendo las siguientes condiciones:
\begin{enumerate}
  \item $|y| \geq 1$
  \item $|xy| \leq p$
  \item $\forall i\geq 0, xy^iz \in L$
\end{enumerate}

En particular, sea $w = a^nb^2$, con $n$ un primo mayor que $p$. Como $|w|\geq p$,
debemos poder escribir $w$ como $xyz$. Analizamos los posibles casos:
\begin{enumerate}
  \item $w = \smallunderbracket{a^r}_x \smallunderbracket{a^s}_y
    \smallunderbracket{bb}_z$

    Como $r + s = n > p$, y a su vez $r + s \leq p$ por la condición 2, este caso es
    absurdo. Lo mismo sucede en casos en los que $y$ cubre no sólo parte de las
    $a$, sino que también una o las dos $b$.

  \item $w = \smallunderbracket{a^r}_x \smallunderbracket{a^s}_y \smallunderbracket{a^tbb}_z$

    Ya que por la condición 3, $\forall i\geq 0(xy^iz \in L)$, tomamos en
    particular $i = r + s + t + 1 = n + 1$ para bombear, es decir $w' =
    xy^{n+1}z$. Entonces:
\begin{align*}
  |w'|_a & = r + si + t\\
        & = r + s(r + s + t + 1) + t\\
        & = r + sr + s^2 + st + s + t\\
        & = n + sr + s^2 + st\\
        & = n + s(r + s + t)\\
        & = n + sn\\
        & = n(1 + s)
\end{align*}
Dado que $n$ es primo, y $s = |y| \geq 1$ por la condición 1, $|w'|_a$ es un número
compuesto, por lo que $w' \not\in L$, contradiciendo la condición 3.
\end{enumerate}
Hemos visto que no hay configuración de $x, y, z$ que cumpla con el lema,
llegando a una contradicción. Por lo tanto, el lenguaje $L$ no es regular.

\paragraph{Ejercicio 2} Transformar paso a paso en autómata finito
deterministico mínimo a la expresión regular {\tt $(a^* + b^* + c^*)abb$}.
Justificar cada paso realizado en la transformación.

\paragraph{Solución}
Usando la construcción McNaughton-Yamada (también conocida como construcción
Thompson), construimos un $\emptystr$-NFA que
reconoce el mismo lenguaje denotado por la expresión regular. Un autómata que
reconozca {\tt a} (análogo para {\tt b} y {\tt c}) equivale a:
\[
\begin{tikzpicture} [node distance=2cm, every state/.style={thick, minimum size=16pt}]
  \node[state, initial] (A) {};
  \node[state, right of=A, accepting] (B) {};
  \draw (A) edge node {{\tt a}} (B);
\end{tikzpicture}
\]
Para que reconozca la clausura Kleene ${\tt a}^*$, siguiendo al pie de la letra
la construcción, tendríamos:
\[
\begin{tikzpicture} [node distance=2cm, every state/.style={thick, minimum size=16pt}]

    \node[state, node distance=3cm] (q15) {};
    \node[state, initial, left of=q15] (q16) {};
    \node[state, right of=q15] (q17) {};
    \node[state, accepting, right of=q17] (q18) {};

    \draw (q15) edge node {{\tt a}} (q17);
    \draw (q16) edge node {$\emptystr$} (q15);
    \draw (q17) edge node {$\emptystr$} (q18);
    \draw (q17) edge[bend right=80, looseness=1.5] node[sloped, above]
    {$\emptystr$} (q15);
    \draw (q16) edge[bend right=40] node[above] {$\emptystr$} (q18);
\end{tikzpicture}
\]
Pero esto complica pasos posteriores, y no es difícil ver que es equivalente al
autómata:
\[
\begin{tikzpicture} [node distance=2cm, every state/.style={thick, minimum size=16pt}]
  \node[state, initial, accepting] (A) {};
  \draw[loop above] (A) edge node {{\tt a}} (A);
\end{tikzpicture}
\]
Simplificando un poco (ya que si seguimos la construcción estrictamente, el
automata resultante es mayor), para reconocer {\tt $(a^*+b^*+c^*)$} usamos:
\[
\begin{tikzpicture} [node distance=2cm, every state/.style={thick, minimum size=16pt}]
  \node[state, initial] (i) {};
  \node[state, right of=i] (b) {};
  \node[state, above of=b] (a) {};
  \node[state, below of=b] (c) {};
  \node[state, right of=b, accepting] (d) {};
  %\draw(a) edge[out=15, in=345, looseness=8] node {a} (a);
  \draw[loop above] (a) edge node {{\tt a}} (a);
  \draw[loop above] (b) edge node {{\tt b}} (b);
  \draw[loop above] (c) edge node {{\tt c}} (c);
  \draw (i) edge node[above, sloped] {$\emptystr$} (a);
  \draw (i) edge node {$\emptystr$} (b);
  \draw (i) edge node[below, sloped] {$\emptystr$} (c);
  \draw (a) edge node[above, sloped] {$\emptystr$} (d);
  \draw (b) edge node {$\emptystr$} (d);
  \draw (c) edge node[below, sloped] {$\emptystr$} (d);
\end{tikzpicture}
\]
Para {\tt $abb$}, es simple ver que el siguiente lo reconoce:
\[
\begin{tikzpicture} [node distance=2cm, every state/.style={thick, minimum size=16pt}]
  \node[state, initial] (i) {};
  \node[state, right of=i] (a) {};
  \node[state, right of=a] (b) {};
  \node[state, right of=b, accepting] (c) {};
  \draw (i) edge node {{\tt a}} (a);
  \draw (a) edge node {{\tt b}} (b);
  \draw (b) edge node {{\tt b}} (c);
\end{tikzpicture}
\]
Finalmente, concatenamos ambas máquinas, para obtener un $\emptystr$-NFA que
reconozca la expresión regular:
\[
\begin{tikzpicture} [node distance=2cm, every state/.style={thick, minimum size=16pt}]
  \node[state, initial] (i) {$q_0$};
  \node[state, right of=i] (b) {$q_2$};
  \node[state, above of=b] (a) {$q_1$};
  \node[state, below of=b] (c) {$q_3$};
  \node[state, right of=b] (d) {$q_4$};
  \draw[loop above] (a) edge node {{\tt a}} (a);
  \draw[loop above] (b) edge node {{\tt b}} (b);
  \draw[loop above] (c) edge node {{\tt c}} (c);
  \draw (i) edge node[above, sloped] {$\emptystr$} (a);
  \draw (i) edge node {$\emptystr$} (b);
  \draw (i) edge node[below, sloped] {$\emptystr$} (c);
  \draw (a) edge node[above, sloped] {$\emptystr$} (d);
  \draw (b) edge node {$\emptystr$} (d);
  \draw (c) edge node[below, sloped] {$\emptystr$} (d);

  \node[state, right of=d] (a2) {$q_5$};
  \node[state, right of=a2] (b2) {$q_6$};
  \node[state, right of=b2, accepting] (c2) {$q_7$};
  \draw (d) edge node {{\tt a}} (a2);
  \draw (a2) edge node {{\tt b}} (b2);
  \draw (b2) edge node {{\tt b}} (c2);
\end{tikzpicture}
\]

Luego, calculamos las clausuras-$\emptystr$ para llevar el $\emptystr$-NFA a un
NFA:\@

\begin{center}
\begin{tabular}{rc}
  & Clausura-$\emptystr$  \\
  \toprule
  $q_0$ & $\{q_0, q_1, q_2, q_3, q_4\}$ \\
  $q_1$ & $\{q_1, q_4\}$ \\
  $q_2$ & $\{q_2, q_4\}$\\
  $q_3$ & $\{q_3, q_4\}$\\
  $q_4$ & $\{q_4\}$\\
  $q_5$ & $\{q_5\}$\\
  $q_6$ & $\{q_6\}$\\
  $q_7$ & $\{q_7\}$
\end{tabular}
\hfill
\begin{tabular}{rccc}
  & {\tt a} & {\tt b} & {\tt c} \\
  \toprule
  $\to q_0$ & $\{q_1, q_4, q_5\}$ & $\{q_2, q_4\}$ & $\{q_3, q_4\}$ \\
  $q_1$ & $\{q_1, q_4, q_5\}$ & $\emptyset$ & $\emptyset$\\
  $q_2$ & $\{q_5\}$ & $\{q_2, q_4\}$ & $\emptyset$\\
  $q_3$ & $\{q_5\}$ & $\emptyset$ & $\{q_3, q_4\}$ \\
  $q_4$ & $\{q_5\}$ & $\emptyset$  & $\emptyset$\\
  $q_5$ & $\emptyset$ & $\{q_6\}$ & $\emptyset$\\
  $q_6$ & $\emptyset$ & $\{q_7\}$ & $\emptyset$\\
  $\prescript{*}{}{q_7}$ & $\emptyset$ & $\emptyset$ & $\emptyset$
\end{tabular}
\end{center}

En base a esto, podemos obtener un DFA equivalente:

\begin{center}
\begin{tabular}{rccc}
  & {\tt a} & {\tt b} & {\tt c} \\
  \toprule
  $\to Q_0$ & $Q_{145}$ & $Q_{24}$ & $Q_{34}$ \\
  $Q_{145}$ & $Q_{145}$ & $Q_6$ & $\emptyset$ \\
  $Q_{24}$ & $Q_5$ & $Q_{24}$ & $\emptyset$ \\
  $Q_{34}$ & $Q_5$ & $\emptyset$ & $Q_{34}$ \\
  $Q_5$ & $\emptyset$ & $Q_6$ & $\emptyset$ \\
  $Q_6$ & $\emptyset$ & $Q_7$ & $\emptyset$ \\
  $\prescript{*}{}{Q_7}$ & $\emptyset$ & $\emptyset$ & $\emptyset$
\end{tabular}
\end{center}
El cual tiene el siguiente diagrama de estados:
\[
\begin{tikzpicture} [node distance=2.4cm, every state/.style={thick, minimum
  size=31pt}, every edge/.style={draw,->, >=stealth, auto, semithick,font=\tt}]
  \node[state, initial] (i) {$Q_0$};
  \node[state, right of=i] (q24) {$Q_{24}$};
  \node[state, below of=q24] (q34) {$Q_{34}$};
  \node[state, right of=q24] (q5) {$Q_5$};
  \node[state, above of=q24] (q145) {$Q_{145}$};

  \node[state, right of=q5] (q6) {$Q_6$};
  \node[state, right of=q6, accepting] (q7) {$Q_7$};

  \draw (i) edge[out=90, in=180] node[sloped] {a} (q145);
  \draw (i) edge node {b} (q24);
  \draw (i) edge[out=270, in=180] node[sloped, below] {c} (q34);

  \draw[loop above] (q145) edge node {a} (q145);
  \draw (q145) edge[out=0, in=90] node[sloped] {a} (q6);

  \draw (q24) edge node {a} (q5);
  \draw[loop above] (q24) edge node {b} (q24);

  \draw (q34) edge[out=0, in=270] node[sloped, below] {a} (q5);
  \draw[loop above] (q34) edge node {c} (q34);

  \draw (q5) edge node {a} (q6);
  \draw (q6) edge node {b} (q7);
\end{tikzpicture}
\]

Se obvió el estado trampa $\emptyset$ para simplificar el diagrama, pero considerar
que cualquier transición faltante en el diagrama implica una transición al estado $\emptyset$.

Finalmente, minimizamos el DFA usando el algoritmo de Moore: 
\begin{align*}
  \frac{Q}{E_0} &= \{\{Q_0, Q_{24}, Q_{34}, \emptyset, Q_{145}, Q_5, Q_6\}, \{Q_7\}\}\\
  \frac{Q}{E_1} &= \{\{Q_0, Q_{24}, Q_{34}, \emptyset, Q_{145}, Q_5\}, \{Q_6\}, \{Q_7\}\}\\
  \frac{Q}{E_2} &= \{\{Q_0, Q_{24}, Q_{34}, \emptyset\}, \{Q_{145}, Q_5\}, \{Q_6\}, \{Q_7\}\}\\
  \frac{Q}{E_3} &= \{\{Q_0, Q_{24}, Q_{34}\}, \{\emptyset\}, \{Q_{145}\} , \{Q_5\}, \{Q_6\}, \{Q_7\}\}\\
  \frac{Q}{E_4} &= \{\{Q_0\}, \{Q_{24}\}, \{Q_{34}\}, \{\emptyset\}, \{Q_{145}\} , \{Q_5\}, \{Q_6\}, \{Q_7\}\}
\end{align*}
No es posible seguir subdividiendo las particiones, por lo tanto, el DFA mínimo
resulta ser el que habíamos encontrado en el paso anterior.

\end{document}

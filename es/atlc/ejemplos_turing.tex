\documentclass{article}

\usepackage[utf8]{inputenc}
\usepackage{graphicx}
\usepackage{hyperref}
\renewcommand*{\figureautorefname}{Figura}

\usepackage{caption}

\usepackage{float}
\usepackage{epigraph}
\usepackage{multicol}
\usepackage[bottom,flushmargin]{footmisc} 

\usepackage{adjustbox}
\usepackage[spanish]{babel}
\usepackage{tikz}
\usetikzlibrary{babel} % fix problem with babel in spanish and tikz arrows

\usepackage[spanish=spanish]{csquotes}
\DeclareQuoteStyle{mystyle}
{\guillemotleft}{\guillemotright}
{\textquoteleft}{\textquoteright}
\setquotestyle{mystyle}

\usetikzlibrary{decorations.text}
\usetikzlibrary{decorations.pathreplacing}
\usetikzlibrary{decorations.pathmorphing, calc}
\usetikzlibrary{chains}
\newcommand{\emptystr}{\varepsilon}
\newcommand{\lp}{\text{\tt (}}
\newcommand{\rp}{\text{\tt )}}
\newcommand{\X}{\text{\tt X}}
\newcommand{\Y}{\text{\tt Y}}
\newcommand{\N}{\text{\tt N}}
\newcommand{\blank}{\text{\tt \textvisiblespace}}

\usetikzlibrary{automata, positioning, arrows,
fit, % for the dashed boxes on Thompson's construction
calc
}
\tikzset{%
  node distance=3cm, % specifies the minimum distance between two nodes. Change if necessary.
  every state/.style={thick}, % sets the properties for each ’state’ node
  double distance=2.5pt,
  shorten >= 2pt, shorten <= 2pt,
  initial text=$ $,
  every edge/.style={%
    draw,->, >=stealth, auto, semithick
  }
}

\usepackage{hyperref}
\hypersetup{pdfauthor={Cristian Adrián Ontivero}}
\usetikzlibrary{shapes}
\graphicspath{{imgs/}}

\newlength\tindent
\setlength{\tindent}{\parindent}
\setlength{\parindent}{0pt}
\renewcommand{\indent}{\hspace*{\tindent}}

%These tell TeX which packages to use.
\usepackage{array,epsfig}
\usepackage{amsmath}
\usepackage{amsfonts}
\usepackage{amssymb}
\usepackage{amsxtra}
\usepackage{amsthm}
\usepackage{mathrsfs}

%Here I define some theorem styles and shortcut commands for symbols I use often
\theoremstyle{definition}
\newtheorem{defn}{Definición}
\newtheorem{thm}{Teorema}
\newtheorem{cor}{Corolario}
\newtheorem*{rmk}{Remark}
\newtheorem{lem}{Lema}
\newtheorem*{joke}{Joke}

\newtheorem{ex}{Ejemplo}
\newcommand{\exautorefname}{Ejemplo}

\newtheorem{exercise}{Ejercicio}
\newcommand{\exerciseautorefname}{Ejercicio}

\newtheorem{soln}{Solución}
\newtheorem{prop}{Proposición}

%Pagination stuff.
\setlength{\topmargin}{-.3 in}
\setlength{\oddsidemargin}{0in}
\setlength{\evensidemargin}{0in}
\setlength{\textheight}{9.in}
\setlength{\textwidth}{6.5in}
\pagestyle{empty}

\def\isodate{\leavevmode\hbox{\the\year-\twodigits\month-\twodigits\day}}
\def\twodigits#1{\ifnum#1<10 0\fi\the#1}
\begin{document}
\begin{center}
  {\LARGE Máquinas de Turing}\\[.2cm]
  Cristian Adrián Ontivero \\[.05cm]%
  \isodate%
\end{center}

Formalmente, una máquina de Turing es una tupla $(Q, \Sigma, \Gamma, \blank, \delta, q_0, F)$, donde:
\begin{itemize}
  \setlength\itemsep{0pt}
   \item $Q$ es el conjunto finito de estados.
   \item $\blank \in \Gamma \setminus \Sigma$ representa la celda vacía o \textit{blank}.
   \item $\Gamma$ es el conjunto finito de símbolos de la cinta. 
   \item $\Sigma$ es el alfabeto de símbolos de entrada.
   \item $q_0 \in Q$ es el estado inicial.
   \item $\delta$ es la función de transición.
   \item $F \subset Q$ es el conjunto de estados finales.
 \end{itemize}

Para computar, una máquina de Turing se vale de cambios en su estado, en la
cinta, y en dónde apunta la cabecera. Se llama \textbf{configuración} de la
máquina de Turing a una terna de estas cosas. Dadas las cadenas $\alpha$ y
$\beta$ sobre el alfabeto $\Gamma$ de la cinta, escribimos $\alpha~q~\beta$ para
la configuración cuyo estado es $q$, los contenidos de la cinta son
$\alpha\beta$, y la cabecera apunta al primer símbolo de $\beta$. A la derecha
de $\beta$, la cinta esta vacía (consta sólo de \blank).

Informalmente, un procedimiento (\textit{procedure}) es una secuencia finita de
instrucciones que pueden ser llevadas a cabo mecánicamente. Cuando consideramos
una máquina de Turing como un procedimiento, decimos que el procedimiento es un
\textbf{algoritmo} si la TM se detiene (\textit{halts}) para toda entrada. 

\subsubsection*{Parity checker}
Para hacer una máquina que acepte cadenas binarias con paridad par (o peso de
Hamming par), basta con un DFA, pero en el siguiente ejemplo lo haremos con una
máquina de Turing.

La idea es construir una máquina de Turing que acepte una cadena binaria si la
cantidad de {\tt 1}s es par, y si no la rechace. Por ejemplo, dada la siguiente
cinta con el cabezal apuntando al comienzo de la cadena de entrada:

\begin{center}
 \begin{tikzpicture}
  \tikzset{tape/.style={minimum size=.5cm, draw, font=\tt}}
\begin{scope}[start chain=0 going right, node distance=0mm]
  \foreach \x [count=\i] in {0,1,1,0,\blank,\blank}{%
    \ifnum\i=6 % if last node reset outer sep to 0pt
      \node [on chain=0, tape, outer sep=0pt] (n\i) {\x};
      \draw (n\i.north east) -- ++(.1,0) decorate [decoration={zigzag, segment length=.12cm, amplitude=.02cm}] {-- ($(n\i.south east)+(+.1,0)$)} -- (n\i.south east) -- cycle;
     \else
        \ifnum\i=1
        \node [on chain=0, tape, outer sep=0pt] (n\i) {\x};
        \draw (n\i.north west) -- ++(-.1,0) decorate [decoration={zigzag,
        segment length=.12cm, amplitude=.2mm}] {-- ($(n\i.south west)+(-.1,0)$)}
        -- (n\i.south west) -- cycle;
        \else 
          \node [on chain=0, tape] (n\i) {\x};
        \fi
     \fi
   }
   %\node [right=.25cm of n11] {$\cdots$};
   \node [tape, below=5mm of n1] (q0) {$q_0$};
   \draw [->] (q0) to (n1); 
  \end{scope}  
 \end{tikzpicture}
\end{center}
Una solución simple sería recorrer la cinta partiendo de un estado inicial
$q_0$, y cambiar de estado al ver un $1$, digamos a un estado $q_1$. De esta
forma, $q_0$ indica paridad par, mientras que $q_1$ indica paridad impar. Si al
finalizar la cadena la máquina está en el estado $q_0$, acepta. Dado que la
definición de máquina de Turing vista en la cátedra requiere explícitamente de
un estado de aceptación, agregamos un estado $q_{\text{a}}$ en el que la
máquina entra si la cadena se acepta. Razonando con el ejemplo, llegamos a la
siguiente función de transición:
\begin{align*}
 \delta(q_0, 0) &= (q_0, X, R) \\
 \delta(q_0, 1) &= (q_1, X, R) \\
 \delta(q_1, 0) &= (q_1, X, R) \\
 \delta(q_1, 1) &= (q_1, X, R) \\
 \delta(q_0, \text{\blank}) &= (q_{\text{a}}, X, R)
\end{align*}

Toda transición no presente en $\delta$ causa que la máquina entre en un estado
de error, y por lo tanto no acepte la palabra, por lo que crear un estado
\enquote{trampa} es innecesario. El diagrama de estados de la máquina en cuestión se observa en la \autoref{fig:state_diagram}.

\begin{figure}[ht] % ’ht’ tells LaTeX to place the figure ’here’ or at the top of the page
\centering % centers the figure
\begin{tikzpicture}[node distance=3cm, every edge/.style={draw,->, >=stealth, auto, semithick }]
    \node[state, initial] (q0) {$q_0$};
    \node[state, right of=q0] (q1) {$q_1$};
    \node[state, accepting, below of=q0] (qa) {$q_{\text{a}}$};
    \draw (q0) edge[bend left] node {1/X,R} (q1);
    \draw (q1) edge[bend left] node {1/X,R} (q0);
    \draw (qa) edge[<-] node {\blank/X,R} (q0);
    \draw (q0) edge[loop above] node {0/X,R} (q0);
    \draw (q1) edge[loop above] node {0/X,R} (q1);
  \end{tikzpicture}
\caption{Diagrama de estados.}\label{fig:state_diagram}
\end{figure}

La máquina de Turing pasa por las siguientes configuraciones de la
\autoref{fig:configurations}:

\begin{figure}[ht] % ’ht’ tells LaTeX to place the figure ’here’ or at the top of the page
\centering % centers the figure
{%
\setlength\tabcolsep{0mm}
\tt
 \begin{tabular}{cccccc}
   $q_0$ & 0   & 1 & 1 & 0 & \blank \\
   X   & $q_0$ & 1 & 1 & 0 & \blank \\
   X   & X & $q_1$ & 1 & 0 & \blank \\
   X   & X & X & $q_0$ & 0 & \blank \\
   X   & X & X & X & $q_0$ & \blank \\
   X   & X & X & X & X & $q_{\text{a}}$\\
 \end{tabular}
 }
\caption{Configuraciones por las que transiciona la TM.}\label{fig:configurations}
 \end{figure}

Formalmente, definimos la máquina de Turing del ejemplo como una 7-tupla $(Q,
\Sigma, \Gamma, \blank, q_0, \delta, F)$ donde:
\begin{multicols}{2}
\raggedcolumns%
\begin{itemize}
  \setlength\itemsep{0pt}
   \item $Q = \{q_0, q_1, q_{\text{a}}\}$
   \item $\Sigma = \{{\tt 0,1}\}$
   \item $\Gamma = \{{\tt 0,1, X, \blank}\}$
   \item $\blank \in \Gamma \setminus \Sigma$, el espacio en
     blanco de la cinta
   \item $q_0$ es el estado inicial
   \item $\delta$, la función de transición previamente definida
   \item $F = \{ q_{\text{a}} \}$
 \end{itemize}
\end{multicols}

\subsubsection*{Parenthesis checker}

Se conoce como lenguaje de Dyck (\textit{Dyck Language})\footnote{Por Walther
von Dyck, un matemático alemán.} al lenguaje de
paréntesis bien balanceados. Es un CFL, por lo que basta con un PDA para
reconocerlo, pero diseñaremos una TM para la misma tarea. Para esto, nos
basaremos en un ejemplo. Supongamos que tenemos la siguiente cinta:

\begin{center}
 \begin{tikzpicture}
  \tikzset{tape/.style={minimum size=.5cm, draw, font=\tt}}
\begin{scope}[start chain=0 going right, node distance=0mm]
  \foreach \x [count=\i] in {\blank, {(}, {(}, {)}, {(}, {)}, {)},
    {(}, {)}, \blank}{%
    \ifnum\i=10 % if last node reset outer sep to 0pt
      \node [on chain=0, tape, outer sep=0pt] (n\i) {\x};
      \draw (n\i.north east) -- ++(.1,0) decorate [decoration={zigzag, segment length=.12cm, amplitude=.02cm}] {-- ($(n\i.south east)+(+.1,0)$)} -- (n\i.south east) -- cycle;
     \else
        \ifnum\i=1
        \node [on chain=0, tape, outer sep=0pt] (n\i) {\x};
        \draw (n\i.north west) -- ++(-.1,0) decorate [decoration={zigzag, segment length=.12cm, amplitude=.2mm}] {-- ($(n\i.south west)+(-.1,0)$)} -- (n\i.south west) -- cycle;
        \else 
          \node [on chain=0, tape] (n\i) {\x};
        \fi
     \fi
   }
   \node [tape, below=5mm of n2] (q0) {$q_0$};
   \draw [->] (q0) to (n2); 
  \end{scope}  
 \end{tikzpicture}
\end{center}

La idea es buscar un paréntesis derecho, marcarlo con \X, buscar el correspondiente
izquierdo, marcarlo, etc., \enquote{zigzageando} en la cinta. Al llegar a la primera celda en
blanco, sabemos que se agotaron los \rp, en cuyo caso barremos a izquierda. Si
hallamos una celda en blanco, vamos al estado de \textit{halt} $q_h$ y marcamos
{\tt Y} en la cinta para indicar que está balanceado. Si encontramos un \lp,
habían más paréntesis izquierdos que derechos, por lo que vamos a $q_h$ y
marcamos \N. Con esto en mente, llegamos a la siguiente función de transición:

\begin{center}
 \begin{tabular}{c@{\hskip 36pt}c@{\hskip 36pt}c}
   $\delta(q_0, \lp) = (q_0, \lp, R)$ & $\delta(q_1, \lp) = (q_0, \X, R)$ & $\delta(q_2, \X) = (q_2, \X, L)$ \\
   $\delta(q_0, \rp) = (q_1, \X, L)$  & $\delta(q_1, \X) = (q_1, \X, L)$ & $\delta(q_2, \blank) = (q_h, \Y, R)$ \\
   $\delta(q_0, \X)  = (q_0, \X, R)$  & $\delta(q_1, \blank) = (q_h, \N, R)$ & $\delta(q_2, \lp) = (q_h, \N, R)$\\ 
   $\delta(q_0, \blank) = (q_2, \blank, L)$ & & 
 \end{tabular}
\end{center}

El diagrama de estados de la máquina resultante se observa en la
\autoref{fig:state_diagram2}.
\begin{figure}[ht] % ’ht’ tells LaTeX to place the figure ’here’ or at the top of the page
\centering % centers the figure
\begin{tikzpicture}[node distance=2.8cm, every edge/.style={draw,->, >=stealth, auto, semithick }]
    \node[state, initial above] (q0) {$q_0$};
    \node[state, right of=q0] (q1) {$q_1$};
    \node[state, below of=q0] (q2) {$q_2$};
    \node[state, accepting, right of=q2] (qa) {$q_h$};

    \draw (q0) edge[out=195, in=165, looseness=8] node[text width=1cm] {\X/\X,R \\ \lp/\lp,R} (q0);
    \draw (q1) edge[out=15, in=345, looseness=8] node {\X/\X,L} (q1);
    \draw (q2) edge[out=195, in=165, looseness=8] node {\X/\X,L} (q2);

    \draw (q0) edge node {\rp/\X,L} (q1);
    \draw (q2) edge node[text width=1cm] {\blank/\Y,R \\ \lp/\N,R} (qa);

    \draw (q2) edge[<-] node {\blank/\blank, L} (q0);
    \draw (q1) edge node {\blank/\N, L} (qa);
  \end{tikzpicture}
  \caption{Diagrama de estados del \textit{parenthesis checker}.}\label{fig:state_diagram2}
\end{figure}

\subsection*{Notas finales}
Ambos ejemplos fueron sacados casi verbatim de la clase 26 del IIT Madras, por
la Dr.\ Kamala Krithivasan.
%\newpage 
\bibliography{refs}
\bibliographystyle{alpha}

\end{document}



\documentclass[spanish]{article}

\usepackage{color}
\definecolor{darkblue}{RGB}{49,130,189}

% Access modes
\newcommand{\acmo}[1]{\underline{\texttt{#1}}}

% Print today's date in ISO format (YYYY-MM-DD).
\def\isodate{\leavevmode\hbox{\the\year-\twodigits\month-\twodigits\day}}
\def\twodigits#1{\ifnum#1<10 0\fi\the#1}


\usepackage{skak}
\newcommand{\cd}{\text{CD}}
\newcommand{\coi}{\text{COI}}

\usepackage{calculation}

\usepackage[utf8]{inputenc}
\usepackage[es-tabla]{babel}
\usepackage{graphicx}
\usepackage{caption}
\usepackage{float}
\usepackage{epigraph}
\usepackage{multicol}
\usepackage[bottom,flushmargin]{footmisc} 

\usepackage{tikz}
\usetikzlibrary{arrows.meta}
\usetikzlibrary{babel}
\usetikzlibrary{decorations.pathreplacing}
\usetikzlibrary{positioning}

\usepackage{hyperref}
\hypersetup{pdfauthor={Cristian Adrián Ontivero}}
\usetikzlibrary{shapes}
\graphicspath{{imgs/}}

\newlength\tindent%
\setlength{\tindent}{\parindent}
\setlength{\parindent}{0pt}
\renewcommand{\indent}{\hspace*{\tindent}}

\renewcommand*{\tableautorefname}{Tabla}
\renewcommand*{\figureautorefname}{Figura}

%These tell TeX which packages to use.
\usepackage{array,epsfig}
\usepackage{amsmath}
\usepackage{amsfonts}
\usepackage{amssymb}
\usepackage{amsxtra}
\usepackage{amsthm}
\usepackage{mathrsfs}

\newtheorem{ex}{Ejemplo}
\newcommand{\exautorefname}{Ejemplo}

\newtheorem{exercise}{Ejercicio}
\newcommand{\exerciseautorefname}{Ejercicio}

\newtheorem{soln}{Solución}
\newtheorem{prop}{Proposición}

%Pagination stuff.
\setlength{\topmargin}{-.3 in}
\setlength{\oddsidemargin}{0in}
\setlength{\evensidemargin}{0in}
\setlength{\textheight}{9.in}
\setlength{\textwidth}{6.5in}
\pagestyle{empty}

\begin{document}
\begin{center}
  {\LARGE Muralla China: Modelo Híbrido}\\[.2cm]
  Cristian Adrián Ontivero \\[.05cm]%
\end{center}

\vspace{0.2 cm}

El modelo Muralla China fue descrito por F.C. Brewer y Michael J. Nash en 1989,
quienes identificaron este en políticas usadas en ambientes financiaros~\cite{bn89}.
La finalidad del modelo es prevenir flujos que causen conflictos de interés. Por
ejemplo, un asesor de un banco no debería tener acceso a información de
otro banco (para evitar uso indebido de esta con el fin de beneficiar uno sobre
otro).

%``Muralla China'' (\textit{chinese wall policy}) por sus autores, un término que
%ha tenido sus detractores a través de los años.

En este modelo, a un sujeto se le restringe la información en base a qué
información accedió previamente, en contraste con Bell-La~Padula, donde se
restringe en base a clasificaciones. Por esto, se dice que la Muralla China
posee un aspecto dinámico (o temporal).

Para poder establecer comparaciones con BLP, los autores usaron muchos
términos iguales. Así, el modelo se refiere a personas como sujetos. También
cuenta con objetos, pero se basa en una organización jerárquica de estos:
\begin{itemize}
  \itemsep 0cm
      \item En el nivel inferior se tiene ítems de información, cada uno
        correspondiendo a una compañía. Se refiere a estos simplemente como
        objetos.
      \item En el nivel intermedio, se agrupan objetos asociados a una misma
        compañía. Estas asociaciones se conocen como \textit{company datasets}
        (CDs).
      \item En el nivel superior, se agrupan los CDs cuyas compañías compiten, y
        se refiere a cada una de estas agrupaciones como clases de conflicto de
        interés (\textit{conflict of interest class}).
\end{itemize}
Definido esto, el modelo presenta una serie de axiomas, donde los más
importantes son los siguientes dos:
\paragraph{Regla de lectura BN} Un sujeto $s$ puede leer un objeto $o$ sólo si alguno de
los siguientes se cumplen:
\begin{itemize}
  \itemsep 0cm
  \item $o$ está en el mismo CD que otro objeto previamente leido por $s$.
  \item $o$ pertenece a un COI dentro del cual $s$ no ha leido ningún objeto.
\end{itemize}
Esta regla a veces se conoce como principio de seguridad simple, por analogía
con BLP.\@
\paragraph{Regla de escritura BN} Un sujeto $s$ puede escribir un objeto $o$ sólo si:
\begin{itemize}
  \itemsep 0cm
  \item $s$ puede leer $o$ bajo la regla de lectura BN, y
  \item $s$ no puede leer ningún objeto que esté en un CD distinto al CD en
    donde solicita acceso de escritura.
\end{itemize}
Esta regla a veces se conoce como principio-*. La segunda regla en Brewer-Nash
tiene por finalidad evitar que un troyano pueda filtrar información, prohibiendo
el flujo indebido de información, al igual que Bell-La~Padula y Biba con sus
propiedades-*. Esto se observa en la \autoref{fig:BNstar}. En esta, primero dos
sujetos $s_1$ y $_s2$ leen objetos dentro de dos CDs distintos, que estan en un
mismo COI.\@ Paso siguiente, $s_1$, infectado por un troyano, escribe la
información de $o_1$ en un objeto $o_3$ dentro de un COI distinto.  Finalmente,
$s_2$ aprovecha esto leyendo $o_3$, efectivamente logrando obtener información
del $\cd_1$ troyano corriendo bajo el sujeto $s_2$ escribe.

\newcommand{\circled}[1]{{\textcircled{\scriptsize{#1}}}}
\begin{figure}[h!]
  \centering 
  \begin{minipage}{.96\textwidth}
    \centering
  \begin{tikzpicture}
    \node (COI1) at (2.2,3.5) {$\text{COI}_1$};
    \draw [thick,color=darkblue] (-.5,.5) rectangle (4.9, 3.9);

    \node (COI2) at (8.1,3.5) {$\text{COI}_2$};
    \draw [thick,color=darkblue] (5.4,.5) rectangle (10.8, 3.9);

    \node (CD1) at (1,2.7) {$\text{CD}_1$};
    %\node (o1) at (1,2) {$o_1$};
    \node[circle,fill,inner sep=1pt,outer sep=2pt, label=above:$o_1$] (o1) at
    (.5,1.5) {};
    \node[circle,fill,inner sep=1pt,outer sep=2pt] (o4) at (1,2) {};
    \node[circle,fill,inner sep=1pt,outer sep=2pt] (o5) at (1.5,1.25) {};
    \draw [thick,dashed,color=darkblue] (0,1) rectangle (2, 3.2);

    \node (CD2) at (3.4,2.7) {$\text{CD}_2$};
    \node[circle,fill,inner sep=1pt,outer sep=2pt, label=above:$o_2$] (o2) at (2.7,1.2) {};
    \node[circle,fill,inner sep=1pt,outer sep=2pt] (o6) at (3.5,1.4) {};
    \node[circle,fill,inner sep=1pt,outer sep=2pt] (o7) at (3.9,1.9) {};
    \draw [thick,dashed,color=darkblue] (2.4,1) rectangle (4.4, 3.2);

    \node (CD3) at (6.9,2.7) {$\text{CD}_3$};
    \draw [thick,dashed,color=darkblue] (5.9,1) rectangle (7.9, 3.2);
    \draw [thick,dashed,color=darkblue] (8.3,1) rectangle (10.3, 2);
    \draw [thick,dashed,color=darkblue] (8.3,2.2) rectangle (10.3, 3.2);


    \node[circle,fill,inner sep=1pt,outer sep=2pt, label=above:$o_3$] (o3) at
    (7.6,1.4) {};
    \node[shape=circle,draw,inner sep=2pt,outer sep=2pt] (s2) at (5.2,0) {$s_2$};
    \node[shape=circle,draw,inner sep=2pt,outer sep=2pt] (s1) at (5.2,-1.2) {$s_1$};
    \node[color=red,below right=-2.5mm of s1] {\symknight};
    \draw [<-,thick] (s1) to [out=180,in=270] node[below,sloped,outer sep=1mm] {\circled{2}~\acmo{r}} (o1);
    %\draw [<-] (s2) to node[below,sloped] {\acmo{r}} (o6);
    \draw [->,thick] (s1) to [out=0,in=300] node[below,sloped,outer sep=1mm] {\circled{3}~\acmo{w}} (o3);
    \draw [<-,thick] (s2) to [out=0,in=240] node[below,sloped,pos=.3,outer sep=1mm] {\circled{4}~\acmo{r}} (o3);
    \draw [<-] (s2) to [out=180,in=270] node[below,sloped,outer sep=1mm] {\circled{1}~\acmo{r}} (o2);

    \node[circle,fill,inner sep=1pt,outer sep=2pt] at (8.6,1.2) {};
    \node[circle,fill,inner sep=1pt,outer sep=2pt] at (9.6,1.4) {};
    \node[circle,fill,inner sep=1pt,outer sep=2pt] at (9.2,1.7) {};

    \node[circle,fill,inner sep=1pt,outer sep=2pt] at (9.9,2.4) {};
    \node[circle,fill,inner sep=1pt,outer sep=2pt] at (9.1,2.9) {};
    \node[circle,fill,inner sep=1pt,outer sep=2pt] at (8.8,2.5) {};

  \end{tikzpicture}
  \caption{Flujo de información entre dos CDs dentro de un mismo COI\@. El
    sujeto $s_1$, infectado por un troyano (\symknight), escribe en $o_3$, y esto es
aprovechado por $s_2$. Este escenario motiva la propiedad-*, que lo
prohibe.\label{fig:BNstar}}
  \end{minipage}
\end{figure}


Al proponer su modelo, Brewer y Nash concluyen la política Muralla China ``no
puede ser correctamente representada por el modelo Bell-La~Padula''~\cite{bn89}.
Posteriormente Ravi S. Sandhu argumentó que el modelo como propuesto por Brewer y Nash es
demasiado restrictivo para ser práctico, producto del error que cometen sus
autores de no distinguir entre los términos usuario, principal, y sujeto (sólo
hablan de sujetos). 

Los usuarios son aquellos del mundo real (humanos). Los principales son unidades
de control de acceso y autorización. Existe una relación \textit{one-to-many}
entre el usuario y los principales, es decir, un usuario puede tener uno o más
principales, pero todo principal está asociado a un único usuario. Por último,
los sujetos son procesos (programas) ejecutando en lugar del principal. Un
principal puede no estar haciendo nada, o corriendo varios procesos en un mismo
momento. Cada sujeto usualmente (aunque no siempre) está asociado a un único
principal, y posee los permisos del principal que lo invocó. Por simplicidad, a
veces se pueden considerar a los principales y los sujetos como lo mismo, pero
siempre debería verse a los usuarios como multiples principales. Por ejemplo, en
la \autoref{fig:users} se tiene una usuario Amy, con distintos principales y
cada uno corriendo ciertos programas.
\begin{figure}[ht]
\[
  \texttt{amy}
  \begin{cases}
    \texttt{amy.top-secret} &
    \begin{cases}
      \text{Aplicación de email} \\
      \text{Editor de texto} \\
      \text{Base de datos} \\
    \end{cases}\\
    \texttt{amy.secret} &
    \begin{cases}
      \hspace{2mm}\vdots \\
    \end{cases}\\
    \texttt{amy.confidential} &
    \begin{cases}
      \hspace{2mm}\vdots \\
    \end{cases}\\
    \texttt{amy.unclassified} &
    \begin{cases}
      \text{Navegador web} \\
    \end{cases}\\
  \end{cases}\\
\] 
\caption{Amy.\label{fig:users}}
\end{figure}
Habiendo explicitado las diferencias entre estos términos, Sandhu propuso un
modelo menos restrictivo, con el cual sí es posible modelar una política de
Muralla China con Bell-La~Padula, ya que una muralla china no es más que otra
política basada en un retículo~\cite{Sandhu92,Sandhu92b,Sandhu93}.

\begin{enumerate}
  \item Un usuario $u$ puede leer un objeto $o$ sólo si $u$ nunca leyó un objeto $o'$
tal que:
  \begin{enumerate}
    \item $\coi(o) = \coi(o')$
    \item $\cd(o) \neq \cd(o')$
  \end{enumerate}

\item Un sujeto $s$ asociado a un usuario $u$ puede leer un objeto $o$ si $u$
  puede leer $o$.
\item Un sujeto $s$ puede escribir $o$ sólo si:
  \begin{enumerate}
    \item $s$ puede leer $o$, y
    \item $s$ nunca leyó un objeto $o'$ tal que $\cd(o) \neq \cd(o')$.
  \end{enumerate}
\end{enumerate}

Estas condiciones requieren llevar un registro del conjunto de objetos leidos
por cada usuario y sujeto.
Las primeras dos condiciones aseguran que un usuario nunca pueda violar la
muralla leyendo información de dos CDs diferentes dentro de un mismo COI\@.
La tercera condición asegura que dos o más usuarios no puedan violar la muralla
realizando una secuencia de operaciones de lectura y escritura\@.

%by performing a series of read and write operations.  Suppose that S1 has
%previously read from CD1, and S2 has previously read from CD2.  Consider the
%following sequence of operations, based on the figure above.

%    S1 reads information from an object in CD1.
%
%    S1 writes that information to object O6 in CD3.
%
%    S2 reads that information from O6.
%
%At the end of this sequence, S2 would have read information pertaining to both CD1 and CD2, which would violate the Chinese Wall policy since both CDs are in the same COI.  But Condition 3b prevents the write operation by restricting when a subject may write:  once a subject reads two objects from different CDs, that subject may never write any object.  So for read--write access, a user must create a distinct subject for each CD.  For read-only access, a user can create a single subject to read from several COIs.  
%
%
%Notar que H no es una matriz de control de acceso; no refleja los
%permisos permitidos, si no los otorgados.
%
%Axioma 1: $\cd_1 = \cd_2 \implies \coi_1 = \coi_2$.
%Por contraposición lógica, tenemos el siguiente corolario:
%Corolario 1: $\coi_1 \neq \coi_2 \implies \cd_1 \neq \cd_2$.
%




\bibliography{refs}
\bibliographystyle{unsrt}

\end{document}

\documentclass[]{book}

\usepackage[utf8]{inputenc}
\usepackage[yyyymmdd,hhmmss]{datetime}
%These tell TeX which packages to use.
\usepackage{array,epsfig}
\usepackage{amsmath}
\usepackage{amsfonts}
\usepackage{amssymb}
\usepackage{amsxtra}
\usepackage{amsthm}
\usepackage{mathrsfs}
\usepackage{color}

%Here I define some theorem styles and shortcut commands for symbols I use often
\theoremstyle{definition}
\newtheorem{defn}{Definition}
\newtheorem{thm}{Theorem}
\newtheorem{cor}{Corollary}
\newtheorem*{rmk}{Remark}
\newtheorem{lem}{Lemma}
\newtheorem*{joke}{Joke}
\newtheorem{ex}{Example}
\newtheorem*{soln}{Solution}
\newtheorem{prop}{Proposition}

\newcommand{\lra}{\longrightarrow}
\newcommand{\ra}{\rightarrow}
\newcommand{\surj}{\twoheadrightarrow}
\newcommand{\graph}{\mathrm{graph}}
\newcommand{\bb}[1]{\mathbb{#1}}
\newcommand{\Z}{\bb{Z}}
\newcommand{\Q}{\bb{Q}}
\newcommand{\R}{\bb{R}}
\newcommand{\C}{\bb{C}}
\newcommand{\N}{\bb{N}}
\newcommand{\M}{\mathbf{M}}
\newcommand{\m}{\mathbf{m}}
\newcommand{\MM}{\mathscr{M}}
\newcommand{\HH}{\mathscr{H}}
\newcommand{\Om}{\Omega}
\newcommand{\Ho}{\in\HH(\Om)}
\newcommand{\bd}{\partial}
\newcommand{\del}{\partial}
\newcommand{\bardel}{\overline\partial}
\newcommand{\textdf}[1]{\textbf{\textsf{#1}}\index{#1}}
\newcommand{\img}{\mathrm{img}}
\newcommand{\ip}[2]{\left\langle{#1},{#2}\right\rangle}
\newcommand{\inter}[1]{\mathrm{int}{#1}}
\newcommand{\exter}[1]{\mathrm{ext}{#1}}
\newcommand{\cl}[1]{\mathrm{cl}{#1}}
\newcommand{\ds}{\displaystyle}
\newcommand{\vol}{\mathrm{vol}}
\newcommand{\cnt}{\mathrm{ct}}
\newcommand{\osc}{\mathrm{osc}}
\newcommand{\LL}{\mathbf{L}}
\newcommand{\UU}{\mathbf{U}}
\newcommand{\support}{\mathrm{support}}
\newcommand{\AND}{\;\wedge\;}
\newcommand{\OR}{\;\vee\;}
\newcommand{\Oset}{\varnothing}
\newcommand{\st}{\ni}
\newcommand{\wh}{\widehat}

%Pagination stuff.
\setlength{\topmargin}{-.3 in}
\setlength{\oddsidemargin}{0in}
\setlength{\evensidemargin}{0in}
\setlength{\textheight}{9.in}
\setlength{\textwidth}{6.5in}
\pagestyle{empty}



\begin{document}


\begin{center}
  {\Large Criptografía y Seguridad \the\year~(72.44)\\[.2cm]
Guía 1 - Criptografía Clásica}\\
%\textbf{NAME}\\ %You should put your name here
%Due: DATE %You should write the date here.
\end{center}

\vspace{0.2 cm}


\subsection*{Ejercios}
\begin{enumerate}
\item\label{norms}
  Dar una definición formal de los algoritmos \textbf{Gen} , \textbf{Enc} y
  \textbf{Dec} para los siguientes esquemas: 
\begin{enumerate}
  \item Cifrado de rotación.
  \item Cifrado de sustitución monoalfabética.
  \item Cifrado de Vigenère.
\end{enumerate}

\item ¿Por qué la composición de dos sistemas de sustitución simple no provee más seguridad que el uso de uno solo? Ejemplificar. 

\item Descifrar el siguiente criptograma, sabiendo que fue encriptado usando un
  cifrado de rotación, que se corresponde a un texto en español (27 letras), y
  los espacios fueron suprimidos.
  
\begin{center}
VKXYKBKXGKSGWAKQQGYIUYGYWAKXKGQRKSZKJKYKKYIUSYKMAÑX
\end{center}
¿Qué estrategia utilizaste? 

\item 
  \begin{enumerate}
    \item Cifrar según Vigenère el mensaje M = ``UN VINO DE MESA'' con la clave
      K = ``BACO'' sin usar la tabla, sólo con operaciones modulares.
    \item En un sistema de cifrado de Vigenère la clave a usar puede ser o bien CERO, o bien COMPADRE, ¿cuál de las dos conviene usar y por qué?
    \item Mostrar, con un ejemplo, que la composición de dos cifrados Vigenère resulta en otro cifrado Vigenère.
  \end{enumerate}

\item Teniendo en cuenta que la frecuencia de aparición de letras en español es la siguiente: 
  Decir, para cada criptograma, si se ha obtenido mediante técnicas de sustitución monoalfabética, sustitución polialfabética, o de transposición. 
  \begin{enumerate}
    \item Criptograma 1:
      \begin{center}
        \verb+KOZFVPCYVCWVZHMZLCIOHIFIZGJCZTVVXIGJLZHYZLGVMNVLYZ+
      \end{center}
    \item Criptograma 2:
\begin{verbatim}
      HHMBIWSIPSNNTAWVITQWMEAQVNSPGQJNWELXMJDIBYUGNNRMEUDEMZIBTMYMB
      MWURBTIZXNCWZIUPZUQNRMEGJLWRVROPMREUMXXXAXDIPUVFEASMBSASCETAE
      WOYYAKUSWEABSASCRECIOMEWTQOMYALMTXRAGEWSQQHJDXMVJEAFIRNDUIANW
\end{verbatim}
    \item Criptograma 3:
      \begin{center}
        \verb+DERTNYLANAOTAABADEAXCEEAIDEJLXHRSUAUJUMXELAATECRTRNAZBIRESOX+
      \end{center}
\end{enumerate}
\item Se recibe el siguiente criptograma:
  
\begin{verbatim}
JGAZN WINHY LZDYV BBJLC QHTNK UDQXM OXJNO ZMUSP NONYJ MTEJH QHQFO 
OPUPB CYAÑJ ONCNN QHNMO NDHKU TJMQC MOPNF AOXNT NLOAZ MJDQY MOZCJ 
RNBAO QTUIE NFAIX TLXJG AZMJA XJVAZ MUDNM YLNLJ MUMUY HVUMH TÑIGD 
XDQUC LSJPI BCUSF NUGXX GEEXK AEJME SJÑEN ASLHL BAEYJ ROJXA CQTCN 
MYPUC UNMJW OYNHZ NKUOG AJDUJ XENRY TENJS CNMON TYJNM JYFXF IGJMI 
BUUSN TFAPN FAFKU ROJNY CTUYN BYSGJ VACAU CGQWA ZMJJH JHSNT PAPXM 
GNECO GJUTE NCNGJ GEGAJ SPNUL GDMAÑ JDOFD NPUNN PNTGE NMJSN TTOFD 
KIOXS SQNNF BATOC XMMNV ÑEZNM EZBOS NTUSQ BUDBT JRBBU YPQZI OQFTB
ANIBV MEDDY RUMUP NAULB OMAED HVHNF OCJOS NMJ
\end{verbatim}

Si se conoce que ha sido cifrado mediante el algoritmo de Vigenère, se pide: 
  \begin{enumerate}
    \item Comprobar la longitud de la clave. 
    \item Encontrar la clave del sistema y desencriptar sólo los diez primeros caracteres. 
  \end{enumerate}
Para ello tener en cuenta lo siguiente: 

\begin{itemize}
  \item Listar todas las secuencias repetidas de por lo menos 3 caracteres, junto con la distancia a la que se encuentran. 
  \item Ayuda: Aparecen cuatro cadenas de cuatro caracteres que se repiten en el criptograma: 
      \begin{center}
      \verb+JGAZ+, \verb+NMON+, \verb+PNFA+ y \verb+AZMJ+
      \end{center}
  \item Estimar cuál puede ser la longitud y obtener la frecuencia de aparición de cada letra como primera de cada bloque. 
  \end{itemize}

\item Se cuenta con cifrado, producto de una transposición por columnas
  (cada ``n'' columnas se reacomodó el texto original) y un cifrado de rotación. 
  \begin{enumerate}
    \item ¿Qué estrategia usarías para recuperar el mensaje original? 
    \item Si el texto cifrado tiene ``m'' caracteres, ¿cuántas pruebas requeriría un ataque de fuerza bruta? 
  \end{enumerate}

\item Mostrar que los siguientes cifrados son fáciles de quebrar mediante un
  ataque de texto plano elegido (\textit{chosen-plaintext attack})
  \begin{enumerate}
    \item Cifrado de sustitución monoalfabética.
    \item Cifrado de Vigenère.
  \end{enumerate}

\item	Prove that $\ds \bigg| \, \|x\|-\|y\| \, \bigg| \leq \|x-y\|$.

\item	The quantity $\|y-x\|$ is called the \textdf{distance} between $x$ and
$y$.  Prove and interpret the ``triangle inequality'':
$$\|z-x\| \leq \|z-y\| + \|y-x\|.$$

\item\label{caushw}	Let $f$ and $g$ be integrable on $[a,b]$.
\begin{enumerate}
	\item	Prove the integral version of the Cauchy-Schwarz inequality:
	$$\left|\int_a^b fg\right| \leq \left(\int_a^b
	f^2\right)^{1/2}\left(\int_a^b g^2\right)^{1/2}.$$
	Hint:  Consider separately the cases $0 = \int_a^b(f-t g)^2$ for
	some $t\in\R$, and $0<\int_a^b(f-t g)^2$ for all
	$t\in\R$.
	\item	If equality holds, must $f=t g$ for some $t\in\R$? 
	What if $f$ and $g$ are continuous?
	\item	Show that the Cauchy-Schwarz inequality is a special case of
	(a).

\end{enumerate}

\item	A linear transformation $T:\R^n\lra\R^n$ is \textdf{norm preserving} if
$$\|T(x)\|=\|x\|,$$ for all $x\in\R^n$, and \textdf{inner product preserving} if
$$\ip{Tx}{Ty} = \ip xy,$$ for all $x,y\in\R^n$.
\begin{enumerate}
	\item	Prove that $T$  is norm preserving if and only if it is inner
	product preserving.
	
	\item	Prove that such a linear transformation is 1-1, and $T^{-1}$ is
	norm preserving (and inner product preserving).
\end{enumerate}

\item\label{bddlin}	If $T:\R^m\lra\R^n$ is a linear transformation, show that there is a
number $M$ such that $\|T(h)\|\leq M\|h\|$ for all $h\in\R^m$.  Hint: Estimate
$\|T(h)\|$ in terms of $\|h\|$ and the entries in the matrix for $T$.

\item	If $x,y\in\R^n$, and $z,w\in\R^m$, show that $\ds\ip{(x,z)}{(y,w)} = \ip xy
+ \ip zw$, and $\ds\|(x,z)\| = \sqrt{\|x\|^2 + \|z\|^2}$.  Note that $(x,z)$ and
$(y,w)$ denote points in $\R^{n+m}$.

\item	If $x,y\in\R^n$, then $x$ and $y$ are called \textdf{perpendicular} (or
\textdf{orthogonal}), and we write $x\perp y$, if $\ip xy = 0$.  If $x\perp y$,
prove that $\|x+y\|^2 = \|x\|^2 + \|y\|^2$.

\end{enumerate}


\subsection*{Exercises for Section~1.2: More Topology: Open and Closed Sets in
$\R^n$}
\begin{enumerate}
\item\label{topology}	Prove that the union of any (even infinite) number of open sets is open.
 Prove that the intersection of two (and hence of finitely many) open sets is
 open.  Give a counterexample for the intersection of infinitely many open sets.

\item\label{clAB}If $A\subset B\subset\R^n$, prove that
	$$\cl{A}\subset\cl{B},\quad\mbox{ and }\quad\inter{A}\subset\inter{B}.$$

\item	Prove that if $B$ is an open subset of $A$, then $B\subset \inter(A)$.  Note that this says that $\inter(A)$ is the largest open subset of $A$.

\item	\label{openset}Prove that the $n$-dimensional ball centered at $a$ of radius $r$, 
	$$\ds B^n(a;r) = \left\{x\in\R^n:\|x-a\|< r\right\}$$ is open. 

\item	Find the interior, exterior, and boundary of the sets:
	$$B^n = \left\{x\in\R^n : \|x\| \leq 1\right\},$$
	$$S^{n-1} = \left\{x\in\R^n : \|x\| = 1\right\},$$
	$$\Q^n = \left\{x\in\R^n : \mbox{ each } x^i\mbox{ is rational}\right\}.$$
\begin{soln}
	% Put your answers here.
\end{soln}


\item	If $A\subset[0,1]$ is the union of open intervals $(a_i,b_i)$ such that
each rational number in $(0,1)$ is contained in some $(a_i,b_i)$, show that
$\bd A = [0,1] - A$.

\item	If $A$ is a closed set that contains every rational number $r\in[0,1]$,
show that $[0,1]\subset A$.

\item	Graph generic open balls in $\R^2$ with respect to each of the ``non-Euclidean'' norms, $\|\cdot\|_1$ and $\|\cdot\|_\infty$. 
What shapes are they?
\begin{soln}
	% Put your figure and your write-up here.  See the website on how to
	% add a figure.
\end{soln}

\end{enumerate}



\end{document}



\documentclass{article}

\usepackage[utf8]{inputenc}
\usepackage{hyperref}
\hypersetup{pdfauthor={Cristian Adrián Ontivero}}
\usepackage[yyyymmdd,hhmmss]{datetime}
\usepackage[spanish,es-tabla]{babel}
\usepackage{graphicx}
\usepackage{caption}
\usepackage{float}
\usepackage{color}
\usepackage{epigraph}
\usepackage[hang,flushmargin]{footmisc} 
\usepackage{tikz}
\usetikzlibrary{shapes}
\graphicspath{{imgs/}}

\definecolor{darkblue}{RGB}{49,130,189}

\newlength\tindent
\setlength{\tindent}{\parindent}
\setlength{\parindent}{0pt}
\renewcommand{\indent}{\hspace*{\tindent}}

\renewcommand*{\tableautorefname}{Tabla}
\renewcommand*{\figureautorefname}{Figura}

%These tell TeX which packages to use.
\usepackage{array,epsfig}
\usepackage{amsmath}
\usepackage{amsfonts}
\usepackage{amssymb}
\usepackage{amsxtra}
\usepackage{amsthm}
\usepackage{mathrsfs}
\usepackage{color}

%Here I define some theorem styles and shortcut commands for symbols I use often
\theoremstyle{definition}
\newtheorem{defn}{Definición}
\newtheorem{thm}{Teorema}
\newtheorem{cor}{Corolario}
\newtheorem*{rmk}{Remark}
\newtheorem{lem}{Lema}
\newtheorem*{joke}{Joke}

\newtheorem{ex}{Ejemplo}
\newcommand{\exautorefname}{Ejemplo}

\newtheorem{exercise}{Ejercicio}
\newcommand{\exerciseautorefname}{Ejercicio}

\newtheorem{soln}{Solución}
\newtheorem{prop}{Proposición}

\newcommand{\lra}{\longrightarrow}
\newcommand{\ra}{\rightarrow}
\newcommand{\surj}{\twoheadrightarrow}
\newcommand{\graph}{\mathrm{graph}}
\newcommand{\bb}[1]{\mathbb{#1}}
\newcommand{\Z}{\bb{Z}}
\newcommand{\Q}{\bb{Q}}
\newcommand{\R}{\bb{R}}
\newcommand{\C}{\bb{C}}
\newcommand{\N}{\bb{N}}
\newcommand{\M}{\mathbf{M}}
\newcommand{\m}{\mathbf{m}}
\newcommand{\MM}{\mathscr{M}}
\newcommand{\HH}{\mathscr{H}}
\newcommand{\Om}{\Omega}
\newcommand{\Ho}{\in\HH(\Om)}
\newcommand{\bd}{\partial}
\newcommand{\del}{\partial}
\newcommand{\bardel}{\overline\partial}
\newcommand{\textdf}[1]{\textbf{\textsf{#1}}\index{#1}}
\newcommand{\img}{\mathrm{img}}
\newcommand{\ip}[2]{\left\langle{#1},{#2}\right\rangle}
\newcommand{\inter}[1]{\mathrm{int}{#1}}
\newcommand{\exter}[1]{\mathrm{ext}{#1}}
\newcommand{\cl}[1]{\mathrm{cl}{#1}}
\newcommand{\ds}{\displaystyle}
\newcommand{\vol}{\mathrm{vol}}
\newcommand{\cnt}{\mathrm{ct}}
\newcommand{\osc}{\mathrm{osc}}
\newcommand{\LL}{\mathbf{L}}
\newcommand{\UU}{\mathbf{U}}
\newcommand{\support}{\mathrm{support}}
\newcommand{\AND}{\;\wedge\;}
\newcommand{\OR}{\;\vee\;}
\newcommand{\Oset}{\varnothing}
\newcommand{\st}{\ni}
\newcommand{\wh}{\widehat}

%Pagination stuff.
\setlength{\topmargin}{-.3 in}
\setlength{\oddsidemargin}{0in}
\setlength{\evensidemargin}{0in}
\setlength{\textheight}{9.in}
\setlength{\textwidth}{6.5in}
\pagestyle{empty}



\begin{document}


\begin{center}
\Large Criptografía y Seguridad (72.44)\\[.05cm]
Control de Accesos \\[.05cm]
\the\year
\end{center}

\vspace{0.2 cm}
\section{Control de Accesos}
Normalmente en sistemas operativos primero se autentica a los principales
(mediante algún mecanismo como contraseñas, o Kerberos), y posteriormente se
media el acceso de estos a recursos, como pueden ser archivos, puertos, etc.
Estos controles de acceso pueden modelarse con una matriz de permisos de acceso,
donde las columnas se usan para archivos, y las filas para los usuarios.

Estas matrices de control de accesos se pueden utilizar tanto para modelar, como
para implementar mecanismos de protección, pero tienen la desventaja de no ser
escalables (por ejemplo, una multinacional con 40.000 empleados y unos 1.000 recursos
a los que mediar su acceso, requeriría una matriz de 40.000.000 de entradas). 

En general, es preferible usar métodos más compactos de almacenar y administrar
esta información, y para esto se suele comprimir los usuarios, o comprimir los
permisos.

Para comprimir usuarios, se suele utilizar grupos o roles para administrar los
privilegios de un conjunto de usuarios.

Para comprimir los permisos, existen dos formas de almacenar la matriz
de control de accesos:
\begin{enumerate}
  \item Por columnas, junto con el recurso al que la columna hace referencia.
    Esto se conoce como listas de control de accesos (ACL, pronunciadas
    ``ackle'' en inglés).
  \item Por filas, conocidas como capacidades (\textit{capabilities}, o en menor
    medida \textit{tickets}).
\end{enumerate}

\begin{table}[h!]
  \centering
\begin{tabular}{l|c|c|c} 
 Usuario & apuntes.tex & apuntes.pdf & script.sh \\ 
 \hline
 Alice   & \verb+rw+   & \verb+r+    & \verb+-+ \\ 
 Bob     & \verb+r+    & \verb+r+    & \verb+rx+ \\ 
 Charlie & \verb+-+    & \verb+-+    & \verb+rwx+ \\
\end{tabular}
\caption{Lista de control de accesos (ACL).}
\end{table}

En Unix (y Linux), los archivos no


Acceso Discrecional
Acceso Mandatorio

En Unix, write access no implica read access. Aplicados a un directorio,
significan:
- read: listar el contenido del directorio;
- write: crear o renombrar un archivo en el directorio;
- execute: buscar el directorio;

formal statement of your security policy = A security model.

Los PKI son una reimplementación de un viejo concepto de control de accesos:
capacidades.

Las MAC a veces surgen en otras areas bajo distintos nombres. Un ejemplo son los
empleados médicos, quienes tienen permiso para ver la historia clínica de un
paciente, pero no pueden transferir este permiso a empleados no médicos.

\iffalse
\section{Biba: Modelo de Integridad}
\subsection{Low-water mark}
\subsection{Política Anillo}
\subsection{Integridad Estricta}
\section{Muralla China: Modelo híbrido}
\fi

\bibliography{refs}
\bibliographystyle{unsrt}

\section*{Apéndice}
\begin{soln}
  \vspace{.1cm}
\begin{enumerate}
  \item El diagrama de Hasse del retículo formado por los niveles de seguridad y
    la relación $\mathsf{dom}$ se observa en \autoref{fig:lattice}. Las flechas
    representan la relación de dominancia, y pueden pensarse como el sentido del
    flujo de información.

\begin{figure}[htp]
  \centering
  \begin{tikzpicture}
    \node (TSNE) at (2,8) {(TS, \{N, E\})};
    \node (TSE)  at (6,7) {(TS, \{E\})};
    \node (TSN)  at (0,7) {(TS, \{N\})};

    \node (SNE) at (2,6) {(S, \{N, E\})};
    \node (SE)  at (6,5) {(S, \{E\})};
    \node (TS)  at (4,6) {(TS, $\emptyset$)};
    \node (SN)  at (0,5) {(S, \{N\})};

    \node (CNE) at (2,4) {(C, \{N, E\})};
    \node (CE)  at (6,3) {(C, \{E\})};
    \node (S)   at (4,4) {(S, $\emptyset$)};
    \node (CN)  at (0,3) {(C, \{N\})};

    \node (UNE) at (2,2) {(U, \{N, E\})};
    \node (UE)  at (6,1) {(U, \{E\})};
    \node (C)   at (4,2) {(C, $\emptyset$)};
    \node (UN)  at (0,1) {(U, \{N\})};
    \node (E)   at (4,0) {(U, $\emptyset$)};
    \draw[<-] (TSNE) -- (SNE);
    \draw[<-] (TSNE) -- (TSN);
    \draw[<-] (TSNE) -- (TSE);
    \draw[<-] (TSE) -- (SE);
    \draw[<-] (TSN) -- (SN);
    \draw[<-] (TSN) -- (TS);
    \draw[<-] (TSE) -- (TS);

    \draw[<-] (SNE) -- (CNE);
    \draw[<-] (SNE) -- (SN);
    \draw[<-] (SNE) -- (SE);
    \draw[<-] (SE) -- (CE);
    \draw[<-] (SN) -- (CN);
    \draw[<-] (TS) -- (S);
    \draw[<-] (SN) -- (S);
    \draw[<-] (SE) -- (S);

    \draw[<-] (CNE) -- (UNE);
    \draw[<-] (CNE) -- (CN);
    \draw[<-] (CNE) -- (CE);
    \draw[<-] (CE) -- (UE);
    \draw[<-] (CN) -- (UN);
    \draw[<-] (S) -- (C);
    \draw[<-] (CN) -- (C);
    \draw[<-] (CE) -- (C);

    \draw[<-] (UNE) -- (UN);
    \draw[<-] (UNE) -- (UE);
    \draw[<-] (UN) -- (E);
    \draw[<-] (UE) -- (E);
    \draw[<-] (C) -- (E);
  \end{tikzpicture}
  \caption{Retículo de niveles de seguridad.\label{fig:lattice} }
\end{figure}
  \item Sí, $L(\text{Presidente})$ domina a $L(\text{Costo del p.\ nuclear})$ y
    $L(\text{Costo del ejercito})$, por lo que en base a la propiedad-ss, esta
    permitido.
\item No, el mayor no tiene acceso al número de unidades nucleares por que su
  nivel de seguridad no incluye la categoría N. En otras palabras, porque 
  $(\text{C, \{E\}})~\neg\mathsf{dom}~(\text{C, \{N\}})$.
\item Sí, el nivel de seguridad del coronel domina al nivel de ambos objetos.
\item Sí.
\item Ambos pueden, ya que $L(\text{Código Nuclear}) = (\text{TS, \{N\}})$
  domina tanto a (C, \{N\}) (nivel del coronel) como a (U, \{N\}) (nivel del
  soldado), por lo que no contradicen la propiedad-*.
\item Bell-La Padula no garantiza integridad, solo confidencialidad. Esto
  significa que un sujeto con clasificación \textit{Unclassified}, puede borrar
  (accidentalmente o no) información secreta. Para prevenir esto, a veces se
  usa una propiedad-* modificada que requiere $L(s) = L(o)$; es decir, todo
  sujeto puede escribir en su mismo nivel, pero no en mayores.
\end{enumerate}
\end{soln}
\end{document}



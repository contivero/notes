\documentclass[spanish]{article}

\usepackage{color}
\definecolor{darkblue}{RGB}{49,130,189}

% Access modes
\newcommand{\acmo}[1]{\underline{\texttt{#1}}}

% Print today's date in ISO format (YYYY-MM-DD).
\def\isodate{\leavevmode\hbox{\the\year-\twodigits\month-\twodigits\day}}
\def\twodigits#1{\ifnum#1<10 0\fi\the#1}


\usepackage[utf8]{inputenc}
\usepackage[es-tabla]{babel}
\usepackage{graphicx}
\usepackage{caption}
\usepackage{float}
\usepackage{epigraph}
\usepackage{multicol}
\usepackage[bottom,flushmargin]{footmisc} 

\usepackage{tikz}
\usetikzlibrary{arrows.meta}
\usetikzlibrary{babel}
\usetikzlibrary{decorations.pathreplacing}

\usepackage{hyperref}
\hypersetup{pdfauthor={Cristian Adrián Ontivero}}
\usetikzlibrary{shapes}
\graphicspath{{imgs/}}

\newlength\tindent
\setlength{\tindent}{\parindent}
\setlength{\parindent}{0pt}
\renewcommand{\indent}{\hspace*{\tindent}}

\renewcommand*{\tableautorefname}{Tabla}
\renewcommand*{\figureautorefname}{Figura}

%These tell TeX which packages to use.
\usepackage{array,epsfig}
\usepackage{amsmath}
\usepackage{amsfonts}
\usepackage{amssymb}
\usepackage{amsxtra}
\usepackage{amsthm}
\usepackage{mathrsfs}

%Here I define some theorem styles and shortcut commands for symbols I use often
\theoremstyle{definition}
\newtheorem{defn}{Definición}
\newtheorem{thm}{Teorema}
\newtheorem{cor}{Corolario}
\newtheorem*{rmk}{Remark}
\newtheorem{lem}{Lema}
\newtheorem*{joke}{Joke}

\newtheorem{ex}{Ejemplo}
\newcommand{\exautorefname}{Ejemplo}

\newtheorem{exercise}{Ejercicio}
\newcommand{\exerciseautorefname}{Ejercicio}

\newtheorem{soln}{Solución}
\newtheorem{prop}{Proposición}

\newcommand{\lra}{\longrightarrow}
\newcommand{\ra}{\rightarrow}
\newcommand{\surj}{\twoheadrightarrow}
\newcommand{\graph}{\mathrm{graph}}
\newcommand{\bb}[1]{\mathbb{#1}}
\newcommand{\Z}{\bb{Z}}
\newcommand{\Q}{\bb{Q}}
\newcommand{\R}{\bb{R}}
\newcommand{\C}{\bb{C}}
\newcommand{\N}{\bb{N}}
\newcommand{\M}{\mathbf{M}}
\newcommand{\m}{\mathbf{m}}
\newcommand{\MM}{\mathscr{M}}
\newcommand{\HH}{\mathscr{H}}
\newcommand{\Om}{\Omega}
\newcommand{\Ho}{\in\HH(\Om)}
\newcommand{\bd}{\partial}
\newcommand{\del}{\partial}
\newcommand{\bardel}{\overline\partial}
\newcommand{\textdf}[1]{\textbf{\textsf{#1}}\index{#1}}
\newcommand{\img}{\mathrm{img}}
\newcommand{\ip}[2]{\left\langle{#1},{#2}\right\rangle}
\newcommand{\inter}[1]{\mathrm{int}{#1}}
\newcommand{\exter}[1]{\mathrm{ext}{#1}}
\newcommand{\cl}[1]{\mathrm{cl}{#1}}
\newcommand{\ds}{\displaystyle}
\newcommand{\vol}{\mathrm{vol}}
\newcommand{\cnt}{\mathrm{ct}}
\newcommand{\osc}{\mathrm{osc}}
\newcommand{\LL}{\mathbf{L}}
\newcommand{\UU}{\mathbf{U}}
\newcommand{\support}{\mathrm{support}}
\newcommand{\AND}{\;\wedge\;}
\newcommand{\OR}{\;\vee\;}
\newcommand{\Oset}{\varnothing}
\newcommand{\st}{\ni}
\newcommand{\wh}{\widehat}

%Pagination stuff.
\setlength{\topmargin}{-.3 in}
\setlength{\oddsidemargin}{0in}
\setlength{\evensidemargin}{0in}
\setlength{\textheight}{9.in}
\setlength{\textwidth}{6.5in}
\pagestyle{empty}

\begin{document}
\begin{center}
  {\LARGE Biba: Modelo de Integridad}\\[.2cm]
  Cristian Adrián Ontivero \\[.05cm]%
  \isodate
\end{center}


\vspace{0.2 cm}

En 1975 Kenneth. J. Biba publicó en un reporte el primer modelo de integridad
\cite{Biba75}.  Este reporte fue posteriormente revisado y abierto al público en
1977 \cite{Biba77}, aproximadamente un año luego de la última publicación de
MITRE del modelo de Bell-La~Padula. El modelo de Biba fue también desarrollado
dentro de MITRE Corporation, como parte del \textit{Secure General Purpose
Computer Project} de la Fuerza Aérea de Estados Unidos. En sus reportes, Biba
describió un modelo en el que se usaban controles similares a los de BLP, pero
para integridad en vez de confidencialidad, y haciendo una examinación de cómo es
posible mantener la validez de información.  Para salvaguardar contra la
modificación indebida de información permitida en Bell-La~Padula, Biba introdujo
los conceptos de niveles de integridad y política de integridad. La idea básica
de su modelo es que la información de baja integridad no debería fluir a objetos
de mayor integridad (pero si vice-versa). El modelo suele resumirse como
\textit{``read up, write down''} (lo opuesto a Bell-La~Padula).

%Una política de este tipo es aplicable en mucho casos, por ejemplo, es
%permisible y hasta positivo citar una revista de divulgación científica en un
%blog, pero lo inverso es, cuando menos, cuestionable.

Mientras BLP se ocupa de proteger contra la diseminación no autorizada de
información, Biba se ocupa de identificar y hacer cumplir la correcta
modificación de información. Este modelo consta de:
\begin{itemize}
\itemsep0em 
  \item Un conjunto $S$ de sujetos.
  \item Un conjunto $O$ de objetos.
  \item Tres modos de acceso: \acmo{o} (\textit{observe}), \acmo{m}
    (\textit{modify}), \acmo{i} (\textit{invoke}). Formalmente, estos son
    relaciones (los dos primeros subconjuntos de $S \times O$, el tercero
    subconjunto de $S \times S$). Dados sujetos $s,s' \in S$ y un objeto $o \in
    O$, decimos:
    
  \begin{itemize}
    \item $s~\acmo{o}~o$ si $s$ tiene la capacidad de observar $o$.
    \item $s~\acmo{m}~o$ si $s$ tiene la capacidad de modificar $o$.
    \item $s~\acmo{i}~s'$ si $s$ tiene la capacidad de invocar $s'$.
  \end{itemize}
   
  \item Un conjunto de niveles de integridad $I$ (\textit{integrity levels}) con un orden parcial $\leq$.
\end{itemize}

Tanto sujetos como objetos son primitivas del modelo, y qué son en la práctica
dependerá de la implementación. Los sujetos son aquellos elementos del sistema
que realizan acceso de información (son ``procesadores de información''). Los
objetos son aquellos elementos que son accedidos (son ``repositorios de
información''). Los modos de acceso son abstractos, con observación y
modificación siendo análogos a lectura y escritura en BLP. La invocación, por el
contrario, no es lo mismo que ejecución. La invocación es un pedido lógico de
servicio, de un sujeto a otro. Invocación representa un acceso de control entre
distintos sujetos, mientras que ejecución es el acceso de un sujeto a un objeto
con el fin de obtener instrucciones. De hecho, para los propósitos del modelo de
Biba, ejecución es equivalente a observación (ya que para ejecutar, un sujeto
necesita observar las instruciones dentro del objeto).

Los niveles de integridad son elementos de $I = C \times \mathcal{P}(K)$, donde
$C$ es el conjunto de clases de integridad (parcialmente ordenadas), y 
$\mathcal{P}(K)$ es el conjunto partes de $K$, siendo $K$ el conjunto de
compartimientos. Esto es análogo a los niveles de seguridad de Bell-La Padula.
En el primer reporte se usan a modo de ejemplo las clases de integridad
\textit{Important} (I), \textit{Very Important} (VI), y \textit{Crucial} (C), en
orden creciente. Aun así, estas no son particularmente estandar, y las clases
usadas en la práctica dependerán del ámbito en que se use el
modelo\footnote{Poco después de mencionar estas tres clases de integridad, Biba
  procede (por simplicidad o descuido) a mezclar su terminología, usando las
clases de integridad mencionadas como si fueran niveles de integridad. Esto no
quita validez a su modelo, ya que usarlas como niveles es isomorfo a usar clases
de integridad con un conjunto vacío de compartimientos.}. 

Dependiendo de qué tan confiable es un individuo, se le asigna un nivel
de integridad. Este mismo criterio se usa para asignarle un nivel de
seguridad. En contraste, la asignación de un nivel de integridad a un objeto se
hace en base a distintos criterios que los usados al asignarle un nivel de
seguridad: los niveles de integridad se asignan para prevenir sabotaje de
información; los de seguridad para prevenir divulgación de información. Esto
lleva a la conclusión que es preferible usar distintos valores para los niveles
de integridad y seguridad (es decir, distintas clases y distintas
categorías/compartimientos), ya que la semántica (los criterios de asignación)
es distinta. El modelo cuenta también con una función $il\colon S \times O \to I$ que define
el nivel de integridad de sujetos y objetos.
\subsection*{Políticas Mandatorias de Integridad}

En su reporte, Biba describió tres políticas mandatorias de integridad. Todas
las políticas comparten las siguientes dos propiedades (dadas como axiomas
por Biba):
\begin{enumerate}
  \item Un sujeto $s$ puede modificar un objeto $o$ sólo si $il(o) \leq il(s)$.

    Esta propiedad a veces se la conoce por el nombre de propiedad-* de
    integridad.

  \item Un sujeto $s_1$ puede invocar a un sujeto $s_2$ sólo si $il(s_2) \leq il(s_1)$.

    A veces conocida como propiedad de invocación (\textit{invocation property}).
\end{enumerate}
La propiedad (1) asegura que la modificación directa maliciosa es imposible. Si
un sujeto pudiera alterar un objeto de mayor confianza, podría implantarle
información de menor integridad (porque el sujeto es menos confiable). Si esto
pasara, de cierta manera el objeto se volvería tan confiable como el sujeto, por
lo que se lo prohibe.
La propiedad (2) previene la modificación indirecta de objetos de mayor integridad por
sujetos, a través de otro sujeto de mayor integridad. Con estas dos se maneja el
problema de la modificación directa de información.

\subsubsection*{Política Low-Water Mark}
A veces conocida bajo el acrónimo LOMAC (\textit{Low-water Mark Mandatory Access
Control}), esta política hace uso de un función\footnote{Llamada \underline{min}
por Biba, pero evitaremos el nombre por ser este levemente engañoso.}
$\text{inf}\colon \mathcal{P}(I) \to I$, que devuelve el ínfimo (máxima cota
inferior) de los niveles de entrada. Además de las dos propiedades mencionadas,
esta política agrega la siguiente:
\begin{itemize}
  \item[3.1] Luego de cualquier observación de un objeto $o$ por un sujeto $s$, el
    nivel de integridad del sujeto $il'(s)$ inmediatamente luego del acceso,
    se define como:
    \[ il'(s) = \text{inf}\{il(s), il(o)\} \]
  \end{itemize}
La propiedad (3.1) previene modificaciones indirectas indebidas.
De los tres modelos, este es el único dinámico, en el sentido que el valor del
nivel de integridad de un sujeto no es estático, sino que monótono no
creciente. La \textit{low-water mark} (marca de bajamar) hace referencia al
menor nivel de integridad de un objeto observado por un sujeto; efectivamente,
el sujeto se ``marca'' o ``ensucia'' con el menor nivel de integridad al que
accede. Esta política tiene el problema de que los sujetos tienden a reducir su
integridad con el tiempo, y no hay forma de recuperar el nivel original salvo
reinicializar el sujeto.

\subsubsection*{Política Ring}
Esta política esta definida por las propiedades (1) y (2) mencionados
anteriormente, y no agrega ningún otro. Es por esto que solo aborda el problema
de la modificación directa, pero no asegura nada contra la modificación
indirecta. El modelo bajo la política ring\footnote{Su nombre origina de la
similitud con un mecanismo de protección homónimo en Multics.} es estático, ya
que los niveles de integridad de sujetos y objetos se mantienen fijos. Es más
flexible que low-water mark, a expensas de tener menos garantías de integridad.

\subsubsection*{Política Strict Integrity}

Cuando se habla simplemente del ``modelo de Biba'' sin especificar, se hace
referencia a esta política, siendo que es la principal. La misma consiste en las
propiedades (1) y (2), y agrega la siguiente:
\begin{itemize}
  \item[3.2] Un sujeto $s$ puede observar un objeto $o$ sólo si $il(s) \leq il(o)$.

    Esta propiedad a veces se la conoce como propiedad simple de integridad o
    \textit{no write up} (NWU).
  \end{itemize}
Esta política provee capacidades similares a la política low-water mark, pero es
más restrictiva: cuando una lectura en low-water mark alteraría el nivel de
integridad de un sujeto, en strict integrity se la prohibe. La observación clave
de esta política es que integridad y confidencialidad son en cierto sentido
conceptos duales: integridad es una restricción sobre quién puede alterar o
escribir un objeto, y confidencialidad sobre quién puede observarlo o leerlo.
Mientras que la información fluye hacia arriba en Bell-La~Padula, en Biba fluye
hacia abajo. Un esquema de esto se observa en la \autoref{fig:biba-blp}, donde
los $\omega_i$ representan niveles de integridad y los $\lambda_i$ niveles de
seguridad. En ambos casos, los niveles son monótonamente no decrecientes.

\begin{figure}[htp]
  \centering
\begin{minipage}{.76\textwidth}
  \centering
  \begin{tikzpicture}
    \node at (9,4.2) (TS) {\textbf{Bell-La Padula}};
    \node at (9,3.2) (ln) {$\lambda_m$};
    \node at (9,2.2) {$\cdots$};
    \node at (8,2.2) (l1) {$\lambda_{m-1}$};
    \node at (10,2.2) (l2) {$\lambda_{m-2}$};
    \node at (9,0.2) (l0) {$\lambda_0$};
    \draw[<-] (ln) -- (l1);
    \draw[<-] (ln) -- (l2);
    \draw[<-, dash pattern=on 16pt off 2pt on 3pt off 2pt on 3pt off 2pt on 3pt
    off 2pt on 16pt] (l1) to (l0);
    \draw[<-, dash pattern=on 16pt off 2pt on 3pt off 2pt on 3pt off 2pt on 3pt
    off 2pt on 16pt] (l2) to (l0);

    \draw[->, line width=1pt] (7.3,0) to % node[above, sloped] {escribe}
    (7.3,3.2);

    \draw[->, line width=1pt] (0.3,3.2) to % node[above, sloped] {escribe}
    (0.3,0);

    \node at (2,4.2) (Biba) {\textbf{Biba}};
    \node at (2,3.2) (wn) {$\omega_n$};
    \node at (2,2.2) {$\cdots$};
    \node at (1,2.2) (w1) {$\omega_{n-1}$};
    \node at (3,2.2) (w2) {$\omega_{n-2}$};
    \node at (2,0.2) (w0) {$\omega_0$};
    \draw[<-] (w1) -- (wn);
    \draw[<-] (w2) -- (wn);
    \draw[<-, dash pattern=on 16pt off 2pt on 3pt off 2pt on 3pt off 2pt on 3pt
    off 2pt on 16pt] (w0) to (w1);
    \draw[<-, dash pattern=on 16pt off 2pt on 3pt off 2pt on 3pt off 2pt on 3pt
    off 2pt on 16pt] (w0) to (w2);

  \end{tikzpicture}
  \caption{\label{fig:biba-blp} Flujo de información, representado por flechas,
  en Biba y Bell-La~Padula.  Los $\omega_i$ representan niveles de integridad, y
los $\lambda_i$ niveles de seguridad.}
\end{minipage}
\end{figure}

\begin{defn}
Un camino de transferencia de información es una secuencia de objetos
$\langle o_0,\dots,o_{n+1}\rangle$ y una secuencia de sujetos correspondiente
$\langle s_0,\dots,s_{n}\rangle$, tal que
\[\forall i \in \{0,\dots,n\},\hspace{.4cm} s_i~\acmo{o}~o_i\hspace{.2cm}\wedge\hspace{.2cm}s_i~\acmo{m}~o_{i+1}\]
\end{defn}

El siguiente teorema nos asegura que la política de integridad estricta
mantendrá la integridad de los objetos, como definida por la asignación de
niveles de integridad.

\begin{thm}
Si existe un camino de transferencia de información de un objeto $o_0$ a un
objeto $o_{n+1}$, entonces hacer cumplir la política estricta de integridad requiere
$il(o_{n+1}) \leq il(o_0)$.
\end{thm}

\begin{proof}
Supongamos que existe un camino de transferencia de información, entonces por
definición existe la secuencia de sujetos y objetos especificada. Por la
propiedad-* de integridad, y la propiedad simple de integridad, tenemos:
\[\forall i \in \{0,\dots,n\}\hspace{.4cm} il(o_{i+1}) \leq il(s_i) \leq il(o_i) \]
Esto puede observarse en la \autoref{fig:proof} (usando $\geq$ en vez de $\leq$ con
su significado obvio, para explicitar la secuencia de desigualdades).
\begin{figure}[htp]
  \centering
\begin{minipage}{.9\textwidth}
  \centering
  \begin{tikzpicture}
    \node at (0,0) (o0) {$o_0$};
    \node at (3,0) (s0) {$s_0$};
    \node at (6,0) (o1) {$o_1$};
    \node at (9,0) (sn) {$s_n$};
    \node at (12,0) (on1) {$o_{n+1}$};
    \draw[->] (o0) to node[above, sloped] {\acmo{o}} (s0);
    \draw[->] (s0) to node[above, sloped] {\acmo{m}} (o1);
    \draw[->] (s0) to (o1);
    \draw[->] (sn) to node[above, sloped] {\acmo{m}} (on1);
    \draw[->, dash pattern=on 23pt off 2pt on 3pt off 2pt on 3pt off 2pt on 3pt
    off 2pt on 3pt off 2pt on 23pt] (o1) to (sn);

\draw [decorate,decoration={brace,mirror, amplitude=4pt},xshift=-4pt]
(sn.south) -- (on1.south) node [black,midway,yshift=-14pt] {$il(s_n) \geq il(o_{n+1})$};
\draw [decorate,decoration={brace,mirror, amplitude=4pt},xshift=-4pt]
(o1.south) -- (sn.south) node [black,midway,yshift=-14pt] {$\cdots \geq \cdots$};
\draw [decorate,decoration={brace,mirror, amplitude=4pt},xshift=-4pt]
(s0.south) -- (o1.south) node [black,midway,yshift=-14pt] {$il(s_0) \geq il(o_1)$};
\draw [decorate,decoration={brace,mirror, amplitude=4pt},xshift=-4pt]
(o0.south) -- (s0.south) node [black,midway,yshift=-14pt] {$il(o_0) \geq il(s_0)$};
  \end{tikzpicture}
  \caption{\label{fig:proof} Camino de transferencia de información, con las
desigualdades establecidas por las propiedades (1) y (3.2). La observación de
$o$ por $s$ se representa mediante $o \xrightarrow{\acmo{o}} s$, y mediante $s
\xrightarrow{\acmo{m}} o$ la modificación de $o$ por $s$.}
\end{minipage}
\end{figure}

Como $\leq$ es un orden parcial, es por definición transitivo, entonces:
\[\forall i \in \{0,\dots,n\}\hspace{.4cm} il(o_{i+1}) \leq il(o_i) \]
Por lo tanto, $il(o_{n+1}) \leq il(o_0)$.
\end{proof}

En \cite{Sandhu93} vemos que la similitud entre el modelo de integridad estricta
de Biba, y el modelo Bell-La~Padula va más allá de lo mencionado. No es
necesario pensar a los niveles de mayor integridad arriba, y los de menor abajo;
``arriba'' y ``abajo'' son términos relativos, no absolutos. Podemos decir que
los niveles de mayor integridad estan abajo y los de menor arriba, o
equivalentemente, que los niveles de mayor seguridad se encuentran abajo, y los
de menor arriba en Bell-La Padula. De cualquiera de las dos maneras,
coincidiría el sentido del flujo de información en ambos modelos. Con esto vemos
que no hay una diferencia fundamental entre la política de integridad estricta
de Biba, y el modelo de Bell-La Padula: ambos se ocupan del flujo de información
en un retículo de niveles, donde la información solo tiene permitido fluir en un
sentido. Como el sentido del flujo es relativo, un sistema que soporta uno de
estos modelos puede soportar el otro (necesitando quizás reordenar etiquetas
para invertir la relación de dominancia).

%




\bibliography{refs}
\bibliographystyle{unsrt}

\end{document}



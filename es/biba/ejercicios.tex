
\section*{Ejercicios}
\begin{enumerate}
  \item Dado un sistema que usa el modelo Bell-La~Padula para confidencialidad,
    y el modelo Biba para integridad:
  \begin{enumerate}
    \item Si se usaran las mismas clases de seguridad como clases de integridad, ¿qué objetos podría acceder un sujeto dado?
    \item ¿Por qué no se usa este esquema en la práctica?
  \end{enumerate}
\item Dado un modelo de Biba (\textit{strict integrity}) con:
  \begin{itemize}
    \item $C = \{\text{alto, medio, bajo, desconocido}\}$, en orden decreciente.
    \item $K = \{\text{A, B, C}\}$ 
  \end{itemize}
¿Puede un usario con nivel de integridad (medio, \{A, B\}) leer o escribir (o
ambos) documentos con los siguientes niveles de integridad?
\begin{multicols}{3}
\begin{enumerate}

  \item (alto, \{A\})
  \item (bajo, \{A\})
  \item (medio, \{A, B\})
  \item (desconocido, \{C\})
  \item (alto, \{A, C, B\}) 
\end{enumerate}
\end{multicols}

\item Dado un modelo de Biba, con niveles de integridad Bajo, Medio, y Alto, en
  orden creciente. Los sujetos y objetos con sus respectivos niveles como se
  observa:
\begin{table}[h!]
  \centering
\begin{tabular}{l|l} 
 Sujeto & Nivel de Integridad \\ 
 \hline
CEO                 & Alto\\ 
Lider del Proyecto  & Medio\\ 
Redactor            & Bajo\\
\end{tabular}
\end{table}

\begin{table}[h!]
  \centering
\begin{tabular}{l|l} 
Objetos & Nivel de Integridad \\ 
 \hline
Acuerdo de cooperación   & Alto\\ 
Descripción del Proyecto & Medio\\ 
Mensaje de prensa        & Bajo\\
\end{tabular}
\end{table}

Los objetos corresponden a proyectos en común entre dos compañías. El acuerdo de
cooperación es un documento legal que requiere integridad alta. La descripción
del proyecto debe ser confiable, pero borradores de mensajes de prensa no poseen
ningún requerimiento de integridad en particular.

\begin{enumerate}
  \item Mostrar la matriz de control de accesos del sistema. Los permisos son
    \acmo{o} y \acmo{m}. Escribir cualquier
    supuesto necesario.

\item ¿Existe alguna limitación sobre qué objetos puede un sujeto leer que
consideres inadecuada? Explicar estas limitaciones y dar sugerencias de cómo
resolver estos problemas, manteniendo tanto como sea posible la integridad
proveida por el modelo de Biba.

\item ¿Qué otro modelo de integridad sería más adecuado para el caso? Nombrarlo y
explicar sus propiedades principales.
\end{enumerate}
\item ¿Verdadero o falso?:

\begin{enumerate}
  \item El nivel de integridad de sujetos y objetos se mantiene estático en el
    modelo de integridad estricta de Biba.
\end{enumerate}
\end{enumerate}

\item Al igual que para la política de integridad estricta, para low-water mark
  existe un teorema que nos asegura prevención contra modificaciones indirectas
  indebidas.
  
  \begin{thm}
    Si existe un camino de transferencia de información de un objeto $o_0 \in O$
    a un objeto $o_{n+1} \in O$, entonces hacer cumplir la política low-water
    mark requiere $il(o_{n+1}) \leq il(o_0)$.
  \end{thm}
  Demostrar.

\section{Respuestas}
\begin{enumerate}
  \item 
    \begin{enumerate}
      \item Esto equivale a la política trivial de no permitir flujo de
        información a través de niveles. En otras palabras, si los niveles de
        seguridad equivalen a los de integridad, un sujeto solo puede acceder
        objetos en su mismo nivel.

      \item Este esquema es demasiado restrictivo para ser práctico, ya que
        normalmente los procesos (sujetos) necesitan leer objetos en niveles
        iguales o menores.
    \end{enumerate}
\end{enumerate}

\documentclass[]{book}

\usepackage[utf8]{inputenc}
\usepackage[yyyymmdd,hhmmss]{datetime}
%These tell TeX which packages to use.
\usepackage{array,epsfig}
\usepackage{amsmath}
\usepackage{amsfonts}
\usepackage{amssymb}
\usepackage{amsxtra}
\usepackage{amsthm}
\usepackage{mathrsfs}
\usepackage{color}

%Here I define some theorem styles and shortcut commands for symbols I use often
\theoremstyle{definition}
\newtheorem{defn}{Definition}
\newtheorem{thm}{Theorem}
\newtheorem{cor}{Corollary}
\newtheorem*{rmk}{Remark}
\newtheorem{lem}{Lemma}
\newtheorem*{joke}{Joke}
\newtheorem{ex}{Example}
\newtheorem*{soln}{Solution}
\newtheorem{prop}{Proposition}

\newcommand{\lra}{\longrightarrow}
\newcommand{\ra}{\rightarrow}
\newcommand{\surj}{\twoheadrightarrow}
\newcommand{\graph}{\mathrm{graph}}
\newcommand{\bb}[1]{\mathbb{#1}}
\newcommand{\Z}{\bb{Z}}
\newcommand{\Q}{\bb{Q}}
\newcommand{\R}{\bb{R}}
\newcommand{\C}{\bb{C}}
\newcommand{\N}{\bb{N}}
\newcommand{\M}{\mathbf{M}}
\newcommand{\m}{\mathbf{m}}
\newcommand{\MM}{\mathscr{M}}
\newcommand{\HH}{\mathscr{H}}
\newcommand{\Om}{\Omega}
\newcommand{\Ho}{\in\HH(\Om)}
\newcommand{\bd}{\partial}
\newcommand{\del}{\partial}
\newcommand{\bardel}{\overline\partial}
\newcommand{\textdf}[1]{\textbf{\textsf{#1}}\index{#1}}
\newcommand{\img}{\mathrm{img}}
\newcommand{\ip}[2]{\left\langle{#1},{#2}\right\rangle}
\newcommand{\inter}[1]{\mathrm{int}{#1}}
\newcommand{\exter}[1]{\mathrm{ext}{#1}}
\newcommand{\cl}[1]{\mathrm{cl}{#1}}
\newcommand{\ds}{\displaystyle}
\newcommand{\vol}{\mathrm{vol}}
\newcommand{\cnt}{\mathrm{ct}}
\newcommand{\osc}{\mathrm{osc}}
\newcommand{\LL}{\mathbf{L}}
\newcommand{\UU}{\mathbf{U}}
\newcommand{\support}{\mathrm{support}}
\newcommand{\AND}{\;\wedge\;}
\newcommand{\OR}{\;\vee\;}
\newcommand{\Oset}{\varnothing}
\newcommand{\st}{\ni}
\newcommand{\wh}{\widehat}

%Pagination stuff.
\setlength{\topmargin}{-.3 in}
\setlength{\oddsidemargin}{0in}
\setlength{\evensidemargin}{0in}
\setlength{\textheight}{9.in}
\setlength{\textwidth}{6.5in}
\pagestyle{empty}



\begin{document}


\begin{center}
  {\Large Criptografía y Seguridad \the\year~(72.44)\\[.2cm]
Guía 1 - Soluciones}\\
\end{center}

\vspace{0.2 cm}


\subsection*{Ejercios}
\begin{enumerate}
\item Cifrado de rotación:
\begin{itemize}
\item $\mathsf{Gen}$: Elige aleatoriamente $k \in \{0,\dots,25\}$.
\item $\mathsf{Enc}$: Dado un mensaje $m = m_1m_2\dots m_n$, donde cada $m_i$ se
  corresponde con una letra en $\{a,\dots,z\}$, entonces el cifrado es $c =
  c_1c_2\dots c_n$, donde $\forall i, c_i = m_i + k \mod 26$. Cada $c_i$
  también se corresponde con una letra en $\{a,\dots,z\}$.
\item $\mathsf{Dec}$: Dado el cifrado $c = c_1c_2\dots c_n$, entonces el texto plano
  obtenido es $m = m_1m_2\dots m_n$, donde $m_i = c_i - k \mod 26$.
\end{itemize}

Cifrado por sustitución monoalfabética:
\begin{itemize}
\item $\mathsf{Gen}$: Elige aleatoriamente una función de todas las que hay en
  $K$, es decir: $f_k \leftarrow K = \{f_1, f_2, \dots , f_q\}$, donde $q$ es la cantidad de símbolos del alfabeto $\Sigma$, y $\forall i, f_i : \Sigma \longrightarrow \Sigma$ es biyectiva.
\item $\mathsf{Enc}$: Dado un mensaje $m = m_1m_2\dots m_n$, su cifrado es $c = c_1c_2\dots c_n$, donde $\forall i, c_i = f_k(m_i)$.
\item $\mathsf{Dec}$: Dado un cifrado $c = c_1c_2\dots c_n$, el texto plano
  obtenido es $m = m_1m_2\dots m_n$, donde cada $m_i = f^{-1}_k(c_i)$, y $f^{-1}_k$ denota la función inversa de $f_k$ (que existe por ser esta biyectiva).
\end{itemize}

Cifrado de Vigenère:
\begin{itemize}
\item $\mathsf{Gen}$: Elige aleatoriamente una palabra del alfabeto: $k
  \leftarrow \Sigma^+$. Es decir, $k = k_0k_1\dots k_{n-1}$, donde $\forall
  i, k_i \in \Sigma$. Se considera que cada letra se corresponde con un número
  entero de $\{0,\dots,25\}$, lo cual facilita la definición de las operaciones.
  \item $\mathsf{Enc}$: Dado un mensaje $m = m_0m_1\dots m_{n-1}$, entonces el
  cifrado es $c = c_0c_1\dots c_{n-1}$, donde $\forall i, c_i = m_i +
  k_j \mod 26$, con $i \equiv j \mod t$, $1 \leq j \leq t$.
  \item $\mathsf{Dec}$: Dado el cifrado $c = c_1c_2\dots c_n$, entonces el texto
  plano obtenido es $m = m_1m_2\dots m_n$, donde $\forall i, m_i = c_i - k_j \mod 26$, con $i \equiv j \mod t$, $1 \leq j \leq t$.
\end{itemize}
\end{enumerate}
\end{document}



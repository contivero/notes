
\documentclass[]{book}

\usepackage{tikz}
\usepackage{pgfplots}
\usepackage[utf8]{inputenc}
\usepackage[yyyymmdd,hhmmss]{datetime}
\usepackage{wrapfig}
\usepackage[spanish,es-tabla]{babel}
\usepackage{hyperref}
%These tell TeX which packages to use.
\usepackage{array,epsfig}
\usepackage{amsmath}
\usepackage{amsfonts}
\usepackage{amssymb}
\usepackage{amsxtra}
\usepackage{amsthm}
\usepackage{mathrsfs}
\usepackage{color}

%Here I define some theorem styles and shortcut commands for symbols I use often
\theoremstyle{definition}
\newtheorem{defn}{Definition}
\newtheorem{thm}{Theorem}
\newtheorem{cor}{Corollary}
\newtheorem*{rmk}{Remark}
\newtheorem{lem}{Lemma}
\newtheorem*{joke}{Joke}
\newtheorem{ex}{Example}
\newtheorem*{soln}{Solution}
\newtheorem{prop}{Proposition}

\newcommand{\lra}{\longrightarrow}
\newcommand{\ra}{\rightarrow}
\newcommand{\surj}{\twoheadrightarrow}
\newcommand{\graph}{\mathrm{graph}}
\newcommand{\bb}[1]{\mathbb{#1}}
\newcommand{\Z}{\bb{Z}}
\newcommand{\Q}{\bb{Q}}
\newcommand{\R}{\bb{R}}
\newcommand{\C}{\bb{C}}
\newcommand{\N}{\bb{N}}
\newcommand{\M}{\mathbf{M}}
\newcommand{\m}{\mathbf{m}}
\newcommand{\MM}{\mathscr{M}}
\newcommand{\HH}{\mathscr{H}}
\newcommand{\Om}{\Omega}
\newcommand{\Ho}{\in\HH(\Om)}
\newcommand{\bd}{\partial}
\newcommand{\del}{\partial}
\newcommand{\bardel}{\overline\partial}
\newcommand{\textdf}[1]{\textbf{\textsf{#1}}\index{#1}}
\newcommand{\img}{\mathrm{img}}
\newcommand{\ip}[2]{\left\langle{#1},{#2}\right\rangle}
\newcommand{\inter}[1]{\mathrm{int}{#1}}
\newcommand{\exter}[1]{\mathrm{ext}{#1}}
\newcommand{\cl}[1]{\mathrm{cl}{#1}}
\newcommand{\ds}{\displaystyle}
\newcommand{\vol}{\mathrm{vol}}
\newcommand{\cnt}{\mathrm{ct}}
\newcommand{\osc}{\mathrm{osc}}
\newcommand{\LL}{\mathbf{L}}
\newcommand{\UU}{\mathbf{U}}
\newcommand{\support}{\mathrm{support}}
\newcommand{\AND}{\;\wedge\;}
\newcommand{\OR}{\;\vee\;}
\newcommand{\Oset}{\varnothing}
\newcommand{\st}{\ni}
\newcommand{\wh}{\widehat}

%Pagination stuff.
\setlength{\topmargin}{-.3 in}
\setlength{\oddsidemargin}{0in}
\setlength{\evensidemargin}{0in}
\setlength{\textheight}{9.in}
\setlength{\textwidth}{6.5in}
\pagestyle{empty}

\def\mybar#1{%%
  #1 & {\color{black}\rule{#1ex}{8pt}}}

\begin{document}

\begin{center}
  {\Large Criptografía y Seguridad \the\year~(72.44)\\[.2cm]
Guía 1 - Criptografía Clásica}\\
%\textbf{NAME}\\ %You should put your name here
%Due: DATE %You should write the date here.
\end{center}

\vspace{0.2 cm}


\subsubsection*{Ejercicios}
\begin{enumerate}
\item\label{norms}
  Dar una definición formal de los algoritmos $\mathsf{Gen}$, $\mathsf{Enc}$ y
  $\mathsf{Dec}$ para los siguientes esquemas: 
\begin{enumerate}
  \item Cifrado de rotación.
  \item Cifrado de sustitución monoalfabética.
  \item Cifrado de Vigenère.
\end{enumerate}

\item ¿Por qué la composición de dos sistemas de sustitución simple no provee más seguridad que el uso de uno solo? Ejemplificar. 

\item Descifrar el siguiente criptograma, sabiendo que fue encriptado usando un
  cifrado de rotación, que se corresponde a un texto en español (27 letras), y
  los espacios fueron suprimidos. ¿Qué estrategia utilizaste? 
\begin{center}
\verb+VKXYKBKXGKSGWAKQQGYIUYGYWAKXKGQRKSZKJKYKKYIUSYKMAÑX+
\end{center}

\item 
  \begin{enumerate}
    \item Cifrar según Vigenère el mensaje $m$ = \verb+un vino de mesa+ con la clave
      $k$ = \verb+baco+ sin usar la tabla, sólo con operaciones modulares.
    \item En un sistema de cifrado de Vigenère la clave a usar puede ser o bien
      \verb+cero+, o bien \verb+compadre+, ¿cuál de las dos conviene usar y por qué?
    \item Mostrar, con un ejemplo, que la composición de dos cifrados Vigenère resulta en otro cifrado Vigenère.
  \end{enumerate}

\item Teniendo en cuenta que la frecuencia de aparición de letras en español
  dada en la \autoref{tab:freq}, decir, para cada criptograma, si se ha obtenido mediante técnicas de sustitución monoalfabética, sustitución polialfabética, o de transposición. 
  \begin{itemize}
    \item Criptograma 1:

        \verb+KOZFVPCYVCWVZHMZLCIOHIFIZGJCZTVVXIGJLZHYZLGVMNVLYZ+
    \item Criptograma 2:
      
      \verb+HHMBIWSIPSNNTAWVITQWMEAQVNSPGQJNWELXMJDIBYUGNNRMEUDEMZIBTMYMB+

      \verb+MWURBTIZXNCWZIUPZUQNRMEGJLWRVROPMREUMXXXAXDIPUVFEASMBSASCETAE+

      \verb+WOYYAKUSWEABSASCRECIOMEWTQOMYALMTXRAGEWSQQHJDXMVJEAFIRNDUIANW+
    \item Criptograma 3:

        \verb+DERTNYLANAOTAABADEAXCEEAIDEJLXHRSUAUJUMXELAATECRTRNAZBIRESOX+
      \end{itemize}
\item Se recibe el siguiente criptograma:
\begin{verbatim}
JGAZN WINHY LZDYV BBJLC QHTNK UDQXM OXJNO ZMUSP NONYJ MTEJH QHQFO 
OPUPB CYAÑJ ONCNN QHNMO NDHKU TJMQC MOPNF AOXNT NLOAZ MJDQY MOZCJ 
RNBAO QTUIE NFAIX TLXJG AZMJA XJVAZ MUDNM YLNLJ MUMUY HVUMH TÑIGD 
XDQUC LSJPI BCUSF NUGXX GEEXK AEJME SJÑEN ASLHL BAEYJ ROJXA CQTCN 
MYPUC UNMJW OYNHZ NKUOG AJDUJ XENRY TENJS CNMON TYJNM JYFXF IGJMI 
BUUSN TFAPN FAFKU ROJNY CTUYN BYSGJ VACAU CGQWA ZMJJH JHSNT PAPXM 
GNECO GJUTE NCNGJ GEGAJ SPNUL GDMAÑ JDOFD NPUNN PNTGE NMJSN TTOFD 
KIOXS SQNNF BATOC XMMNV ÑEZNM EZBOS NTUSQ BUDBT JRBBU YPQZI OQFTB
ANIBV MEDDY RUMUP NAULB OMAED HVHNF OCJOS NMJ
\end{verbatim}

Si se conoce que ha sido cifrado mediante el algoritmo de Vigenère, se pide: 
  \begin{enumerate}
    \item Comprobar la longitud de la clave. 
    \item Encontrar la clave del sistema y desencriptar sólo los diez primeros caracteres. 
  \end{enumerate}
Para ello tener en cuenta lo siguiente: 

\begin{itemize}
  \item Listar todas las secuencias repetidas de por lo menos 3 caracteres, junto con la distancia a la que se encuentran. 
  \item Ayuda: Aparecen cuatro cadenas de cuatro caracteres que se repiten en el criptograma: \verb+JGAZ+, \verb+NMON+, \verb+PNFA+ y \verb+AZMJ+
  \item Estimar cuál puede ser la longitud y obtener la frecuencia de aparición de cada letra como primera de cada bloque. 
  \end{itemize}

\item Se cuenta con cifrado, producto de una transposición por columnas
  (cada $x$ columnas se reacomodó el texto original) y un cifrado de rotación. 
  \begin{enumerate}
    \item ¿Qué estrategia usarías para recuperar el mensaje original? 
    \item Si el texto cifrado tiene $n$ caracteres, ¿cuántas pruebas requeriría un ataque de fuerza bruta? 
  \end{enumerate}

\item Mostrar que los siguientes cifrados son fáciles de quebrar mediante un
  ataque de texto plano elegido (\textit{chosen-plaintext attack}).
  \begin{enumerate}
    \item Cifrado de sustitución monoalfabética.
    \item Cifrado de Vigenère.
  \end{enumerate}
\end{enumerate}
  \begin{table}[!h]
\centering
\caption{Frecuencia de letras en el español.%\protect\footnotemark
}\label{tab:freq}
\footnotesize
\begin{tabular}{lrl}
Letra & \% & \\ \hline
a & \mybar{12,53}\\
b & \mybar{1,42}\\
c & \mybar{4,68}\\
d & \mybar{5,86}\\
e & \mybar{13,68}\\
f & \mybar{0,69}\\
g & \mybar{1,01}\\
h & \mybar{0,70}\\
i & \mybar{6,25}\\
j & \mybar{0,44}\\
k & \mybar{0,02}\\
l & \mybar{4,97}\\
m & \mybar{3,15}\\
n & \mybar{6,71}\\
ñ & \mybar{0,31}\\
o & \mybar{8,68}\\
p & \mybar{2,51}\\
q & \mybar{0,88}\\
r & \mybar{6,87}\\
s & \mybar{7,98}\\
t & \mybar{4,63}\\
u & \mybar{3,93}\\
v & \mybar{0,90}\\
w & \mybar{0,01}\\
x & \mybar{0,22}\\
y & \mybar{0,90}\\
z & \mybar{0,52}\\
\end{tabular}
\end{table}
\end{document}


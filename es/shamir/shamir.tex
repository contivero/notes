\documentclass[spanish]{article}

\usepackage{color}
\definecolor{darkblue}{RGB}{49,130,189}

% Access modes
\newcommand{\acmo}[1]{\underline{\texttt{#1}}}

% Print today's date in ISO format (YYYY-MM-DD).
\def\isodate{\leavevmode\hbox{\the\year-\twodigits\month-\twodigits\day}}
\def\twodigits#1{\ifnum#1<10 0\fi\the#1}


\usepackage[utf8]{inputenc}
\usepackage[es-tabla]{babel}
\unaccentedoperators% Evitar que se acentue mod en español
\usepackage{graphicx}
\usepackage{caption}
\usepackage{float}
\usepackage{epigraph}
\usepackage{multicol}
\usepackage[bottom,flushmargin]{footmisc} 

\usepackage{tikz}
\usetikzlibrary{arrows.meta}
\usetikzlibrary{babel}
\usetikzlibrary{decorations.pathreplacing}

\usepackage{hyperref}
\hypersetup{pdfauthor={Cristian Adrián Ontivero}}
\usetikzlibrary{shapes}
\graphicspath{{imgs/}}

\newlength\tindent
\setlength{\tindent}{\parindent}
\setlength{\parindent}{0pt}
\renewcommand{\indent}{\hspace*{\tindent}}

\renewcommand*{\tableautorefname}{Tabla}
\renewcommand*{\figureautorefname}{Figura}

%These tell TeX which packages to use.
\usepackage{array,epsfig}
\usepackage[fleqn]{amsmath}
\usepackage{amsfonts}
\usepackage{amssymb}
\usepackage{amsxtra}
\usepackage{amsthm}
\usepackage{mathrsfs}

%Here I define some theorem styles and shortcut commands for symbols I use often
\theoremstyle{definition}
\newtheorem{defn}{Definición}
\newtheorem{thm}{Teorema}
\newtheorem{cor}{Corolario}
\newtheorem*{rmk}{Remark}
\newtheorem{lem}{Lema}
\newtheorem*{joke}{Joke}

\newtheorem{ex}{Ejemplo}
\newcommand{\exautorefname}{Ejemplo}

\newtheorem{exercise}{Ejercicio}
\newcommand{\exerciseautorefname}{Ejercicio}

\newtheorem{soln}{Solución}
\newtheorem{prop}{Proposición}

\newcommand{\lra}{\longrightarrow}
\newcommand{\ra}{\rightarrow}
\newcommand{\surj}{\twoheadrightarrow}
\newcommand{\graph}{\mathrm{graph}}
\newcommand{\bb}[1]{\mathbb{#1}}
\newcommand{\Z}{\bb{Z}}
\newcommand{\Q}{\bb{Q}}
\newcommand{\R}{\bb{R}}
\newcommand{\C}{\bb{C}}
\newcommand{\N}{\bb{N}}
\newcommand{\M}{\mathbf{M}}
\newcommand{\m}{\mathbf{m}}
\newcommand{\MM}{\mathscr{M}}
\newcommand{\HH}{\mathscr{H}}
\newcommand{\Om}{\Omega}
\newcommand{\Ho}{\in\HH(\Om)}
\newcommand{\bd}{\partial}
\newcommand{\del}{\partial}
\newcommand{\bardel}{\overline\partial}
\newcommand{\textdf}[1]{\textbf{\textsf{#1}}\index{#1}}
\newcommand{\img}{\mathrm{img}}
\newcommand{\ip}[2]{\left\langle{#1},{#2}\right\rangle}
\newcommand{\inter}[1]{\mathrm{int}{#1}}
\newcommand{\exter}[1]{\mathrm{ext}{#1}}
\newcommand{\cl}[1]{\mathrm{cl}{#1}}
\newcommand{\ds}{\displaystyle}
\newcommand{\vol}{\mathrm{vol}}
\newcommand{\cnt}{\mathrm{ct}}
\newcommand{\osc}{\mathrm{osc}}
\newcommand{\LL}{\mathbf{L}}
\newcommand{\UU}{\mathbf{U}}
\newcommand{\support}{\mathrm{support}}
\newcommand{\AND}{\;\wedge\;}
\newcommand{\OR}{\;\vee\;}
\newcommand{\Oset}{\varnothing}
\newcommand{\st}{\ni}
\newcommand{\wh}{\widehat}

%Pagination stuff.
\setlength{\topmargin}{-.3 in}
\setlength{\oddsidemargin}{0in}
\setlength{\evensidemargin}{0in}
\setlength{\textheight}{9.in}
\setlength{\textwidth}{6.5in}
\pagestyle{empty}

\begin{document}
\begin{center}
  {\LARGE Esquema de Umbral $(k,n)$ }\\[.2cm]
  Cristian Adrián Ontivero \\[.05cm]%
\end{center}


\vspace{0.2 cm}

\subsection*{Ejemplo}

Dado un esquema de umbral $(3,5)$, con $p = 17$. Se sabe que $a(1) = 8, a(3) =
10, a(5) = 11$. Hallar el secreto.

Se tiene el polinomio $a(x) = a_2x^2 + a_1x + a_0$. Junto con los puntos dados,
se forma el siguiente sistema de ecuaciones en $\Z_{17}$:
\[
\arraycolsep=2pt
  \begin{array}{rrrrrrr}
    a_2 &+& a_1 &+&  a_0 &=& 8\\
    9a_2 &+& 3a_1 &+& a_0 &=& 10\\
    8a_2 &+& 5a_1 &+& a_0 &=& 11
  \end{array}
\]

Resolviendo este sistema de ecuaciones en $\Z_{17}$ mediante eliminación de
Gauss-Jordan, obtenemos el polinomio $a(x) = 2x^2 + 10x + 13$. El secreto (el
término independiente) es por lo tanto $13$.

Otro método es por ejemplo, mediante el polinomio interpolador de Lagrange.
Dando un conjunto de $k+1$ puntos $(x_0, y_0),\dots,(x_k,y_k)$, donde todos los
$x_j$ son distintos, el polinomio interpolador en la forma de Lagrange es una
combinación lineal:
\[ 
  L(x) = \sum_{i=1}^k y_i \ell_i(x)
\] 
\[
  \ell_i(x) = \prod_{\substack{1 \leq m \leq k\\m \neq i}} \frac{x - x_m}{x_i - x_m}
\]
Ya que $a_0 = a(0)$, reemplazando $x = 0$ en el polinomio interpolador obtenemos:
\[ L(0) = \sum_{i=1}^k y_i \prod_{\substack{1 \leq m \leq k\\m \neq i}} \frac{x_m}{x_m - x_i} \] 
Volviendo al ejemplo anterior:
\[
  \arraycolsep=2pt
\begin{array}{lccccccl}
  \ell_1(0) &=& \dfrac{3}{3-1}\cdot\dfrac{5}{5-1} &=& \dfrac{15}{8} &=& 4 &\bmod{~17}\\[2ex]
  \ell_2(0) &=& \dfrac{1}{1-3}\cdot\dfrac{5}{5-3} &=& \dfrac{5}{-4} &=& 3 &\bmod{~17}\\[2ex]
  \ell_3(0) &=& \dfrac{1}{1-5}\cdot\dfrac{3}{3-5} &=& \dfrac{3}{8} &=& 11 &\bmod{~17}
\end{array}
\]
  \begin{align*}
    a_0 = a(0) &=\sum_{i=0}^k y_i \ell_i(0) \bmod{17}\\
    &= 8\cdot4 + 10\cdot3 + 11\cdot11 = 183 = 13 \bmod{17}\\
  \end{align*}
  \subsection*{Ejercicio}
  \textit{Extensibilidad:} Es fácil generar nuevas sombras, extendiendo un esquema de umbral
  $(k,n)$, o otro $(k, n+1)$. Explicar cómo hacerlo, y bajo qué condiciones es
  posible.

  Solución: evaluar el polinomio en un punto más, siempre y cuando $n+1 < p$,
  siendo $p$ el primo del cuerpo finito utilizado.

%\bibliography{refs}
%\bibliographystyle{unsrt}

\end{document}



\documentclass[spanish]{article}

\usepackage{color}
\definecolor{darkblue}{RGB}{49,130,189}

% Access modes
\newcommand{\acmo}[1]{\underline{\texttt{#1}}}

% Print today's date in ISO format (YYYY-MM-DD).
\def\isodate{\leavevmode\hbox{\the\year-\twodigits\month-\twodigits\day}}
\def\twodigits#1{\ifnum#1<10 0\fi\the#1}


\usepackage[utf8]{inputenc}
\usepackage[es-tabla]{babel}
\usepackage{graphicx}
\usepackage{caption}
\usepackage{float}
\usepackage{epigraph}
\usepackage{multicol}
\usepackage[bottom,flushmargin]{footmisc} 

\usepackage{tikz}
\usetikzlibrary{arrows.meta}
\usetikzlibrary{babel}
\usetikzlibrary{decorations.pathreplacing}

\usepackage{hyperref}
\hypersetup{pdfauthor={Cristian Adrián Ontivero}}
\usetikzlibrary{shapes}
\graphicspath{{imgs/}}

\newlength\tindent
\setlength{\tindent}{\parindent}
\setlength{\parindent}{0pt}
\renewcommand{\indent}{\hspace*{\tindent}}

\renewcommand*{\tableautorefname}{Tabla}
\renewcommand*{\figureautorefname}{Figura}

%These tell TeX which packages to use.
\usepackage{array,epsfig}
\usepackage{amsmath}
\usepackage{amsfonts}
\usepackage{amssymb}
\usepackage{amsxtra}
\usepackage{amsthm}
\usepackage{mathrsfs}

%Here I define some theorem styles and shortcut commands for symbols I use often
\theoremstyle{definition}
\newtheorem{defn}{Definición}
\newtheorem{thm}{Teorema}
\newtheorem{cor}{Corolario}
\newtheorem*{rmk}{Remark}
\newtheorem{lem}{Lema}
\newtheorem*{joke}{Joke}

\newtheorem{ex}{Ejemplo}
\newcommand{\exautorefname}{Ejemplo}

\newtheorem{exercise}{Ejercicio}
\newcommand{\exerciseautorefname}{Ejercicio}

\newtheorem{soln}{Solución}
\newtheorem{prop}{Proposición}

\newcommand{\lra}{\longrightarrow}
\newcommand{\ra}{\rightarrow}
\newcommand{\surj}{\twoheadrightarrow}
\newcommand{\graph}{\mathrm{graph}}
\newcommand{\bb}[1]{\mathbb{#1}}
\newcommand{\Z}{\bb{Z}}
\newcommand{\Q}{\bb{Q}}
\newcommand{\R}{\bb{R}}
\newcommand{\C}{\bb{C}}
\newcommand{\N}{\bb{N}}
\newcommand{\M}{\mathbf{M}}
\newcommand{\m}{\mathbf{m}}
\newcommand{\MM}{\mathscr{M}}
\newcommand{\HH}{\mathscr{H}}
\newcommand{\Om}{\Omega}
\newcommand{\Ho}{\in\HH(\Om)}
\newcommand{\bd}{\partial}
\newcommand{\del}{\partial}
\newcommand{\bardel}{\overline\partial}
\newcommand{\textdf}[1]{\textbf{\textsf{#1}}\index{#1}}
\newcommand{\img}{\mathrm{img}}
\newcommand{\ip}[2]{\left\langle{#1},{#2}\right\rangle}
\newcommand{\inter}[1]{\mathrm{int}{#1}}
\newcommand{\exter}[1]{\mathrm{ext}{#1}}
\newcommand{\cl}[1]{\mathrm{cl}{#1}}
\newcommand{\ds}{\displaystyle}
\newcommand{\vol}{\mathrm{vol}}
\newcommand{\cnt}{\mathrm{ct}}
\newcommand{\osc}{\mathrm{osc}}
\newcommand{\LL}{\mathbf{L}}
\newcommand{\UU}{\mathbf{U}}
\newcommand{\support}{\mathrm{support}}
\newcommand{\AND}{\;\wedge\;}
\newcommand{\OR}{\;\vee\;}
\newcommand{\Oset}{\varnothing}
\newcommand{\st}{\ni}
\newcommand{\wh}{\widehat}

%Pagination stuff.
\setlength{\topmargin}{-.3 in}
\setlength{\oddsidemargin}{0in}
\setlength{\evensidemargin}{0in}
\setlength{\textheight}{9.in}
\setlength{\textwidth}{6.5in}
\pagestyle{empty}

\begin{document}
\begin{center}
  {\LARGE Minimización de DFAs \\[.3cm]
	Teorema de Myhill-Nerode y Algoritmo de Moore }\\[.2cm]
  Cristian Adrián Ontivero \\[.05cm]%
  \isodate
\end{center}

\vspace{0.2 cm}

El \textit{índice}\footnote{En inglés: \textit{index}. } de una relación de
equivalencia, es la cantidad de clases de equivalencia que genera. Una relación
de equivalencia $R$ sobre un conjunto $G$ se dice \textit{invariante por
derecha}\footnote{En inglés \textit{right invariant}, también \textit{right
congruent}.} si: 

\[ \forall x,y \in G (x R y \implies \forall z \in G, xz R yz)\]


Teorema Myhill-Nerode

Las siguientes tres afirmaciones son equivalentes:
\begin{enumerate}
  \item El conjunto $L \subseteq \Sigma^*$ es reconozido por un autómata finito.
  \item $L$ es la unión de algunas de las clases de equivalencia de una relación de equivalencia invariante por derecha y de índice finito. 
  \item Sea $R$ una relación definida por: 
	\[ \forall x, y \in \Sigma^* : xRy \iff \forall z \in \Sigma^*(xz \in L \iff yz \in L) \]
  Entonces $R$ es de índice finito.
\end{enumerate}

Demostración:
 \[ TODO \]


Dado un lenguaje $L$ sobre un alfabeto finito $\Sigma$, se conoce por
\textit{relación de equivalencia Myhill-Nerode}, a la relación $\sim$ de
equivalencia invariante por derecha, asociada al lenguaje $L$ y definida
por: 

\[ \forall x,y \in \Sigma^* : x \sim y \iff \forall z \in \Sigma^* (xz \in L \iff yz \in L)\]


\bibliography{refs}
\bibliographystyle{unsrt}

\end{document}



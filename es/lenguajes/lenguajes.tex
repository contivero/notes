\documentclass[spanish]{article}

\usepackage{color}
\definecolor{darkblue}{RGB}{49,130,189}

% Access modes
\newcommand{\acmo}[1]{\underline{\texttt{#1}}}

% Print today's date in ISO format (YYYY-MM-DD).
\def\isodate{\leavevmode\hbox{\the\year-\twodigits\month-\twodigits\day}}
\def\twodigits#1{\ifnum#1<10 0\fi\the#1}


\usepackage[utf8]{inputenc}
\usepackage[es-tabla]{babel}
\usepackage{graphicx}
\usepackage{caption}
\usepackage{float}
\usepackage{epigraph}
\usepackage{multicol}
\usepackage{enumitem}
\usepackage[bottom,flushmargin]{footmisc} 

\usepackage{tikz}
\usetikzlibrary{arrows.meta}
\usetikzlibrary{babel}
\usetikzlibrary{decorations.pathreplacing}

\usepackage{hyperref}
\hypersetup{pdfauthor={Cristian Adrián Ontivero}}
\usetikzlibrary{shapes}
\graphicspath{{imgs/}}

\newlength\tindent
\setlength{\tindent}{\parindent}
\setlength{\parindent}{0pt}
\renewcommand{\indent}{\hspace*{\tindent}}

\renewcommand*{\tableautorefname}{Tabla}
\renewcommand*{\figureautorefname}{Figura}

%These tell TeX which packages to use.
\usepackage{array,epsfig}
\usepackage{amsmath}
\usepackage{amsfonts}
\usepackage{amssymb}
\usepackage{amsxtra}
\usepackage{amsthm}
\usepackage{mathrsfs}

%Here I define some theorem styles and shortcut commands for symbols I use often
\theoremstyle{definition}
\newtheorem{defn}{Definición}
\newtheorem{thm}{Teorema}
\newtheorem{cor}{Corolario}
\newtheorem*{rmk}{Remark}
\newtheorem{lem}{Lema}
\newtheorem*{joke}{Joke}

\newtheorem{ex}{Ejemplo}
\newcommand{\exautorefname}{Ejemplo}

\newtheorem{exercise}{Ejercicio}
\newcommand{\exerciseautorefname}{Ejercicio}

\newtheorem{soln}{Solución}
\newtheorem{prop}{Proposición}

\newcommand{\lra}{\longrightarrow}
\newcommand{\ra}{\rightarrow}
\newcommand{\surj}{\twoheadrightarrow}
\newcommand{\graph}{\mathrm{graph}}
\newcommand{\bb}[1]{\mathbb{#1}}
\newcommand{\Z}{\bb{Z}}
\newcommand{\Q}{\bb{Q}}
\newcommand{\R}{\bb{R}}
\newcommand{\C}{\bb{C}}
\newcommand{\N}{\bb{N}}
\newcommand{\M}{\mathbf{M}}
\newcommand{\m}{\mathbf{m}}
\newcommand{\MM}{\mathscr{M}}
\newcommand{\HH}{\mathscr{H}}
\newcommand{\Om}{\Omega}
\newcommand{\Ho}{\in\HH(\Om)}
\newcommand{\bd}{\partial}
\newcommand{\del}{\partial}
\newcommand{\bardel}{\overline\partial}
\newcommand{\textdf}[1]{\textbf{\textsf{#1}}\index{#1}}
\newcommand{\img}{\mathrm{img}}
\newcommand{\ip}[2]{\left\langle{#1},{#2}\right\rangle}
\newcommand{\inter}[1]{\mathrm{int}{#1}}
\newcommand{\exter}[1]{\mathrm{ext}{#1}}
\newcommand{\cl}[1]{\mathrm{cl}{#1}}
\newcommand{\ds}{\displaystyle}
\newcommand{\vol}{\mathrm{vol}}
\newcommand{\cnt}{\mathrm{ct}}
\newcommand{\osc}{\mathrm{osc}}
\newcommand{\LL}{\mathbf{L}}
\newcommand{\UU}{\mathbf{U}}
\newcommand{\support}{\mathrm{support}}
\newcommand{\AND}{\;\wedge\;}
\newcommand{\OR}{\;\vee\;}
\newcommand{\Oset}{\varnothing}
\newcommand{\st}{\ni}
\newcommand{\wh}{\widehat}

%Pagination stuff.
\setlength{\topmargin}{-.3 in}
\setlength{\oddsidemargin}{0in}
\setlength{\evensidemargin}{0in}
\setlength{\textheight}{9.in}
\setlength{\textwidth}{6.5in}
\pagestyle{empty}

\begin{document}
\begin{center}
  {\LARGE Pequeña Cronología y Bibliografía Sobre Teoría de Lenguajes}\\[.2cm]
  Cristian Adrián Ontivero \\[.05cm]%
  \isodate%
\end{center}

\vspace{0.2 cm}

A continuación se presenta una (muy) acotada cronología de los principales resultados
que sentaron las bases de lenguajes formales y autómatas. La mayoría se toma
de~\cite{Gre81}, que presenta un \textit{survey} muy completo del área hasta el
81. 

\begin{itemize}
  \item[1936] Máquinas de Turing~\cite{Tur36}.
  %\item[1943] Post - Formal Reductionsof the General Combinatorial decision problem.
  %\item[1947] Post - Recursive unsolvability of a problem of Thue.

Ya por 1947, las gramáticas (o sistemas generadores) y máquinas en la cima de la
jerarquía de Chomsky eran conocidas, y se sabía que eran equivalentes.

  \item[1951] Kleene~\cite{Kle51}:

\begin{itemize}
\item establece correspondencia entre ``conjuntos regulares''\footnote{Llamados \textit{regular events} por el.} y automatas finitos.
\item da origen a las expresiones regulares.
\item demuestra que la clase de conjuntos aceptados por automatas finitos es la clase más pequeña que contiene a todos los conjuntos finitos, y es cerrada bajo union, producto (concatenación), y clausura de Kleene.
\end{itemize}

  \item [1957-58] Teorema Myhill-Nerode\footnote{En el libro seguido por la
    cátedra, corresponde al Teorema 3.1.} \cite{Nerode58}, el cual da
    condiciones necesarias y suficientes para que un lenguaje sea regular.

  \item[1959] \begin{enumerate}[leftmargin=.5cm, label=\alph*.]
    \item Jerarquía de Chomsky~\cite{Cho59}.

    \item Autómata finito no deterministico (NFA)~\cite{RS59}. En 1976, Rabin y
    Scott ganaron el Turing Award:

    \textit{[\ldots]for their joint paper ``Finite Automata and Their Decision
      Problem,'' which introduced the idea of nondeterministic machines, which
    has proved to be an enormously valuable concept. Their (Scott~\&~Rabin)
  classic paper has been a continuous source of inspiration for subsequent work
in this field.}
\end{enumerate}

  \item[1961] Lema de Bombeo (o ``Teorema $uvwxy$'')~\cite{BarHillel61}.

    Posteriormente hecho más fuerte en 1968~\cite{Ogden1968}.
   \item[1964] Kuroda completa la jerarquía de Chomsky de lenguajes y máquinas,
     al definir los automatas linearmente acotados no deterministas
     (NLBAs)~\cite{Kur64}.

\end{itemize}

De aquí en adelante, la teoría formal de lenguajes diverge de la lingüística
matemática o computacional, y a partir de 1965 aproximadamente, se puede
considerar a la teoría de lenguajes formales como una rama de teórica de la
informática.


\section*{Bibliografía}

Algunos posibles libros de referencia son:

\begin{itemize}
  \item \underline{Formal Languages and Their Relation to Automata} (Hopcroft \& Ullman)

Primer libro en juntar y organizar información sobre teoría de autómatas y
lenguajes formales, hasta ese entonces dispersa en articulos cientificos. Si
bien el libro es del 69, para ese entonces ya se encontraba completa la
jerarquía de Chomsky, y \textbf{es el seguido por la cátedra}.
El libro se encuentra fuera de impresión, pero puede conseguirse fácilmente
versiones en PDFs, e imprimirlas si lo desean.

\item \underline{Introduction to Automata Theory, Languages and Computation} (Hopcroft,
  Motwani \& Ullman)

  Sucesor del libro anterior, más adaptado a \textit{``undergraduates''}. Puede
  ser más amigable como una primera introducción, pero muchos se quejan de que
  se perdió la brevedad del original, y varios de mejores ejercicios fueron
  quitados.~\cite{wikiCinderella}

\item \underline{Introduction to the Theory of Computation} (Sipser)
  
  El libro ``estandar'' de muchos cursos similares en la actualidad. Altamente
  recomendado, si bien en lo que respecta a la materia, sólo nos incumbe la
  primera parte (la segunda trata sobre teoría de complejidad, en la cuál no
  nos adentraremos).

\item \underline{Automata and Computability} (Kozen)

  Otra alternativa. Se lo suele recomendar como una introducción a la materia.

\end{itemize}

\bibliography{refs}
\bibliographystyle{alpha}

\end{document}



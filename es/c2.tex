\documentclass[]{book}

\usepackage[utf8]{inputenc}
\usepackage[yyyymmdd,hhmmss]{datetime}
%These tell TeX which packages to use.
\usepackage{array,epsfig}
\usepackage{amsmath}
\usepackage{amsfonts}
\usepackage{amssymb}
\usepackage{amsxtra}
\usepackage{amsthm}
\usepackage{mathrsfs}
\usepackage{color}

%Here I define some theorem styles and shortcut commands for symbols I use often
\theoremstyle{definition}
\newtheorem{defn}{Definition}
\newtheorem{thm}{Theorem}
\newtheorem{cor}{Corollary}
\newtheorem*{rmk}{Remark}
\newtheorem{lem}{Lemma}
\newtheorem*{joke}{Joke}
\newtheorem{ex}{Example}
\newtheorem*{soln}{Solution}
\newtheorem{prop}{Proposition}

\newcommand{\lra}{\longrightarrow}
\newcommand{\ra}{\rightarrow}
\newcommand{\surj}{\twoheadrightarrow}
\newcommand{\graph}{\mathrm{graph}}
\newcommand{\bb}[1]{\mathbb{#1}}
\newcommand{\Z}{\bb{Z}}
\newcommand{\Q}{\bb{Q}}
\newcommand{\R}{\bb{R}}
\newcommand{\C}{\bb{C}}
\newcommand{\N}{\bb{N}}
\newcommand{\M}{\mathbf{M}}
\newcommand{\m}{\mathbf{m}}
\newcommand{\MM}{\mathscr{M}}
\newcommand{\HH}{\mathscr{H}}
\newcommand{\Om}{\Omega}
\newcommand{\Ho}{\in\HH(\Om)}
\newcommand{\bd}{\partial}
\newcommand{\del}{\partial}
\newcommand{\bardel}{\overline\partial}
\newcommand{\textdf}[1]{\textbf{\textsf{#1}}\index{#1}}
\newcommand{\img}{\mathrm{img}}
\newcommand{\ip}[2]{\left\langle{#1},{#2}\right\rangle}
\newcommand{\inter}[1]{\mathrm{int}{#1}}
\newcommand{\exter}[1]{\mathrm{ext}{#1}}
\newcommand{\cl}[1]{\mathrm{cl}{#1}}
\newcommand{\ds}{\displaystyle}
\newcommand{\vol}{\mathrm{vol}}
\newcommand{\cnt}{\mathrm{ct}}
\newcommand{\osc}{\mathrm{osc}}
\newcommand{\LL}{\mathbf{L}}
\newcommand{\UU}{\mathbf{U}}
\newcommand{\support}{\mathrm{support}}
\newcommand{\AND}{\;\wedge\;}
\newcommand{\OR}{\;\vee\;}
\newcommand{\Oset}{\varnothing}
\newcommand{\st}{\ni}
\newcommand{\wh}{\widehat}

\newcommand{\adv}{$\mathcal{A}~$}
\newcommand{\Enc}{$\mathsf{Enc}$}

%Pagination stuff.
\setlength{\topmargin}{-.3 in}
\setlength{\oddsidemargin}{0in}
\setlength{\evensidemargin}{0in}
\setlength{\textheight}{9.in}
\setlength{\textwidth}{6.5in}
\pagestyle{empty}



\begin{document}


\begin{center}
  {\Large Criptografía y Seguridad \the\year~(72.44)\\[.2cm]
Guía 2 - Criptografía Simétrica}\\
\end{center}

\vspace{0.2 cm}

\subsection*{Ejercicios}
\begin{enumerate}
\item Dado un Criptosistema con las siguientes características:
  \begin{itemize}
    \item Espacio de mensajes $\mathcal{M} =\{a, b\}$, con:
      \[P [M=a] =0,25; P[M=b]=0,75\]
  \item Espacio de claves $\mathcal{K} = \{k_1,k_2,k_3\}$, donde $\Pi = (\mathsf{Gen}, \mathsf{Enc}, \mathsf{Dec})$ genera una clave según las siguientes probabilidades:
    \[P[K=k_1]=0,5 ; P[K=k_2]=P[K=k_3]=0,25\]
  \end{itemize}
  El algoritmo $\mathsf{Enc}$ está definido por la tabla


\begin{tabular}{c|cc}
    & a & b \\ \hline
$k_1$ & 1 & 2 \\
$k_2$ & 2 & 3 \\
$k_3$ & 3 & 4 \\
\end{tabular}

Por lo que el espacio de cifrados es $\mathcal{C}=\{ \mathsf{Enc}_k(x)~|~x \in M
\land k \in K\} = \{1,2,3,4\}$. Se pide:
\begin{enumerate}
\item Hallar la distribución de probabilidades de $C$.
\item Demostrar de las cuatro maneras vistas en la teoría, que el sistema no tiene secreto perfecto.
\end{enumerate}
\item Probar o encontrar un contraejemplo de la siguiente afirmación:

En un criptosistema que posee la propiedad de secreto perfecto se cumple que
para toda distribución sobre el espacio de mensajes $\mathcal{M}$, para todo
$m,m' \in \mathcal{M}$ y para todo $c\in \mathcal{C}$:
\[Pr[M=m|C=c] = Pr[M=m'|C=c]\]

\item Para los siguientes incisos, considerar el alfabeto inglés (de 26 símbolos):
\begin{enumerate}
\item Demostrar que si se encripta un solo símbolo, entonces el cifrado de rotación tiene secreto perfecto.
\item ¿Cuál es el mayor tamaño que puede tener el espacio de mensajes $\mathcal{M}$ como para que el cifrado de sustitución monoalfabética posea secreto perfecto?
\item Mostrar cómo usar el cifrado de Vigenère para encriptar una palabra de longitud $t$ y tener secreto perfecto.
\end{enumerate}
\item Dado un criptosistema $\Pi = (\mathsf{Gen}, \mathsf{Enc}, \mathsf{Dec})$
  de Vigenère, sobre un espacio de mensajes $\mathcal{M} = \Sigma^3$ (strings de
  3 caracteres), donde $\Sigma$ es el alfabeto inglés. El algoritmo
  $\mathsf{Gen}$ elige primero el período $t$ de la clave en forma aleatoria y uniforme, con $t
  \in \{1, 2, 3\}$. Luego, elige la clave $k \in \Sigma^*$, tal que $k = t$.
\begin{enumerate}
\item Mostrar un ejemplo de cómo el criptosistema cifra un mensaje.
\item Calcular la probabilidad de éxito del experimento
  $\text{PrivK}_{\mathcal{A},\Pi}^\text{CPA}$ para un adversario que hace lo siguiente:
\begin{enumerate}
  \item $\mathcal{A}$ emite primero los mensajes ${m_0 = \verb+aab+, m_1 = \verb+abb+}$.
  \item Al recibir el cifrado $c$, $\mathcal{A}$ emite \verb+'0'+ si el primer símbolo de $c$ es igual al segundo, y emite \verb+'1'+ en caso contrario.
\end{enumerate}
\item De acuerdo a lo calculado en el inciso anterior, ¿tiene secreto perfecto?
\end{enumerate}

\item Demostrar que los siguientes cifrados son vulnerables a un ataque de texto plano elegido (\textit{chosen-plaintext attack}).
\begin{enumerate}
  \item Cifrado de sustitución monoalfabética.
  \item Cifrado de Vigenère.
\end{enumerate}
\item Con el modo ECB, si hay un error en un bloque del texto cifrado transmitido, solamente afecta al bloque de texto claro correspondiente.
\begin{enumerate}
\item En el modo CBC (ver figura) un error de un bit en P1, ¿a través de cuántos bloques de texto cifrado se propaga?
\item En el modo CBC, un error de un bit en C1, a través de cuántos bloques de texto descrifrado se propaga?
\item Si se produce un error de un bit en la transmisión de un carácter del texto cifrado en modo CFB de ocho bits, ¿hasta dónde se propaga el error?
\end{enumerate}
\item Considerando el siguiente cifrado de bloque:
  \[\mathsf{Enc}_k(m) = m \cdot k \mod 32\]
\begin{enumerate}
\item ¿Cuál son el tamaño del bloque, y el espacio efectivo de la clave?
\item Encriptar el mensaje \verb+24 17 26 25 12+ usando modo CBC con vector de
  inicialización $IV = 19$ y clave $k = 7$.
\item Desencriptar en modo CBC.
\end{enumerate}

\item Una involución (o función involutiva), es una función que es su propia
  inversa. Para DES, una clave $k$ se llama débil (\textit{weak key}), si es tal que
  el cifrado es involutivo, es decir: $\forall m, Enc_k(Enc_k(m)) = m$. Analizar
  por qué una clave formada por todos sus bits en 0, o todos sus bits en 1, es
  una clave débil de DES. Hay otras dos claves débiles, ¿cuáles serían?
\end{enumerate}
\end{document}



\documentclass{article}

\usepackage[utf8]{inputenc}
\usepackage{caption}
\usepackage{graphicx}

% Change language to change quotation style
\usepackage[english]{babel}
\usepackage{csquotes}

\usepackage[
  backend=biber,
  style=alphabetic,
  sorting=nyt
]{biblatex}
\addbibresource{refs.bib}

\usepackage{hyperref}
\hypersetup{pdfauthor={Cristian Adrián Ontivero}}

\graphicspath{{imgs/}}

\newlength\tindent%
\setlength{\tindent}{\parindent}
\setlength{\parindent}{0pt}
\renewcommand{\indent}{\hspace*{\tindent}}

%These tell TeX which packages to use.
\usepackage{array,epsfig}
\usepackage{amsmath}
\usepackage{amsfonts}
\usepackage{amssymb}
\usepackage{amsxtra}
\usepackage{amsthm}
\usepackage{mathrsfs}

%Pagination stuff.
\setlength{\topmargin}{-.3 in}
\setlength{\oddsidemargin}{0in}
\setlength{\evensidemargin}{0in}
\setlength{\textheight}{9.in}
\setlength{\textwidth}{6.5in}
\pagestyle{empty}

\begin{document}

To argue about something, we better agree on what that something is. The term
object-oriented programming has been used for a wide spectrum of programming
languages. For a purist relying on etymology, OOP means Alan Kay's OOP, where
message-passing is the central concept. In practice, Java is the highest
ranking programming language (according to GitHub and the Tiobe index)
usually touted as OOP, followed by Python and C++. Kay famously said\cite{Kay97}:

\begin{displayquote}
 \enquote{\textit{I invented the term object-oriented, and I can tell you, I did not
 have C++ in mind}}
\end{displayquote}

which gives an idea of how different are some of the conceptions of what OOP is.
As Moseley \& Marks put it\cite{MM06}, object-oriented has been used for \enquote{everything from Java-style
class-based to Self-style prototype-based languages, from single-dispatch to
CLOS-style multiple-dispatch languages, and from traditional passive objects
to the active/actor styles}.

For starters, we will follow Peter Wegner's proposal that objects, classes
and inheritance are the essential ingredients of object-oriented
languages\cite{Weg87}.

%Similar to how Asimov said \enquote{Properly read, the Bible is the most
%potent force for atheism ever conceived}, Blosch's Effective Java reads as a
%case against Java.

As such, we will start with Objects, after all, as it's usually said, in OOP
\enquote{everything is an object}.
\subsection{Objects}
Objects are the amalgamation of properties (variables or fields), behavior
(methods), and identity. Thus, from the get-go we have these three things
coupled.

% To think: operations don't compose as easily as in FP.
You now have two types of functions, those associated to a class (methods)
and those that take an instance and operate on it (static methods, e.g.\
those usually defined in a \textit{Util} class or similarly named). When we need to mix both, functions don't easily compose:

%You need to name the parameter!
%fun(Book b) {
  %List<Book> books = Util.getBooksBy(b.getAuthor())
%}
%vs.
%getBooksBy . getAuthor


One upside though, is that it allows IDEs to give suggestions while coding
(e.g. IntelliJ Idea's intellisense).

\subsection{Classes}
TODO

\subsection{Implementation Inheritance}
\textit{Implementation inheritance} refers to when a class extends another, as opposed
to \textit{interface inheritance}, which refers to when a class implements an
interface, or an interface extends another.

\begin{itemize}
  \item Inheritance means that in order to understand the behaviour of one object, I have to understand the behaviour of all its ancestors first.
  \item \enquote{Implementation inheritance causes the same intertwining and
brittleness that have been observed when goto statements are overused. As a
result, OO systems often suffer from complexity and lack of reuse}
---John Ousterhout\cite{Ous97}

\end{itemize}

I've seen firsthand wasted work because something that wasn't supposed to be
extended, was; while using Java, a class had not been declared \texttt{final}
(to disallow extending it), when it should have been. Kotlin properly learnt
from Java's mistakes, and makes any class \texttt{final} (nonextensible)
by default, requiring the user to explicitely use the keyword \texttt{open}
when allowing inheritance is desired. This is in accordance with Item~17
of~\cite{Blo08} (Design and document for inheritance or else prohibit it).

The mantra ``favor composition over inheritance'' (which seems to have
originated with \cite{GoF94}) is well-known by now. The whole of Item~16
in~\cite{Blo08} is worth reading regarding the pitfalls of inheritance, but
as a rough summary, in it we find mentioned two issues with overriding
methods:

\begin{itemize}
  \item Inheritance violates encapsulation\cite{DBLP:conf/oopsla/Snyder86}:
  the superclass's instance variables are accesible to the subclasses.
  Subclasses are dependent on the implementation details of its superclass to
  function, usually requiring the subclass to evolve in tandem with its
  superclass.
  \item New methods in the superclass can break conditions on which its
  subclasses, were Blosch concludes by saying \enquote{This is not a purely
  theoretical problem. Several security holes of this nature had to be fixed
  when \texttt{Hashtable} and \texttt{Vector} were retrofitted to participate
  in the Collections Framework.}
\end{itemize}

In Java, even if we don't override methods, we still have a (minor, but
present) issue that if we added a method to the subclass, and in a later
release the superclass acquires a method with the same signature but
different return type, our subclass won't compile anymore (I don't know if
other languages have a solution to this or if they also stop compiling).

%citing examples from the Java libraries. If the Java library writers
%themselves have problems deciding when it is appropriate to inherit, what can
%we expect from the average programmer?


% Almost verbatim
  %Unlike method invocation, inheritance violates encapsulation. A subclass
  %depends on the implementation details of its superclass to function. The
  %superclass's implementation may change from release to release, and if it
  %does, the subclass may break, even though its code has not been touched. As
  %a consequence, a subclass must evolve in tandem with its superclass,
  %unless the superclass's authors have designed and documented it
  %specifically for the purpose of being extended.\cite{Blo08}
% Almost verbatim


% Verbatim
  %A related cause of fragility in subclasses is that their superclass can
  %acquire new methods in subsequent releases. Suppose a program depends for
  %its security on the fact that all elements inserted into some collection
  %satisfy some predicate. This can be guaranteed by subclassing the
  %collection and overriding each method capable of adding an element to
  %ensure that the predicate is satisfied before adding the element. This
  %works fine until a new method capable of inserting an element is added to
  %the superclass in a subsequent release. Once this happens, it becomes
  %possible to add an \enquote{illegal} element merely by invoking the new
  %method, which is not overriden in the subclass. This is not a purely
  %theoretical problem. Several security holes of this nature had to be fixed
  %when \texttt{Hashtable} and \texttt{Vector} were retrofitted to participate
  %in the Collections Framework.\cite{Blo08}
% Verbatim

If you use inheritance where composition is appropriate, you needlessly
expose implementation details.

As for when to use inheritance, Blosch gives two main questions to pose
before inheriting:
\begin{enumerate}
  \item Before having class B extend class A, ask yourself: Is every B really
  an A? If you cannot truthfully answer yes to this question, B should not
  extend A.
  \item Does the class that you contemplate extending have any flaws in its
  API? If so, are you comfortable propagating those flaws into your class's
  API?
\end{enumerate}
The second question poses a further issue: inheritance propagates any flaws
in the superclass's API the superclass's API; composition, however, lets you
design a new API that hides these flaws.

The \textit{fragile base-class problem} (FBCP)~\cite{MS98} refers to the
possibility that changes to a superclass can cause problems in its
subclasses, thus making the base class `fragile'.

This can be particularly problematic on open systems, but also an issue in
closed ones~\cite{MS98}. % TODO Check the paper, which references others for this statement.

%This is a negative results or null results paper, i.e. results that fail to
%show an effect.
To be fair, there is at least a study\cite{S17} of open source projects were
the authors didn't find any bugs related to the FBCP, indicating that the
issue could be more theoretical than practical.

Another known problem in the literature is the so called
\textit{Circle-ellipse problem}\cite{Cli94, Maj98} (or
\textit{Square-rectangle problem}), which occurs as a violation of the
\textit{Liskov Substitution Principle} (the L in the SOLID design principles)

In~\cite{MY93} Matsuoka and Yonezawa coined the term \textit{inheritance
anomaly} to to refer to the problems that arise when using inheritance with
concurrency.

\newpage

Functional Programming:
\begin{itemize}
  \item no side effects
  \item persistence (vs. ephimeral data structures)
  \item nonstrict (lazy) evaluation + higher-order functions = better glue.
  \item functions as first class citizens. One can make use of higher-order
  functions in languages such as C or Java, but it is way more cumbersome.
  \item Haskell can support the missing OO concepts (object identity and
  state) through (among other things) Monads. Most (all) aspects of SideEffect-based programming
  can be modelled with Monads, while still retaining ReferentialTransparency,
  which is one of the claimed AdvantagesOfFunctionalProgramming. (See
  OnMonads.)
  \item Higher-kinded types.
\end{itemize}

Philip Wadler, one of the implementators of Generics in Java said:

\enquote{[The core of Functional Languages] is not arbitrary. Their core is
something that was written down once by a logician, and once by a computer
scientist. That is, it was not invented, but discovered. Most of you use
programming languages that are invented, and you can tell, can't you.}

Difference between parametric and ad-hoc polymorphism.

OO: notion of object identity

Most ObjectOrientedLanguages associate polymorphism with subtyping. Most FP
languages make use of ParametricPolymorphism.

Some languages make it easier than others, of course. For example, iterators
always struck me as a needless multiplication of entities; I'd rather be
mapping a function over the collection.

\newpage

CMU eliminated object-oriented programming from their introductory CS
curriculum, stating:
\enquote{it is both anti-modular and anti-parallel by its very nature, and hence
unsuitable for a modern CS curriculum.}\cite{HL11}


As an appeal to authority (consider it merely as an additional curiosity if
you dislike using \textit{argumentum ad verecundiam}):

% the quote was spoken to Bob Crawford in 1988 or so and appeared in the
%journal TUG LINES produced by the Turbo user's group, Issue 32 (August
%September 1989)
\enquote{Object-oriented programming is an exceptionally bad idea which could only have originated in California.}
Edsger Dijkstra

\enquote{object-oriented design is the roman numerals of computing.}
Rob Pike

\enquote{The phrase \enquote{object-oriented} means a lot of things. Half are obvious, and the other half are mistakes.}
Paul Graham

\enquote{I find OOP technically unsound. It attempts to decompose the world in terms
of interfaces that vary on a single type. To deal with the real problems you
need multisorted algebras - families of interfaces that span multiple types.
I find OOP philosophically unsound. It claims that everything is an object.
Even if it is true it is not very interesting - saying that everything is an
object is saying nothing at all. I find OOP methodologically wrong. It starts
with classes. It is as if mathematicians would start with axioms. You do not
start with axioms - you start with proofs. Only when you have found a bunch
of related proofs, can you come up with axioms. You end with axioms. The same
thing is true in programming: you have to start with interesting algorithms.
Only when you understand them well, can you come up with an interface that
will let them work.}
---Stepanov\cite{Gra}

\enquote{C++ is a horrible language. ... C++ leads to really, really bad
design choices. ... In other words, the only way to do good, efficient, and
system-level and portable C++ ends up to limit yourself to all the things
that are basically available in C. And limiting your project to C means that
people don't screw that up, and also means that you get a lot of programmers
that do actually understand low-level issues and don't screw things up with
any idiotic \enquote{object model} crap.}
---Linus Torvalds\cite{Tor}

\enquote{Object-oriented programming languages present many challenges for
program understanding. The extensive use of subtyping and dynamic dispatch
make understanding the flow of values and control a nontrivial task.
Moreover, small source code changes can have unexpected and nonlocal
effects.}\cite{RT01}

In~\cite{Hat98} we find \enquote{an OO case study which details a development project in which project staff tracked in detail corrective-maintenance costs and other maintenance issues}.
According to the author
\enquote{The developers found it much harder to trace faults in the OO C++
design than in the conventional C design. Although this may simply be a
feature of C++, it appears to be more generally observed in the testing of OO
systems, largely due to the distorted and frequently nonlocal relationship
between cause and effect: the manifestation of a failure may be a
\enquote{long way away} from the fault that led to it.}\cite{Hat98}

\enquote{I think the lack of reusability comes in object-oriented languages,
not in functional languages. Because the problem with object-oriented
languages is they've got all this implicit environment that they carry around
with them. You wanted a banana but what you got was a gorilla holding the
banana and the entire jungle. If you have referentially transparent code, if
you have pure functions --- all the data comes in its input arguments and
everything goes out and leaves no state behind---it's incredibly reusable.
You can just reuse it here, there, and everywhere. When you want to use it in
a different project, you just cut and paste this code into your new project.}
--- Joe Armstrong\cite{Sei09}

\enquote{The central core of functional programming is the idea of nonmutable
state---that \texttt{x} isn't the name of a location in memory; it's a value.
So it can't change.}
--- Joe Armstrong\cite{Sei09}

%\enquote{I characterize functional programming---that is, purely functional
%programming, where side effects are somehow really relegated to a world of
%their own---as a radical and elegant attack on the whole enterprise of
%writing programs.}
%---Simon Peyton Jones

\enquote{Sometimes, the elegant implementation is just a function. Not a
method. Not a class. Not a framework. Just a function.}
– John Carmack

\enquote{The kind of things we capture in javadoc are the kinds of things
that ought to be told to a compiler. If it's worth telling another
programmer, it's worth telling the compiler, I think.}
--- Guy Steele\cite{Sei09}

\enquote{I used to be enamored of object-oriented programming. I’m now
finding myself leaning toward believing that it is a plot designed to destroy
joy.}
– Eric Allman

\enquote{I have apologized profusely over the last 20 years for making up the
term object-oriented, because as soon as it started to be misapplied I
realized that I should have used a much more process-oriented term for it.}
---Alan Kay\cite{Kay97}

Dijkstra wrote a letter to the budget council urging them not to replace
Haskell by Java in the introductory courses, saying: \blockquote{Haskell,
though not perfect, is of a quality that is several orders of magnitude
higher than Java.}\cite{Dij01}

Some hacker news quotes

\enquote{[students beginning in] OOP environments often seem to devolve into
strange taxonomical, metaphysical questions: \enquote{Is a Car a Vehicle? Or
does it implement Drivable? Is a Window a Widget itself, or does it only own
Widgets? If it's a Widget itself, then could a Window contain sub-windows?}}


In~\cite{Car96}, Luca Cardelli noted that \enquote{Object-oriented style is
intrinsically less efficient than procedural style. In pure object-oriented
style, every routine is supposed to be a (virtual) method. This introduces
additional indirections [\(\dots\)]}; \enquote{[object-oriented] languages
have extremely poor modularity properties with respect to class extension and
modification}; \enquote{In fairness, designers of object-oriented languages
did not simply \enquote{forget} to include properties such as good type
systems and good modularity: the issues are intrinsically more complex than
in procedural languages.}; \enquote{The type systems of most object-oriented languages are not
expressive enough; programmers must often resort to dynamic checking or to
unsafe features, damaging the robustness of their programs}

\enquote{Conventional imperative and object-oriented programs suffer greatly
from both state-derived and control-derived complexity.}\cite{MM06}

%Verbatim from wikipedia
A study by Potok et al.\ has shown no significant difference in productivity
between OOP and procedural approaches\cite{Pot99}.
%Verbatim from wikipedia

Further references:

Joe Armstrong's Why OO sucks\cite{Arm01}.

The object-oriented model has gotten time wrong.
Are we there yet?\cite{Hic09}

% TODO read:
% http://lambda-the-ultimate.org/node/489
% http://lambda-the-ultimate.org/classic/message4857.html
% http://www.softpanorama.org/SE/anti_oo.shtml
% http://igw.tuwien.ac.at/fit/fit03/oop.html
% https://www.westga.edu/~bquest/1997/object.html
% http://www.oocities.org/tablizer/oopbad.htm
% https://www.yegor256.com/2016/08/15/what-is-wrong-object-oriented-programming.html
% http://okmij.org/ftp/Computation/Subtyping/
% https://crypto.stanford.edu/~blynn/c/object.html
% http://raganwald.com/2014/03/31/class-hierarchies-dont-do-that.html
% https://softwareengineering.stackexchange.com/questions/212515/why-is-tight-coupling-between-functions-and-data-bad
% https://wiki.haskell.org/The_Monad.Reader/Issue3/Functional_Programming_vs_Object_Oriented_Programming
% http://mergeconflict.com/coupling-in-object-oriented-programming/

% Great comment! http://lambda-the-ultimate.org/node/4235#comment-64975

% Shajan Miah's critique
% https://web.archive.org/web/20070306123626/http://members.aol.com/shaz7862/critique.htm

To think:
\begin{center}
\(>>\) With OO, a person is likely to think "Well, I have a bunch of Books
that have Authors, which will be purchased by Customers who have
creditCardNumbers)", whereas what does a normal human begin think when you
ask them to produce a functional solution?

Depends on what value of "normal" you're using! But probably something fairly
similar, since these entities can be represented by types. The difference
(well, a difference) with FP is that operations over these types are often
easier to compose.
\end{center}


%\bibliography{refs}
%\bibliographystyle{unsrt}
\printbibliography

\end{document}
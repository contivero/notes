\documentclass{article}

\usepackage[utf8]{inputenc}
\usepackage{caption}
\usepackage{float}

\usepackage{hyperref}
\hypersetup{pdfauthor={Cristian Adrián Ontivero}}

\usepackage{adjustbox}
\usepackage{tikz}
\usetikzlibrary{positioning}
\usetikzlibrary{external}
\tikzexternalize[prefix=tikz/, up to date check=md5]
\usepackage{ebproof}
\usepackage{booktabs}
\usepackage{multicol}
% Set items in enumerate environment to letters.
\usepackage{enumitem}
\setenumerate[0]{label=(\alph*)}
\usepackage{forest}
\forestset{%
  % Tree following the style of Chiswell & Hodges' Mathematical Logic
  chtree/.style={
    baseline,
    for tree={
      inner sep=1.4pt,
      circle,
      draw,
      s sep'+=20pt,
      fit=band,
    },
  },
  c phantom/.style={draw=none, no edge},
}

\graphicspath{{imgs/}}

\newlength\tindent%
\setlength{\tindent}{\parindent}
\setlength{\parindent}{0pt}
\renewcommand{\indent}{\hspace*{\tindent}}

%These tell TeX which packages to use.
\usepackage{array, epsfig}
\usepackage{amsmath}
\usepackage{amsfonts}
\usepackage{amssymb}
\usepackage{amsxtra}
\usepackage{amsthm}
\usepackage{mathrsfs}

% Use in the array environment to set a column to text
\newcolumntype{R}{>$r<$}
\newcolumntype{L}{>$l<$}
\newcolumntype{C}{>$c<$}

%Here I define some theorem styles and shortcut commands for symbols I use often
\theoremstyle{definition}

\newcommand{\ra}{\rightarrow}
\newcommand{\surj}{\twoheadrightarrow}
\newcommand{\bb}[1]{\mathbb{#1}}
\newcommand{\Z}{\bb{Z}}
\newcommand{\Q}{\bb{Q}}
\newcommand{\R}{\bb{R}}
\newcommand{\C}{\bb{C}}
\newcommand{\N}{\bb{N}}

\newcommand{\lra}{\leftrightarrow}
\newcommand{\true}{\text{T}}
\newcommand{\false}{\text{F}}
% Logic rules
\newcommand{\ii}{\footnotesize (\(\rightarrow\)I)}
\newcommand{\ie}{\footnotesize (\(\rightarrow\)E)}
\newcommand{\ci}{\footnotesize (\(\wedge\)I)}
\newcommand{\ce}{\footnotesize (\(\wedge\)E)}
\newcommand{\di}{\footnotesize (\(\vee\)I)}
\newcommand{\de}{\footnotesize (\(\vee\)E)}
\newcommand{\bi}{\footnotesize (\(\leftrightarrow\)I)}
\newcommand{\be}{\footnotesize (\(\leftrightarrow\)E)}
\newcommand{\negi}{\footnotesize (\(\neg\)I)}
\newcommand{\nege}{\footnotesize (\(\neg\)E)}
\newcommand{\raa}{\footnotesize (RAA)}
\newcommand*\cir[1]{\tikz[baseline=(char.base)]{
            \node[shape=circle,draw,inner sep=1pt] (char) {\footnotesize #1};}}

            \newcommand{\danda}[2]{%
    \tikz[baseline=(tocancel.base)]{%
        \node[inner sep=0pt,outer sep=0pt] (tocancel) {\(#1\)};
        \node[inner sep=0pt,outer sep=1pt,above right=0mm of tocancel] (end)
        {\cir{#2}};
        \draw (tocancel.south west) -- (end.260);
    }\hspace{-1.2em}%
}%
%Pagination stuff.
\setlength{\topmargin}{-.3 in}
\setlength{\oddsidemargin}{0in}
\setlength{\evensidemargin}{0in}
\setlength{\textheight}{9.in}
\setlength{\textwidth}{6.5in}
\pagestyle{empty}

\begin{document}

% Comment for faster compilation, uncomment for final document
%2.2.2)
\begin{enumerate}
  \item
  \[
    \begin{prooftree}
      \hypo{\phi}
      \hypo{\psi}
      \hypo{\chi}
      \infer2[\ci]{(\psi\land\chi)}
      \infer2[\ci]{(\phi \land (\psi\land\chi))}
    \end{prooftree}
  \]
  \item
  \[
    \begin{prooftree}
      \hypo{\chi}
      \hypo{\phi}
      \infer2[\ci]{(\chi\land\phi)}
    \end{prooftree}
  \]

  \item
  \[
    \begin{prooftree}
      \hypo{\phi}
      \hypo{\phi}
      \infer2[\ci]{(\phi\land\phi)}
      \hypo{\phi}
      \infer2[\ci]{((\phi\land\phi)\land \phi)}
    \end{prooftree}
  \]
  \item
  \[
    \begin{prooftree}
      \hypo{\phi}
      \hypo{\psi}
      \infer2[\ci]{(\phi\land\psi)}
      \hypo{\phi}
      \hypo{\psi}
      \infer2[\ci]{(\phi\land\psi)}
      \infer2[\ci]{((\phi\land\psi)\land(\phi\land\psi))}
    \end{prooftree}
  \]

\end{enumerate}

2.3.1)
\begin{enumerate}
  \item
  \[
    \begin{prooftree}
      \hypo{(\phi\land\psi)}
      \infer1[\ce]{\phi}
      \hypo{(\phi\land\psi)}
      \infer1[\ce]{\phi}
      \infer2[\ci]{(\phi\land\phi)}
    \end{prooftree}
  \]

  \item \(\land\) associativity in one direction:
  \[
    \begin{prooftree}
      \hypo{((\phi\land\psi)\land\chi)}
      \infer1[\ce]{(\phi\land\psi)}
      \infer1[\ce]{\phi}
      \hypo{((\phi\land\psi)\land\chi)}
      \infer1[\ce]{(\phi\land\psi)}
      \infer1[\ce]{\psi}
      \hypo{((\phi\land\psi)\land\chi)}
      \infer1[\ce]{\chi}
      \infer2[\ci]{(\psi\land\chi)}
      \infer2[\ci]{(\phi\land(\psi\land\chi))}
    \end{prooftree}
  \]
  \item
  \[
    \begin{prooftree}
      \hypo{(\psi\land\chi)}
      \infer1[\ce]{\chi}
      \hypo{\phi}
      \infer2[\ci]{(\chi\land\phi)}
    \end{prooftree}
  \]
  \item
  \[
    \begin{prooftree}
      \hypo{(\phi\land(\psi\land\chi))}
      \infer1[\ce]{(\psi\land\chi)}
      \infer1[\ce]{\chi}
      \hypo{(\phi\land(\psi\land\chi))}
      \infer1[\ce]{\phi}
      \infer2[\ci]{(\chi\land\phi)}
      \hypo{(\phi\land(\psi\land\chi))}
      \infer1[\ce]{(\psi\land\chi)}
      \infer1[\ce]{\psi}
      \infer2[\ci]{((\chi\land\phi)\land\psi)}
    \end{prooftree}
  \]
\end{enumerate}

2.3.2)
  \[
    \begin{prooftree}
      \hypo{D}
      \infer[rule style=no rule]1{\phi}
      \hypo{D'}
      \infer[rule style=no rule]1{\psi}
      \infer2[\ci]{(\phi\land\psi)}
      \infer1[\ce]{\phi}
    \end{prooftree}
  \]
  Just do \(D\) to derive \(\phi\).

2.3.3) Suppose \(\{\phi_1, \phi_2\} \vdash \psi\). Then:
  \[
    \begin{prooftree}
      \hypo{(\phi_1\land\phi_2)}
      \infer1[\ce]{\phi_1}
      \hypo{(\phi_1\land\phi_2)}
      \infer1[\ce]{\phi_2}
      \infer2[(assumption)]{\psi}
    \end{prooftree}
  \]
Suppose \(\{\phi_1\land\phi_2\} \vdash \psi\). Then:
  \[
    \begin{prooftree}
      \hypo{\phi_1}
      \hypo{\phi_2}
      \infer2[\ci]{(\phi_1\land\phi_2)}
      \infer1[(assumption)]{\psi}
    \end{prooftree}
  \]

2.4.1)
\begin{enumerate}
  \item \(f\) is differentiable \(\ra\) \(f\) is continuous.
  \item \(x\) is positive \(\ra\) \(x\) has a square root.
  \item \(b \neq 0 \ra \dfrac{ab}{b} = a\)
\end{enumerate}

2.4.2) Proof of \(\vdash ((\phi\land\psi)\ra(\psi\land\phi))\)
\begin{enumerate}
  \item
  \[
    \begin{prooftree}
      \hypo{\danda{(\phi\land\psi)}{1}}
      \infer1[\ce]{\psi}
      \hypo{\danda{(\phi\land\psi)}{1}}
      \infer1[\ce]{\phi}
      \infer2[\ci]{(\psi\land\phi)}
      \infer[left label=\cir{1}]1[\ii]{((\phi\land\psi)\ra(\psi\land\phi))}
    \end{prooftree}
  \]
  \item Proof of \(\vdash ((\psi\ra\chi)\ra((\phi\ra\psi)\ra(\phi\ra\chi)))\)
  \[
    \begin{prooftree}
      \hypo{\danda{\phi}{1}}
      \hypo{\danda{(\phi\ra\psi)}{2}}
      \infer2[\ie]{\psi}
      \hypo{\danda{(\psi\ra\chi)}{3}}
      \infer2[\ie]{\chi}
      \infer[left label=\cir{1}]1[\ii]{(\phi\ra\chi)}
      \infer[left label=\cir{2}]1[\ii]{((\phi\ra\psi)\ra(\phi\ra\chi))}
      \infer[left label=\cir{3}]1[\ii]{((\psi\ra\chi)\ra((\phi\ra\psi)\ra(\phi\ra\chi)))}
    \end{prooftree}
  \]
\end{enumerate}
2.4.3)
\begin{enumerate}
  \item \(\vdash (\phi \ra (\psi \ra \phi))\)
  \item \(\{\phi\}\vdash (\phi \ra (\psi \ra \phi))\)
  \item \(\{(\phi\land\psi)\} \vdash (\psi \ra (\psi \land \phi))\)
  \item \(\vdash \phi \ra \phi\)
\end{enumerate}

2.4.4)
\begin{enumerate}
  \item
  \[
    \begin{prooftree}
      \hypo{\danda{\psi}{1}}
      \infer[left label=\cir{1}]1[\ii]{(\psi\ra\psi)}
      \infer1[\ii]{(\phi\ra(\psi\ra\psi))}
    \end{prooftree}
  \]
  \item
  \[
    \begin{prooftree}
      \hypo{\danda{\phi}{1}}
      \infer[left label=\cir{1}]1[\ii]{(\phi\ra\phi)}
      \hypo{\danda{\psi}{2}}
      \infer[left label=\cir{2}]1[\ii]{(\psi\ra\psi)}
      \infer2[\ci]{((\phi\ra\phi)\land(\psi\ra\psi))}
    \end{prooftree}
  \]
  \item
  \[
    \begin{prooftree}
      \hypo{\danda{\phi}{1}}
      \hypo{\danda{(\phi\ra(\theta\ra\psi))}{3}}
      \infer2[\ie]{(\theta\ra\psi)}
      \hypo{\danda{\theta}{1}}
      \infer2[\ie]{\psi}
      \infer[left label=\cir{1}]1[\ii]{(\phi\ra\psi)}
      \infer[left label=\cir{2}]1[\ii]{(\theta\ra(\phi\ra\psi))}
      \infer[left label=\cir{3}]1[\ii]{((\phi\ra(\theta\ra\psi))\ra(\theta\ra(\phi\ra\psi)))}
    \end{prooftree}
  \]
  \item
  \[
    \begin{prooftree}
      \hypo{\danda{\phi}{1}}
      \hypo{(\phi\ra\psi)}
      \infer2[\ie]{\psi}
      \hypo{\danda{\phi}{1}}
      \hypo{(\phi\ra\chi)}
      \infer2[\ie]{\chi}
      \infer2[\ci]{(\psi\land\chi)}
      \infer[left label=\cir{1}]1[\ii]{(\phi\ra(\psi\land\chi))}
    \end{prooftree}
  \]
  \item TODO
  \item
  \[
    \begin{prooftree}
      \hypo{\danda{(\phi\land\psi)}{1}}
      \infer1[\ce]{\psi}
      \hypo{\danda{(\phi\land\psi)}{1}}
      \infer1[\ce]{\phi}
      \hypo{(\phi\ra(\psi\ra\chi))}
      \infer2[\ie]{(\psi\ra\chi)}
      \infer2[\ie]{\chi}
      \infer[left label=\cir{1}]1[\ii]{((\phi\land\psi)\ra\chi)}
    \end{prooftree}
  \]
  \item
  \[
    \begin{prooftree}
      \hypo{\danda{\phi}{1}}
      \hypo{\danda{(\phi\ra\psi)}{3}}
      \infer2[\ie]{\psi}
      \hypo{\danda{(\psi\ra\theta)}{2}}
      \infer2[\ie]{\theta}
      \infer[left label=\cir{1}]1[\ii]{(\phi\ra\chi)}
      \infer[left label=\cir{2}]1[\ii]{((\psi\ra\theta)\ra(\phi\ra\chi))}
      \infer[left label=\cir{3}]1[\ii]{((\phi\ra\psi)\ra((\psi\ra\theta)\ra(\phi\ra\chi)))}
    \end{prooftree}
  \]
  \item
  \[
    \begin{prooftree}
      \hypo{\danda{\phi}{1}}
      \hypo{\danda{(\phi\ra(\psi\land\theta))}{3}}
      \infer2[\ie]{(\psi\land\theta)}
      \infer1[\ce]{\theta}
      \infer[left label=\cir{1}]1[\ii]{(\phi\ra\theta)}
      \hypo{\danda{\phi}{2}}
      \hypo{\danda{(\phi\ra(\psi\ra\theta))}{3}}
      \infer2[\ie]{(\psi\land\theta)}
      \infer1[\ce]{\psi}
      \infer[left label=\cir{2}]1[\ii]{(\phi\ra\psi)}
      \infer2[\ci]{(\psi\ra\theta)\land(\psi\ra\psi)}
      \infer[left label=\cir{3}]1[\ii]{((\phi\ra(\psi\land\theta))\ra((\psi\ra\theta)\land(\psi\ra\psi)))}
    \end{prooftree}
  \]
\end{enumerate}

2.4.5) \((\Rightarrow)\) Suppose \(\{\phi\}\vdash\psi\). Then:
  \[
    \begin{prooftree}
      \hypo{\danda{\phi}{1}}
      \infer1[(assumption)]{\psi}
      \infer[left label=\cir{1}]1[\ii]{(\phi\ra\psi)}
    \end{prooftree}
  \]
\((\Rightarrow)\) Suppose \(\vdash(\phi\ra\psi)\). Then:
  \[
    \begin{prooftree}
      \hypo{\phi}
      \hypo{(\phi\ra\psi)}
      \infer2[\ie]{\psi}
    \end{prooftree}
  \]

2.4.6) Suppose \(D_1\) is a derivation whose undischarged assumptions are all in \(\Gamma \cup \{\phi\}\). Take \(D*\) to be a copy of \(D_1\) where each ocurrence of \(\phi\) is replaced by \(\danda{\phi}{1}\)~. Then take the derivation \(D'_1\) to be:
  \[
    \begin{prooftree}
      \hypo{D*}
      \infer[rule style=no rule]1{\psi}
      \infer[left label=\cir{1}]1[\ii]{(\phi\ra\psi)}
    \end{prooftree}
  \]
\((\Rightarrow)\) Now suppose \(D_2\) is a derivation whose undischarged
assumptions are all in \(\Gamma\). Take \(D'_2\) to be:
  \[
    \begin{prooftree}
      \hypo{\phi}
      \hypo{D_1}
      \infer[rule style=no rule]1{(\phi\ra\psi)}
      \infer2[\ii]{\psi}
    \end{prooftree}
  \]

2.5.1)
\begin{enumerate}
  \item
  \[
    \begin{prooftree}
      \hypo{\phi}
      \hypo{(\phi\lra\psi)}
      \infer1[\be]{(\phi\ra\psi)}
      \infer2[\ii]{\psi}
    \end{prooftree}
  \]
  \item
  \[
    \begin{prooftree}
      \hypo{\danda{\phi}{1}}
      \infer[left label=\cir{1}]1[\ii]{(\phi\ra\phi)}
      \hypo{\danda{\phi}{2}}
      \infer[left label=\cir{2}]1[\ii]{(\phi\ra\phi)}
      \infer2[\bi]{(\phi\ra\phi)}
    \end{prooftree}
  \]
  \item
  \[
    \begin{prooftree}
      \hypo{\danda{\phi}{1}}
      \hypo{(\phi\lra\psi)}
      \infer1[\be]{(\phi\ra\psi)}
      \infer2[\ie]{\psi}
      \hypo{(\psi\lra\chi)}
      \infer1[\be]{(\psi\ra\chi)}
      \infer2[\ie]{\chi}
      \infer[left label=\cir{1}]1[\ii]{(\phi\ra\chi)}

      \hypo{\danda{\chi}{2}}
      \hypo{(\psi\lra\chi)}
      \infer1[\be]{(\chi\ra\psi)}
      \infer2[\ie]{\psi}
      \hypo{(\phi\lra\psi)}
      \infer1[\be]{(\psi\ra\phi)}
      \infer2[\ie]{\phi}
      \infer[left label=\cir{2}]1[\ii]{(\chi\ra\phi)}
      \infer2[\bi]{(\phi\lra\chi)}
    \end{prooftree}
  \]
  \item
  \[
    \begin{prooftree}
      \hypo{\danda{\psi}{1}}
      \infer1[\ii]{(\phi\ra\psi)}
      \hypo{\danda{\phi}{3}}
      \infer1[\ii]{(\psi\ra\phi)}
      \infer2[\bi]{(\phi\lra\psi)}

      \hypo{((\phi\lra\psi)\lra\chi)}
      \infer1[\be]{((\phi\lra\psi)\ra\chi)}
      \infer2[\ie]{\chi}
      \infer[left label=\cir{1}]1[\ii]{(\psi\ra\chi)}

      \hypo{\danda{\phi}{3}}
      \hypo{\danda{\chi}{2}}
      \hypo{((\phi\lra\psi)\lra\chi)}
      \infer1[\be]{(\chi\ra(\phi\lra\psi))}
      \infer2[\ie]{(\phi\lra\psi)}
      \infer1[\be]{(\phi\ra\psi)}
      \infer2[\ie]{\psi}
      \infer[left label=\cir{2}]1[\be]{(\chi\ra\psi)}
      \infer2[\ie]{(\psi\lra\chi)}
      \infer[left label=\cir{3}]1[\be]{(\phi\ra(\psi\lra\chi))}

    \end{prooftree}
  \]
  \item
  \[
    \begin{prooftree}
      \hypo{\danda{\psi}{1}}
      \infer[left label=\cir{1}]1[\ii]{(\psi\ra\psi)}
      \hypo{\danda{\psi}{2}}
      \infer[left label=\cir{2}]1[\ii]{(\psi\ra\psi)}
      \infer2[\bi]{(\psi\lra\psi)}

      \hypo{(\phi\lra(\psi\lra\psi))}
      \infer1[\be]{((\psi\lra\psi)\lra\phi)}
      \infer2[\ie]{\phi}
    \end{prooftree}
  \]
\end{enumerate}

2.5.2) We know that \(\forall \phi,\psi \in S, \phi \sim \psi \text{ iff }
\vdash (\phi\lra\phi)\). We have to show that \(\sim\) is an equivalence
relation.
\begin{itemize}
  \item Reflexive: Let \(\phi\) be a formula. By 2.5.1(b), \(\vdash(\phi\lra\phi)\), then \(\phi\sim\phi\).
  \item Symmetric: Let \(\phi,\psi\) be formulas such that \(\phi\sim\psi\).
  Then \(\vdash(\phi\lra\psi\). By example 2.5.1,
  \(\{(\phi\lra\psi)\}\vdash(\psi\lra\phi)\), so \(\psi\sim\phi\).
  In more detail:
  TODO (add derivation as in the transitive case)

  \item Transitive: Let \(\phi,\psi,\chi\) be formulas such that
  \(\phi\sim\psi\) and \(\psi\sim\chi\). Then \(\vdash(\phi\lra\psi)\) and
  \(\vdash(\psi\lra\chi)\). Then by exercise 2.5.1 (c),
  \(\{(\phi\lra\psi),(\psi\lra\chi)\}\vdash (\phi\lra\chi)\), so
  \((\phi\sim\chi)\). In more detail, we have:
  \[
    \begin{prooftree}
      \hypo{D}
      \infer[rule style=no rule]1{(\phi\lra\psi)}
    \end{prooftree}\hspace{6mm}
    \begin{prooftree}
      \hypo{D'}
      \infer[rule style=no rule]1{(\psi\lra\chi)}
    \end{prooftree}\hspace{6mm}
    \begin{prooftree}
      \hypo{D'}
      \infer[rule style=no rule]1{(\psi\lra\chi)}
    \end{prooftree}
  \]
  \[
    \begin{prooftree}
      \hypo{D_c}
      \infer[rule style=no rule]1{(\phi\lra\chi)}
    \end{prooftree}
  \]
  where \(D_c\) is all the derivation we did in 2.5.1(c).

\end{itemize}

2.5.3) Suppose we have a derivation \(D\) with no undischarged assumptions. Let \(\phi\) by any statement. Then:
\[
    \begin{prooftree}
      \hypo{D}
      \infer[rule style=no rule]1{\psi}
      \hypo{\danda{(\phi\lra\psi)}{1}}
      \infer1[\be]{(\psi\ra\phi)}
      \infer2[\ie]{\phi}
      \infer[left label=\cir{1}]1[\ii]{((\phi\lra\psi)\ra\phi)}

      \hypo{D}
      \infer[rule style=no rule]1{\psi}
      \infer1[\ii]{(\phi\ra\psi)}

      \hypo{\danda{\phi}{2}}
      \infer1[\ii]{(\psi\ra\phi)}
      \infer2[\bi]{(\phi\lra\psi)}
      \infer[left label=\cir{2}]1[\ii]{(\phi\ra(\psi\ra\phi))}
      \infer2[\ii]{((\phi\lra\psi)\lra\phi)}
    \end{prooftree}
\]

2.5.4) Sequent Rule \(\lra I\): If the sequents \((\Gamma \cup \{\phi\}\vdash
\psi)\) and \((\Delta \cup \{\psi\} \vdash \phi)\) are correct, then so is
\((\Gamma \cup \Delta\vdash (\phi\lra\psi))\).

Sequent Rule \(\lra E\): If the sequent \((\Gamma \vdash (\phi\lra\psi))\) is
correct, then so are \((\Gamma\vdash(\phi\ra\psi))\) and \((\Gamma \vdash
(\psi \ra \phi))\).

2.6.1)
\begin{enumerate}
  \item
  \[
    \begin{prooftree}
      \hypo{\danda{(\phi\land(\neg \phi))}{1}}
      \infer1[\ce]{\phi}
      \hypo{\danda{(\phi\land(\neg \phi))}{1}}
      \infer1[\ce]{(\neg\phi)}
      \infer2[\nege]{\bot}
      \infer[left label=\cir{1}]1[\negi]{(\neg(\phi\land(\neg \phi)))}
    \end{prooftree}
  \]
  \item
  \[
    \begin{prooftree}
      \hypo{\danda{(\neg(\phi\ra\psi))}{2}}
      \hypo{\danda{\psi}{1}}
      \infer1[\ii]{(\phi\ra\psi)}
      \infer2[\nege]{\bot}
      \infer[left label=\cir{1}]1[\negi]{(\neg\psi)}
      \infer[left label=\cir{2}]1[\ii]{((\neg(\phi\ra\psi))\ra(\neg\psi))}
    \end{prooftree}
  \]
  \item
  \[
    \begin{prooftree}
      \hypo{\danda{(\phi\land\psi)}{2}}
      \infer1[\ce]{\psi}

      \hypo{\danda{(\phi\land\psi)}{2}}
      \infer1[\ce]{\phi}
      \hypo{\danda{(\phi\ra(\neg\psi))}{1}}
      \infer2[\ie]{(\neg\psi)}
      \infer2[\nege]{\bot}
      \infer[left label=\cir{1}]1[\negi]{(\neg(\phi\ra(\neg\psi)))}
      \infer[left label=\cir{2}]1[\ii]{((\phi\land\psi)\ra(\neg(\phi\ra(\neg\psi))))}
    \end{prooftree}
  \]
  \item
  \[
    \begin{prooftree}
      \hypo{((\neg(\phi\land\psi))\land\phi)}
      \infer1[\ce]{(\neg(\phi\land\psi))}

      \hypo{((\neg(\phi\land\psi))\land\phi)}
      \infer1[\ce]{\phi}
      \hypo{\danda{\psi}{1}}
      \infer2[\ie]{(\phi\land\psi)}
      \infer2[\nege]{\bot}
      \infer[left label=\cir{1}]1[\negi]{(\neg\psi)}
    \end{prooftree}
  \]
  \item
  \[
    \begin{prooftree}
      \hypo{\danda{(\neg\psi)}{2}}
      \hypo{\danda{\phi}{1}}
      \hypo{(\phi\ra\psi)}
      \infer2[\ie]{\psi}
      \infer2[\nege]{\bot}
      \infer[left label=\cir{1}]1[\negi]{(\neg\phi)}
      \infer[left label=\cir{2}]1[\ii]{((\neg\psi)\ra(\neg\phi))}
    \end{prooftree}
  \]
  \item
  \[
    \begin{prooftree}
      \hypo{\danda{(\phi\land(\neg\psi))}{1}}
      \infer1[\ce]{\phi}

      \hypo{(\phi\ra\psi)}
      \infer2[\ie]{\psi}

      \hypo{\danda{(\phi\land(\neg\psi))}{1}}
      \infer1[\ce]{(\neg\psi)}
      \infer2[\nege]{\bot}
      \infer[left label=\cir{1}]1[\negi]{(\neg(\phi\land(\neg\psi)))}
    \end{prooftree}
  \]
\end{enumerate}

2.6.2)
\begin{enumerate}
  \item
  \[
    \begin{prooftree}
      \hypo{\danda{\phi}{2}}
      \hypo{\danda{(\neg\psi)}{1}}
      \hypo{((\neg\psi)\ra(\neg\phi))}
      \infer2[\ie]{(\neg\phi)}
      \infer2[\nege]{\bot}
      \infer[left label=\cir{1}]1[\raa]{\psi}
      \infer[left label=\cir{2}]1[\negi]{(\phi\ra\psi)}
    \end{prooftree}
  \]
  \item
  \[
    \begin{prooftree}
      \hypo{\danda{\phi}{1}}
      \hypo{\danda{(\neg\phi)}{2}}
      \infer2[\nege]{\bot}
      \infer1[\raa]{\psi}
      \infer[left label=\cir{1}]1[\ii]{(\phi\ra\psi)}
      \hypo{\danda{(\neg(\phi\ra\psi))}{3}}
      \infer2[\nege]{\bot}
      \infer[left label=\cir{2}]1[\raa]{\phi}
      \infer[left label=\cir{3}]1[\ii]{((\neg(\phi\ra\psi))\ra\phi)}
    \end{prooftree}
  \]
  \item
  \[
    \begin{prooftree}
      \hypo{\danda{\phi}{2}}
      \hypo{\danda{(\neg\phi)}{1}}
      \infer2[\nege]{\bot}
      \infer1[\raa]{\psi}
      \infer[left label=\cir{1}]1[\ii]{((\neg\phi)\ra\psi)}
      \infer[left label=\cir{2}]1[\ii]{(\phi\ra((\neg\phi)\ra\psi))}
    \end{prooftree}
  \]
  \item
    \begin{prooftree}[separation=1em]
      \hypo{\danda{\phi}{1}}
      \hypo{\danda{(\neg\phi)}{4}}
      \infer2[\nege]{\bot}
      \infer1[\raa]{\psi}
      \infer[left label=\cir{1}]1[\ii]{(\phi\to\psi)}
      \hypo{\danda{\psi}{2}}
      \hypo{\danda{(\neg\psi)}{3}}
      \infer2[\nege]{\bot}
      \infer1[\raa]{\phi}
      \infer[left label=\cir{2}]1[\ii]{(\psi\to\phi)}
      \infer2[\bi]{(\phi\lra\psi)}
      \hypo{(\neg(\phi\lra\psi))}
      \infer2[\nege]{\bot}
      \infer[left label=\cir{3}]1[\raa]{\psi}
      \infer[left label=\cir{4}]1[\ii]{(\neg(\phi\to\psi))}

      \hypo{(\neg(\phi\lra\psi))}

      \hypo{\danda{\psi}{6}}
      \infer1[\ii]{(\phi\to\psi)}
      \hypo{\danda{\phi}{5}}
      \infer1[\ii]{(\psi\to\phi)}
      \infer2[\bi]{(\phi\lra\psi)}
      \infer2[\nege]{\bot}
      \infer[left label=\cir{5}]1[\negi]{(\neg\phi)}
      \infer[left label=\cir{6}]1[\ii]{(\psi\to(\neg\phi))}
      \infer2[\bi]{((\neg\phi)\lra\psi)}
    \end{prooftree}%
\end{enumerate}

2.7.1)
\begin{enumerate}
  \item
  \[
    \begin{prooftree}
      \hypo{\danda{\phi}{1}}
      \infer1[\di]{(\phi\lor\psi)}
      \infer[left label=\cir{1}]1[\ii]{(\phi\ra(\phi\lor\psi))}
    \end{prooftree}
  \]
  \item
  \[
    \begin{prooftree}
      \hypo{(\neg(\phi\lor\psi))}
      \hypo{\danda{\phi}{1}}
      \infer1[\di]{(\phi\lor\psi)}
      \infer2[\nege]{\bot}
      \infer[left label=\cir{1}]1[\negi]{(\neg\phi)}

      \hypo{(\neg(\phi\lor\psi))}
      \hypo{\danda{\psi}{2}}
      \infer1[\di]{(\phi\lor\psi)}
      \infer2[\nege]{\bot}
      \infer[left label=\cir{2}]1[\negi]{(\neg\psi)}

      \infer2[\di]{((\neg\phi)\land(\neg\psi))}
    \end{prooftree}
  \]
  \item
  \[
    \begin{prooftree}
      \hypo{\danda{(\neg\phi)}{1}}
      \infer1[\di]{((\neg\phi)\lor\psi)}
      \hypo{\danda{(\neg(\neg(\phi\lor\psi)))}{3}}
      \infer2[\nege]{\bot}
      \infer[left label=\cir{1}]1[\raa]{\phi}

      \hypo{\danda{\phi}{2}}
      \hypo{\danda{(\phi\ra\psi)}{4}}
      \infer2[\ie]{\psi}
      \infer1[\di]{((\neg\phi)\lor\psi)}
      \hypo{\danda{(\neg((\neg\phi)\lor\psi))}{3}}
      \infer2[\nege]{\bot}
      \infer[left label=\cir{2}]1[\negi]{(\neg\phi)}
      \infer2[\nege]{\bot}
      \infer[left label=\cir{3}]1[\raa]{((\neg\phi)\lor\psi)}
      \infer[left label=\cir{4}]1[\raa]{((\phi\ra\psi)\ra((\neg\phi)\lor\psi))}
    \end{prooftree}
  \]
\end{enumerate}

2.7.2)
\begin{enumerate}
  \item
  \[
    \begin{prooftree}
      \hypo{(\phi\lor\psi)}
      \hypo{\danda{\phi}{1}}
      \infer1[\di]{(\psi\lor\phi)}
      \hypo{\danda{\psi}{1}}
      \infer1[\di]{(\psi\lor\phi)}
      \infer[left label=\cir{1}]3[\de]{(\psi\lor\phi)}
    \end{prooftree}
  \]
  \item
  \[
    \begin{prooftree}
      \hypo{(\phi\lor\psi)}
      \hypo{\danda{\phi}{1}}
      \hypo{(\phi\ra\chi)}
      \infer2[\ie]{\chi}
      \hypo{\danda{\psi}{1}}
      \hypo{(\psi\ra\chi)}
      \infer2[\ie]{\chi}
      \infer[left label=\cir{1}]3[\de]{\chi}
    \end{prooftree}
  \]
  \item
  \[
    \begin{prooftree}
      \hypo{(\phi\lor\psi)}

      \hypo{\danda{\phi}{1}}
      \hypo{(\neg\phi)}
      \infer2[\nege]{\bot}
      \infer1[\raa]{\psi}
      \hypo{\danda{\psi}{1}}
      \infer[left label=\cir{1}]3[\de]{\psi}
    \end{prooftree}
  \]
  \item
  \[
    \begin{prooftree}
      \hypo{\danda{(\phi\lor\psi)}{2}}

      \hypo{\danda{\phi}{1}}
      \hypo{((\neg\phi)\land(\neg\psi))}
      \infer1[\ce]{(\neg\phi)}
      \infer2[\nege]{\bot}

      \hypo{\danda{\psi}{1}}
      \hypo{(\neg\psi)}
      \infer2[\nege]{\bot}
      \infer[left label=\cir{1}]3[\de]{\bot}
      \infer[left label=\cir{2}]1[\negi]{(\neg(\phi\lor\psi))}

    \end{prooftree}
  \]
  \item
  \[
    \begin{prooftree}
      \hypo{\danda{((\neg\phi)\lor(\neg\psi))}{2}}

      \hypo{(\phi\land\psi)}
      \infer1[\ce]{\phi}
      \hypo{\danda{(\neg\phi)}{1}}
      \infer2[\nege]{\bot}

      \hypo{(\phi\land\psi)}
      \infer1[\ce]{\psi}
      \hypo{\danda{(\neg\psi)}{1}}
      \infer2[\nege]{\bot}
      \infer[left label=\cir{1}]3[\de]{\bot}
      \infer[left label=\cir{2}]1[\negi]{(\neg((\neg\phi)\lor(\neg\psi)))}
    \end{prooftree}
  \]
\end{enumerate}
3.1.1)

\begin{enumerate}
  \item
    \begin{forest}
      chtree
      [, label=right:\(\land\)
        [, label=right:\(p\)]
        [, label=right:\(q\)]
      ]
    \end{forest}
  \item
    \begin{forest}
      chtree
      [, label=right:\(p\)]
    \end{forest}
  \item
    \begin{forest}
      chtree
      [, label=right:\(\ra\)
        [, label=right:\(p\)]
        [, label=right:\(\ra\)
          [, label=right:\(q\)]
          [, label=right:\(\ra\)
            [, label=right:\(r\)]
            [, label=right:\(s\)]
          ]
        ]
      ]
    \end{forest}
  \item
    \begin{forest}
      chtree
      [, label=right:\(\lor\)
        [, label=right:\(\neg\)
          [, label=right:\(\ra\)
            [, label=right:\(p_2\)]
            [, label=right:\(\lra\)
              [, label=right:\(p_1\)]
              [, label=right:\(p_0\)]
            ]
          ]
        ]
        [, label=right:\(\ra\)
          [, label=right:\(p_2\)]
          [, label=right:\(\bot\)]
        ]
      ]
    \end{forest}
\end{enumerate}

3.1.2)
\begin{enumerate}
  \item \((\neg\bot)\)
  \item \((\neg(p_2\lor p_0))\)
  \item \( (((((p_0 \land p_1)\land p_2)\land p_4) \land p_5) \land p_6) \)
\end{enumerate}

3.1.3)
\begin{enumerate}
  \item \(\{p, q\}\)
  \item \(\{p\}\)
  \item \(\{p, q, r, s\}\)
  \item \(\{p_0, p_1, p_2\}\). Note that \(\bot\) does not go here.
  \item \(\{p_1, p_2, p_3, p_4, p_5, p_6\}\)
  \item etc...
\end{enumerate}

3.2.1) \(p_1, p_0, (p_1 \lra p_0), (p_2\ra(p1\lra p_0)), (\neg(p_2\ra(p_1\lra p_0))), (p_2\ra\bot), ((\neg(p_2\ra(p_1\lra p_0)))\lor(p_2\ra\bot)) \)

3.2.2)
\begin{multicols}{3}
\begin{enumerate}[label=\underline{\(\pi_{\arabic*}\):}]
  \item
    \begin{forest}
      chtree
      [, label=right:\(p_0\)]
    \end{forest}
  \item
    \begin{forest}
      chtree
      [, label=right:\(\neg\)
        [, label=right:\(p_0\)]
      ]
    \end{forest}
  \item
    \begin{forest}
      chtree
      [, label=right:\(\land\)
        [, label=right:\(p_0\)]
        [, label=right:\(p_1\)]
      ]
    \end{forest}
  \item
    \begin{forest}
      chtree
      [, label=right:\(\lor\)
        [, label=right:\(\neg\)
          [, label=right:\(p_0\)]
        ]
        [, label=right:\(p_1\)]
      ]
    \end{forest}
  \item
    \begin{forest}
      chtree
      [, label=right:\(\ra\)
        [, label=right:\(\neg\)
          [, label=right:\(p_0\)]
        ]
        [, label=right:\(\neg\)
          [, label=right:\(p_1\)]
        ]
      ]
    \end{forest}
  \item
    \begin{forest}
      chtree
      [, label=right:\(\land\)
        [, label=right:\(\ra\)
          [, label=right:\(p_0\)]
          [, label=right:\(p_1\)]
        ]
        [, label=right:\(\neg\)
          [, label=right:\(p_2\)]
        ]
      ]
    \end{forest}
\end{enumerate}
\end{multicols}

3.2.3)
\begin{align*}
  \delta(\pi_1) & = 1 \\
  \delta(\pi_2) & = 1 + 3 = 4\\
  \delta(\pi_3) & = 1 + 3 + 1 = 5\\
  \delta(\pi_4) & = 1 + 3 + 3 + 1 = 8\\
  \delta(\pi_5) & = 1 + 3 + 3 + 3 + 1 = 11\\
  \delta(\pi_5) & = 1 + 3 + 1 + 3 + 3 + 1 = 12\\
  \delta(\pi)   & = \text{\#atoms} + 3\;\text{\#connectors} \\
                & = \text{length of associated formula}
\end{align*}

3.2.4)
\begin{center}
  \adjustbox{valign=t, minipage=.1\textwidth}{
    \begin{forest}
      chtree
      [, label=left:0, label=right:\(\chi\)]
    \end{forest}%
  }
  \adjustbox{valign=t, minipage=.16\textwidth}{
    \begin{forest}
      chtree
        [, label=left:\(2+m\), label=right:\(\neg\)
          [, label=left:\(m\)]
        ]
    \end{forest}%
  }
  \adjustbox{valign=t, minipage=.16\textwidth}{
    \begin{forest}
      chtree
        [, label=left:\(m+n+2\), label=right:\(\square\)
          [, label=left:\(m\)]
          [, label=left:\(n\)]
        ]
    \end{forest}%
  }
\end{center}

3.2.5)
\begin{center}
    \begin{forest}
      for tree={
        inner sep=2pt,
        circle,
        draw,
        s sep'+=50pt,
        fit=band,
      },
      [, label=left:\(2^{2^{2^{15}}} \times 3^{2^{13}} \times 5^3\), label=right:\(\lor\)
        [, label=left:\(2^{2^{15}}\times 3^9\), label=right:\(\neg\)
          [, label=left:\(2^{15}\), label=right:\(p_1\)]
        ]
        [, label=left:\(2^{13}\), label=right:\(p_0\)]
      ]
    \end{forest}
\end{center}

3.2.6) Left \(f, g\colon (N_1,D_1) \to (N_2, D_2)\) be two isomorphisms. We
prove by induction on \(k\) that for each \(k\), \(f\) and \(g\) agree on all
nodes of height \((n-k)\) in \(N_1\), where \(n\) is the height of \((N_1,
D_1)\).
\begin{proof}
\begin{itemize}
  \item Basis \underline{\(k = 0\)}: The nodes at height \(n-0 = n\) are the leaves. Let \(\ell\) be any leaf. Then \(D_1(\ell) = () = D_2(f~\ell) \land D_1(\ell) = () = D_2(g~\ell)\)
  \item Inductive step \(k > 0 \land k \leqslant n\): Suppose \(f\) and \(g\) agree on all nodes of height \(n-(k-1)\). Let \(\mu\) be a node with height \((n-k)\). Then since \(f\) is an isomorphism:
  \[ D_1(\mu) = (\nu_1,\dots,\nu_m) \Longrightarrow D_2(f~\mu) = (f~\nu_1, \dots, f~\nu_m) \]
  Idem for \(g\):
  \[ D_1(\mu) = (\nu_1,\dots,\nu_m) \Longrightarrow D_2(g~\mu) = (g~\nu_1, \dots, g~\nu_m) \]
  By induction hypothesis \(f~\nu_1 = g~\nu_1,\dots,f~\nu_m = g\nu_m\). Then \(D_2(f~\mu) = D_2(g~\mu)\).

  TODO FINISH
\end{itemize}
\end{proof}

3.3.1)
\[
  \begin{array}{lc}
    \text{Initial segments} & \text{Depth} \\
    (                            & 1 \\
    (\neg                        & 1 \\
    (\neg(                       & 2 \\
    (\neg(p_{22}                 & 2 \\
    (\neg(p_{22}\lra             & 2 \\
    (\neg(p_{22}\lra(            & 3 \\
    (\neg(p_{22}\lra(\neg        & 3 \\
    (\neg(p_{22}\lra(\neg\bot    & 3 \\
    (\neg(p_{22}\lra(\neg\bot)   & 2 \\
    (\neg(p_{22}\lra(\neg\bot))  & 1 \\
    (\neg(p_{22}\lra(\neg\bot))) & 0 \\
  \end{array}
\]

3.3.2) Case 2, \(\overline{\mu} = (\neg \phi)\). \(\mu\) has a single
daughter \(\nu_1\), and \(\overline{\nu} = (\neg\overline{\nu}_1)\). By
inductive hypothesis \(\nu_1\) satisfies (a) and (b). The initial segments of
\(\overline{\mu}\) are:
\begin{enumerate}
  \item (, depth 1.
  \item \((\neg\), depth 1.
  \item \((\neg s\), where \(s\) is a proper initial segment of
  \(\overline{\nu}_1\). Since \(\overline{\nu}_1\) satisfies (a), the depth
  \(d[s]\) is at least 1, so the depth \(d[(\neg s]\) is at least 2.
  \item \((\neg\overline{\nu}_1\) Since \(\overline{\nu}_1\) satisfies (a), the depth is \(1+0=1\).
  \item \((\neg\overline{\nu}_1) = \overline{\mu}\) itself. The depth is \(1+0-1=0\) as required.
\end{enumerate}

3.3.3)
\begin{enumerate}
  \item There is no head (function symbol with depth 1).
  \item
    \begin{forest}
      chtree
      [, label=right:\(\land\)
        [, label=right:\(\lor\)
          [, label=right:\(\lra\)
            [, label=right:\(\neg\)
              [, label=right:\(p_1\)]
            ]
            [, label=right:\(\bot\)]
          ]
          [, label=right:\(p_1\)]
        ]
        [, label=right:\(p_2\)]
      ]
    \end{forest}
  \item Not atomic, nor has a head.
    \begin{forest} baseline,
      for tree={
        inner sep=2pt,
        circle,
        draw,
        s sep'+=170pt,
        l sep'+=50pt,
        fit=band,
      },
      [, label=right:\(\land\)
        [, label=right:\((((\neg(p_0 \lor p_1))\land(p_2 \ra p_3)))\ra(p_3\)]
        [, label=right:\(p_4\)]
      ]
    \end{forest}
  \item Not atomic, nor has a head.
    \begin{forest} baseline,
      for tree={
        inner sep=2pt,
        circle,
        draw,
        s sep'+=40pt,
        fit=band,
      },
      [, label=right:\(\land\)
        [, label=right:\(p_1\)]
        [, label=right:\(\lor\)
          [, label=right:\(\neg\neg(p_2\)]
          [, label=right:\(p_0\)]
        ]
      ]
    \end{forest}
  \item Not atomic, nor has a head.
    \begin{forest} baseline,
      for tree={
        inner sep=2pt,
        circle,
        draw,
        s sep'+=40pt,
        fit=band,
      },
      [, label=right:\(\ra\)
        [, label=right:\((\neg p_1)\)]
        [, label=right:\((\neg p_2)\land p_1)\)
          [, label=right:\(p_2) \land p_1\)]
        ]
      ]
    \end{forest}
  \item
    \begin{forest}
      chtree
      [, label=right:\(\lra\)
        [, label=right:\(\land\)
          [, label=right:\(p_1\)]
          [, label=right:\(\lor\)
            [, label=right:\(p_2\)]
            [, label=right:\(p_3\)]
          ]
        ]
        [, label=right:\(\neg\)
          [, label=right:\(\neg\)
            [, label=right:\(p_0\)]
          ]
        ]
      ]
    \end{forest}
\end{enumerate}
3.3.6) Let \(\pi\) be a parsing tree of height \(n\).
TODO

3.3.7)
\begin{enumerate}
  \item Let \(S = LP(\sigma)\). \(p_{\sqrt{2}}\) is not an atomic formula of \(LP(\sigma)\), nor is it of the forms in (2). Then \(p_{\sqrt{2}} \not\in LP(\sigma)\).
  \item TODO
\end{enumerate}

3.3.8)
\begin{enumerate}
  \item
    \begin{forest}
      chtree,
      for tree={s sep'+=40pt},
      [, label=left:\(CpNNp\), label=right:\(\lra\)
        [, label=left:\(p\), label=right:\(p\)]
        [, label=left:\(NNp\), label=right:\(\neg\)
          [, label=left:\(Np\), label=right:\(\neg\)
            [, label=right:\(p\), label=right:\(p\)]
          ]
        ]
      ]
    \end{forest}
  \item
    \begin{enumerate}[label=(\roman*)]
      \item Corresponds to \(((\neg p) \lra q)\)
        \begin{center}
          \begin{forest}
            chtree,
            for tree={s sep'+=30pt},
            [, label=left:\(ENpq\), label=right:\(\lra\)
              [, label=left:\(Nq\), label=right:\(\neg\)
                [, label=left:\(p\), label=right:\(p\)]
              ]
              [, label=left:\(q\), label=right:\(q\)]
            ]
          \end{forest}
        \end{center}
      \item Corresponds to \((((p \ra p) \ra p) \ra p)\)
        \begin{center}
          \begin{forest}
            chtree,
            [, label=left:\(CCCpppp\), label=right:\(\ra\)
              [, label=left:\(CCppp\), label=right:\(\ra\)
                [, label=left:\(Cpp\), label=right:\(\ra\)
                  [, label=left:\(p\), label=right:\(p\)]
                  [, label=left:\(p\), label=right:\(p\)]
                ]
                [, label=left:\(p\), label=right:\(p\)]
              ]
              [, label=left:\(p\), label=right:\(p\)]
            ]
          \end{forest}
        \end{center}
      \item Corresponds to \(((\neg p) \ra (q \lor (p \land (\neg q))))\)
        \begin{center}
          \begin{forest}
            chtree,
            for tree={s sep'+=10pt},
            [, label=left:\(CNpAqKpNq\), label=right:\(\ra\)
              [, label=left:\(Np\), label=right:\(\neg\)
                [, label=left:\(p\), label=right:\(p\)]
              ]
              [, label=left:\(AqKpNq\), label=right:\(\lor\)
                [, label=left:\(q\), label=right:\(q\)]
                [, label=left:\(KpNq\), label=right:\(\land\)
                  [, label=left:\(p\), label=right:\(p\)]
                  [, label=left:\(Nq\), label=right:\(\neg\)
                    [, label=left:\(q\), label=right:\(q\)]
                  ]
                ]
              ]
            ]
          \end{forest}
        \end{center}
    \end{enumerate}
    \item (i) \(ApKqNp\), (ii) \(CCCpqpp\)
    \item
      \begin{enumerate}[label=(\roman*)]
        \item \(((p \lor q) \lra (\neg ((\neg p) \land (\neg q))))\)
        \item \(((q \ra r) \ra ((p \ra q) \ra (p \ra r)))\)
      \end{enumerate}
\end{enumerate}
3.3.9) TODO

3.3.10)
\begin{enumerate}
  \item
    \[\arraycolsep=1.4pt\def\arraystretch{1.2}
      \begin{array}{rl@{\hspace{1cm}}l}
        f(\phi) & = 1 & \text{when \(\phi\) is atomic}\\
        f((\neg \phi)) & = 1 & f(\phi) \\
        f((\phi \square \psi)) & = 1 + f(\phi) + f(\psi)& \text{with } \square \in \{\land, \lor, \ra, \lra \}
      \end{array}
    \]
    \begin{center}
      \adjustbox{valign=t, minipage=.1\textwidth}{
        \begin{forest}
          chtree
          [, label=left:1, label=right:\(\chi\)]
        \end{forest}%
      }
      \adjustbox{valign=t, minipage=.15\textwidth}{
        \begin{forest}
          chtree
            [, label=left:\(1+m\), label=right:\(\neg\)
              [, label=left:\(m\), label=right:\(\chi\)]
            ]
        \end{forest}%
      }
      \adjustbox{valign=t, minipage=.15\textwidth}{
        \begin{forest}
          chtree
            [, label=left:\(1+m+n\), label=right:\(\square\)
              [, label=left:\(m\), label=right:\(\phi\)]
              [, label=left:\(n\), label=right:\(\psi\)]
            ]
        \end{forest}%
      }
    \end{center}
  \item
    \[\arraycolsep=1.4pt\def\arraystretch{1.2}
      \begin{array}{rl@{\hspace{1cm}}l}
        Sub(\phi)                & = \{\phi\} & \text{when \(\phi\) is atomic} \\
        Sub((\neg \phi))         & = \{(\neg \phi)\} \cup Sub(\phi) & \\
        Sub((\phi \square \psi)) & = \{(\phi \square \psi)\} \cup Sub(\phi) \cup Sub(\psi) & \text{with } \square \in \{\land, \lor, \ra, \lra \}
      \end{array}
    \]
    \begin{center}
      \adjustbox{valign=t, minipage=.14\textwidth}{
        \begin{forest}
          chtree
          [, label=left:\(\{\chi\}\), label=right:\(\chi\)]
        \end{forest}%
      }
      \adjustbox{valign=t, minipage=.2\textwidth}{
        \begin{forest}
          chtree
            [, label=left:\(\{\chi\} \cup S\), label=right:\(\neg\)
              [, label=left:\(S\), label=right:\(\chi\)]
            ]
        \end{forest}%
      }
      \adjustbox{valign=t, minipage=.2\textwidth}{
        \begin{forest}
          chtree,
          for tree={
            s sep'+=20pt,
          },
          [, label=left:\(\{\phi \square \psi\} \cup S_1 \cup S_2\), label=right:\(\square\)
            [, label=left:\(S_1\), label=right:\(\phi\)]
            [, label=left:\(S_2\), label=right:\(\psi\)]
          ]
        \end{forest}%
      }
    \end{center}
\end{enumerate}

3.4.1)
\begin{itemize}
  \item[(d)]
  \begin{itemize}
    \item[(iii)] \(\nu\) has right-hand label \((\land \text{E})\), and there
    are formulas \(\phi\) and \(\psi\) such that \(\nu\) has left label
    either \(\phi\) or \(\psi\), and its daughter has left label \((\phi
    \land \psi)\).
    \item[(iv)] \(\nu\) has right-hand label \((\lor \text{I})\), and there are
    formulas \(\phi\) and \(\psi\) such that \(\nu\) has left label \((\phi
    \lor \psi)\), and the left labels on the daughter of \(\nu\) is either
    \(\phi\) or \(\psi\).
    \item[(v)] \(\nu\) has right-hand label \((\leftrightarrow \text{E})\), and
    there are formulas \(\phi\) and \(\psi\) such that \(\nu\) has the left
    label \((\phi \to \psi)\) or \((\psi \to \phi)\), and the left label on the
    daughter of \(\nu\) is \((\phi \leftrightarrow \psi)\).
  \end{itemize}
  \item[(e)] TODO
  \begin{itemize}
    \item[(ii)]
    \item[(iii)]
    \item[(iv)]
  \end{itemize}
\end{itemize}

3.4.2)
\begin{itemize}
  \item Dandahs are missing the number reference. Both \((\to \text{I})\) are
  wrongly applied. The second \((\to \text{E})\) should be a \((\neg
  \text{E})\). The (RAA) is also wrongly applied, as you don't deduce \(\bot\).
  \item TODO
\end{itemize}
3.4.3)
\begin{enumerate}
  \item Let \(\pi\) be a parsing tree for \(LP(\rho)\). Then by definition,
  every leaf is labelled with either \(\bot\) or a symbol from \(\rho\). But
  then since \(\rho \subseteq \sigma\), we have that every leaf is labelled
  with \(\bot\) or a symbol from \(\sigma\).
  Thus, \(\pi\) is also a parsing tree for \(LP(\sigma)\).
  \item Let \(\phi\) be a formula of \(LP(\rho)\). We need to prove that
  \(\phi\) is a formula of \(LP(\sigma)\). We prove that if a formula is in
  \(LP(\rho)\), then it is in \(LP(\sigma)\) by induction on the structure of
  \(\phi\) :
  \begin{itemize}
    \item Case \(\bot\): By definition \(\bot \in LP(\sigma)\).
    \item Case \(a\) (symbol from \(LP(\rho)\)): Since \(\rho \subseteq
    \sigma\), we have that \(a \in \sigma\), and by definition \(a \in
    LP(\sigma)\).
    \item Case \((\neg \psi)\): Since \(\psi\) is a formula of \(LP(\rho)\),
    is also a formula of \(LP(\sigma)\) by inductive hypothesis. But then
    \(\neg \psi = \phi\) is a formula of \(LP(\sigma)\) by definition.
    \item Case \((\psi_1 \square \psi_2)\), where \(\square \in \{\land, \lor, \ra, \lra\}\): Since \(\psi_1\) and \(\psi_2\) are formulas of \(LP(\rho)\),
    they are also formulas of \(LP(\sigma)\) by inductive hypothesis. Then
    \((\psi_1 \square \psi_2) = \phi\) is a formula of \(LP(\sigma)\) by
    definition.
  \end{itemize}
  \item TODO
  \item Suppose that \((\Gamma \vdash_{\rho} \psi)\) is correct. By
  definition 3.4.4. this means that there is a \(\rho\)-derivation \(D\)
  whose conclusion is \(\psi\) and whose undischarged assumptions are all in
  the set \(\Gamma\). But as proved in (c) of this exercise, every
  \(\rho\)-derivation is also a \(\sigma\)-derivation. Thus, \(D\) is a
  \(\sigma\)-derivation, and again by definition 3.4.4 \((\Gamma
  \vdash_{\sigma} \psi)\) is correct too.
\end{enumerate}

3.4.4)
\begin{enumerate}
  \item TODO (in the solutions)
  \item TODO
\end{enumerate}

3.4.5)
\begin{enumerate}
  \item TODO
  \item TODO
\end{enumerate}

3.4.6)
\begin{enumerate}
  \item TODO
  \item TODO
  \item TODO
  \item TODO
\end{enumerate}

3.4.7)
\begin{enumerate}
  \item TODO
  \item TODO
\end{enumerate}

3.4.8) TODO

3.5.1)
\begin{enumerate}
  \item
\[
  \begin{array}{crcl}
    p & (p & \lra & p) \\
    \toprule
    \true & \true & \true & \true \\
    \false & \false & \true & \false \\
  \end{array}
\]
  \item
\[
  \begin{array}{ccc}
    p & q & (p \ra (q \ra p)) \\
    \toprule
    \true & \true & t\\
    \false & \false & t\\
  \end{array}
\]
\item TODO
\item TODO
\item TODO
\item TODO
\item TODO
\end{enumerate}
3.5.2)
\begin{enumerate}
  \item
\[
  \begin{array}{crclcrl}
    p_0 & ((p_0 & \ra & \bot) & \lra & (\neg & p_0)) \\
    \toprule
    \true & \true & \false & \false & \true & \false & \true \\
    \false & \false & \true & \false & \true & \true & \false \\
  \end{array}
\]
Thus, it is satisfiable, and a tautology.
  \item TODO
  \item TODO
  \item TODO
  \item TODO
  \item TODO
  \item TODO
  \item TODO
\end{enumerate}

3.5.3)
\[
  \begin{array}{ccc}
    p & q & \phi \\
    \toprule
    \true & \true & \true \\
    \true & \false & \false \\
    \false & \true & \false \\
    \false & \false & \true \\
  \end{array}
\]
Thus, \(\phi = (p \lra q)\).

3.5.4) TODO

3.5.5) Let \(|\sigma| = k\).
\begin{enumerate}
  \item By definition 3.5.3, a \(\sigma\)-structure is a function \(A\colon
  \sigma \to \{T, F\}\). Then the number of \(\sigma\)-structures is \(2^k\).
  \item \(2^k(k+n)\). Note that this result holds only for the ``verbose''
  truth tables, e.g.\ given \(p \land q\), then \(k = 2\), \(n = 3\) and the
  table is:
    \[
      \begin{array}{ccrcl}
        p & q & (p & \land & q) \\
        \toprule
        \true  & \true  & \true  & \true  & \true \\
        \true  & \false & \true  & \false & \false \\
        \false & \true  & \false & \false & \true \\
        \false & \false & \false & \true  & \false
      \end{array}
    \]
  \item TODO
\end{enumerate}

3.6.1)
Double Negation Law (can all be seen in the same table):
    \[
      \begin{array}{ccc}
        p_1 & \neg p_1 & \neg (\neg p_1) \\
        \toprule
        \true  & \false & \true \\
        \false & \true  & \false
      \end{array}
    \]
Idempotence Laws
    \[
      \begin{array}{ccc}
        p_1 & (p_1 \lor p_1) & (p_1 \land p_1) \\
        \toprule
        \true  & \true  & \true \\
        \false & \false & \false
      \end{array}
    \]
TODO finish

3.6.2) TODO prove by using truth tables (see solutions)

3.6.3) TODO (judging by previous ex., prove by using truth tables)

3.6.4) TODO

3.6.5) (i) \(\Rightarrow\) (ii): Suppose \(\phi\) is a tautology. By definition 3.5.8 (a) every
\(\sigma\)-structure is a model of \(\phi\), i.e. \(A^*(\phi) = \true\) for
all \(\sigma\)-structures A. Then, by the semantics (in particular Definition
3.5.6 (c)) we have \(A^*(\neg \phi) = \false\), so \(\neg \phi\) is a
contradiction.

3.7.1)
\begin{multicols}{2}
\begin{enumerate}
  \item \(p \to p\)
  \item \(q \to q\)
  \item \(q \to p\)
  \item \((r \land ((t \to (\neg p)) \lor q) \lor (\neg(\neg q)))\)
\end{enumerate}
\end{multicols}

3.7.2)
\begin{enumerate}
  \item Start with the De Morgan law:
\[
  \underbrace{(\neg(p_1 \land p_2))}_{\phi_1}  \quad \text{eq} \quad \underbrace{((\neg p_1) \lor (\neg p_2))}_{\phi_2}
\]
Let \(S = p_1 \land p_2/p_1, p_3/p_2\). Then by the substitution theorem, \(\phi_1[S] \; \text{~eq~} \; \phi_2[S]\), i.e.:
\[
  (\neg((p_1 \land p_2) \lor p_3)) \quad \text{eq} \quad ((\neg (p_1 \land p_2)) \lor (\neg p_3))
\]
Take the De Morgan law \((\neg (p_1 \land p_2)) \; eq \; ((\neg p_1) \lor (\neg p_2))\) by De Morgan. Take a propositional variable that does not occur in the formula, say \(r\). Then
\[
\begin{array}{lcr}
  ((\neg (p_1 \land p_2)) \lor (\neg p_3)) & \text{is} & (r \lor (\neg p_3))[(\neg(p_1 \land p_2))/r] \\
  (((\neg p_1) \lor (\neg p_2)) \lor (\neg p_3)) & \text{is} & (r \lor (\neg p_3))[((\neg p_1) \lor (\neg p_2))/r]
\end{array}
\]
Then by the replacement theorem:
\[
  ((\neg (p_1 \land p_2)) \lor (\neg p_3)) \quad \text{eq} \quad (((\neg p_1) \lor (\neg p_2)) \lor (\neg p_3))
\]

\item \begin{proof}
The basis for \(n = 2\) is De Morgan's law, which may simply be proved by
truth tables or some other way. For the inductive step, start with De
Morgan's law \(\neg (p_1 \land p_2) \mathbin{\text{eq}} (\neg p_1) \lor (\neg
p_2)\). By the substitution theorem with \(S = ((\cdots(\phi_1 \land \phi_2)\cdots)\land \phi_n) / p_1, \phi_{n+1} / p_2\):
\[
  \neg (((\cdots(\phi_1 \land \phi_2)\cdots)\land \phi_n) \land \phi_{n+1}) \quad \text{eq} \quad \neg ((\cdots(\phi_1 \land \phi_2)\cdots)\land \phi_n) \lor (\neg \phi_{n+1})
\]
We know by inductive hypothesis that:
\[
  \neg ((\cdots(\phi_1 \land \phi_2)\cdots)\land \phi_n) \quad \text{eq} \quad (\cdots((\neg \phi_1) \lor (\neg \phi_2))\cdots) \lor (\neg \phi_n)
\]
Let \(r\) be any propositional variable not occurring in any of the formulas. Then:
\[
\begin{array}{lcr}
  \neg ((\cdots(\phi_1 \land \phi_2)\cdots)\land \phi_n)) \lor (\neg \phi_{n+1}) & \text{is} & (r \lor (\neg \phi_{n+1}))[\neg ((\cdots(\phi_1 \land \phi_2)\cdots)\land \phi_n)/r]\\
  ((\cdots((\neg \phi_1) \lor (\neg \phi_2))\cdots) \lor (\neg \phi_n)) \lor (\neg \phi_{n+1}) & \text{is} & (r \lor (\neg \phi_{n+1}))[((\cdots((\neg \phi_1) \lor (\neg \phi_2))\cdots) \lor (\neg \phi_n))/r]
\end{array}
\]
Thus, by the replacement theorem
\[
  \neg ((\cdots(\phi_1 \land \phi_2)\cdots)\land \phi_{n+1}) \quad \text{eq} \quad (\cdots((\neg \phi_1) \lor (\neg \phi_2))\cdots) \lor (\neg \phi_{n+1})
\]
\end{proof}
\item TODO
\end{enumerate}
3.7.3)
\begin{enumerate}
  \item TODO
  \item TODO
\end{enumerate}

3.7.4)
\begin{enumerate}
  \item TODO
  \item TODO
\end{enumerate}

3.7.5)
\begin{enumerate}
  \item TODO
  \item TODO
\end{enumerate}

3.7.6) TODO

3.8.1)
\begin{enumerate}
  \item
  \[
    \begin{array}{ccl@{\hspace{1cm}}R}
    \neg (p_1 \to p_2) \lor \neg (p_2 \to p_1) & \equiv & \neg (\neg p_1 \lor p_2) \lor \neg (\neg p_2 \lor p_1)             & \\
                                               & \equiv & (\neg (\neg p_1) \land \neg p_2) \lor (\neg (\neg p_2) \land \neg p_1) & De Morgan \\
                                               & \equiv & (p_1 \land \neg p_2) \lor (p_2 \land \neg p_1)                         & Double Negation
    \end{array}
  \]
  This formula is in DNF. For an equivalent CNF, we continue:
  \[
    \begin{array}{ccl@{\hspace{1cm}}R}
    (p_1 \land \neg p_2) \lor (p_2 \land \neg p_1) & \equiv & ((p_1 \land \neg p_2) \lor p_2) \land ((p_1 \land \neg p_2) \lor \neg p_1) & Distributivity \\
    & \equiv & (p_1 \lor p_2) \land (\neg p_2 \lor p_2) \land (p_1 \lor \neg p_1) \land (\neg p_2 \lor \neg p_1) & Distributivity \\
    \end{array}
  \]
  \item TODO
  \item TODO
  \item TODO
  \item TODO
  \item TODO
  \item TODO
  \item TODO
  \item TODO
  \item TODO
\end{enumerate}

%5.


\newpage
\bibliography{refs}
\bibliographystyle{unsrt}

\end{document}
\documentclass{article}

\usepackage[utf8]{inputenc}
\usepackage{caption}
\usepackage{float}

\usepackage{tikz}
\usetikzlibrary{automata, positioning, arrows,
fit, % for the dashed boxes on Thompson's construction
calc
}

\newcommand{\emptystr}{\varepsilon}

\usepackage{hyperref}
\hypersetup{pdfauthor={Cristian Adrián Ontivero}}
\usepackage{ebproof}

% Set items in enumerate environment to letters.
\usepackage{enumitem}
\setenumerate[0]{label=(\alph*)}

\graphicspath{{imgs/}}

\newlength\tindent%
\setlength{\tindent}{\parindent}
\setlength{\parindent}{0pt}
\renewcommand{\indent}{\hspace*{\tindent}}

\renewcommand*{\tableautorefname}{Tabla}
\renewcommand*{\figureautorefname}{Figura}

%These tell TeX which packages to use.
\usepackage{array,epsfig}
\usepackage{amsmath}
\usepackage{amsfonts}
\usepackage{amssymb}
\usepackage{amsxtra}
\usepackage{amsthm}
\usepackage{mathrsfs}

%Here I define some theorem styles and shortcut commands for symbols I use often
\theoremstyle{definition}
\newtheorem{defn}{Definición}
\newtheorem{thm}{Teorema}
\newtheorem{cor}{Corolario}
\newtheorem*{rmk}{Remark}
\newtheorem{lem}{Lema}
\newtheorem*{joke}{Joke}

\newtheorem{ex}{Ejemplo}
\newcommand{\exautorefname}{Ejemplo}

\newtheorem{exercise}{Ejercicio}
\newcommand{\exerciseautorefname}{Ejercicio}

\newtheorem{soln}{Solución}
\newtheorem{prop}{Proposición}

\newcommand{\ra}{\rightarrow}
\newcommand{\surj}{\twoheadrightarrow}
\newcommand{\graph}{\mathrm{graph}}
\newcommand{\bb}[1]{\mathbb{#1}}
\newcommand{\Z}{\bb{Z}}
\newcommand{\Q}{\bb{Q}}
\newcommand{\R}{\bb{R}}
\newcommand{\C}{\bb{C}}
\newcommand{\N}{\bb{N}}
\newcommand{\M}{\mathbf{M}}
\newcommand{\m}{\mathbf{m}}
\newcommand{\MM}{\mathscr{M}}
\newcommand{\HH}{\mathscr{H}}
\newcommand{\Om}{\Omega}
\newcommand{\Ho}{\in\HH(\Om)}
\newcommand{\bd}{\partial}
\newcommand{\del}{\partial}
\newcommand{\bardel}{\overline\partial}
\newcommand{\textdf}[1]{\textbf{\textsf{#1}}\index{#1}}
\newcommand{\img}{\mathrm{img}}
\newcommand{\ip}[2]{\left\langle{#1},{#2}\right\rangle}
\newcommand{\inter}[1]{\mathrm{int}{#1}}
\newcommand{\exter}[1]{\mathrm{ext}{#1}}
\newcommand{\cl}[1]{\mathrm{cl}{#1}}
\newcommand{\ds}{\displaystyle}
\newcommand{\vol}{\mathrm{vol}}
\newcommand{\cnt}{\mathrm{ct}}
\newcommand{\osc}{\mathrm{osc}}
\newcommand{\LL}{\mathbf{L}}
\newcommand{\UU}{\mathbf{U}}
\newcommand{\support}{\mathrm{support}}
\newcommand{\AND}{\;\wedge\;}
\newcommand{\OR}{\;\vee\;}
\newcommand{\Oset}{\varnothing}
\newcommand{\st}{\ni}
\newcommand{\wh}{\widehat}

\newcommand{\lra}{\leftrightarrow}
% Logic rules
\newcommand{\ii}{\footnotesize ($\rightarrow$I)}
\newcommand{\ie}{\footnotesize ($\rightarrow$E)}
\newcommand{\ci}{\footnotesize ($\wedge$I)}
\newcommand{\ce}{\footnotesize ($\wedge$E)}
\newcommand{\di}{\footnotesize ($\vee$I)}
\newcommand{\de}{\footnotesize ($\vee$E)}
\newcommand{\bi}{\footnotesize ($\leftrightarrow$I)}
\newcommand{\be}{\footnotesize ($\leftrightarrow$E)}
\newcommand{\negi}{\footnotesize ($\neg$I)}
\newcommand{\nege}{\footnotesize ($\neg$E)}
\newcommand{\raa}{\footnotesize (RAA)}
\newcommand*\cir[1]{\tikz[baseline=(char.base)]{
            \node[shape=circle,draw,inner sep=1pt] (char) {\footnotesize #1};}}

            \newcommand{\danda}[2]{%
    \tikz[baseline=(tocancel.base)]{%
        \node[inner sep=0pt,outer sep=0pt] (tocancel) {\(#1\)};
        \node[inner sep=0pt,outer sep=1pt,above right=0mm of tocancel] (end)
        {\cir{#2}};
        \draw (tocancel.south west) -- (end.260);
    }\hspace{-1.2em}%
}%
%Pagination stuff.
\setlength{\topmargin}{-.3 in}
\setlength{\oddsidemargin}{0in}
\setlength{\evensidemargin}{0in}
\setlength{\textheight}{9.in}
\setlength{\textwidth}{6.5in}
\pagestyle{empty}

\begin{document}

2.2.2)
\begin{enumerate}
  \item 
  \[
    \begin{prooftree}
      \hypo{\phi}
      \hypo{\psi}
      \hypo{\chi}
      \infer2[\ci]{(\psi\land\chi)}
      \infer2[\ci]{(\phi \land (\psi\land\chi))}
    \end{prooftree}
  \]
  \item 
  \[
    \begin{prooftree}
      \hypo{\chi}
      \hypo{\phi}
      \infer2[\ci]{(\chi\land\phi)}
    \end{prooftree}
  \]

  \item 
  \[
    \begin{prooftree}
      \hypo{\phi}
      \hypo{\phi}
      \infer2[\ci]{(\phi\land\phi)}
      \hypo{\phi}
      \infer2[\ci]{((\phi\land\phi)\land \phi)}
    \end{prooftree}
  \]
  \item 
  \[
    \begin{prooftree}
      \hypo{\phi}
      \hypo{\psi}
      \infer2[\ci]{(\phi\land\psi)}
      \hypo{\phi}
      \hypo{\psi}
      \infer2[\ci]{(\phi\land\psi)}
      \infer2[\ci]{((\phi\land\psi)\land(\phi\land\psi))}
    \end{prooftree}
  \]

\end{enumerate}

2.3.1)
\begin{enumerate}
  \item
  \[
    \begin{prooftree}
      \hypo{(\phi\land\psi)}
      \infer1[\ce]{\phi}
      \hypo{(\phi\land\psi)}
      \infer1[\ce]{\phi}
      \infer2[\ci]{(\phi\land\phi)}
    \end{prooftree}
  \]

  \item \(\land\) associativity in one direction:
  \[
    \begin{prooftree}
      \hypo{((\phi\land\psi)\land\chi)}
      \infer1[\ce]{(\phi\land\psi)}
      \infer1[\ce]{\phi}
      \hypo{((\phi\land\psi)\land\chi)}
      \infer1[\ce]{(\phi\land\psi)}
      \infer1[\ce]{\psi}
      \hypo{((\phi\land\psi)\land\chi)}
      \infer1[\ce]{\chi}
      \infer2[\ci]{(\psi\land\chi)}
      \infer2[\ci]{(\phi\land(\psi\land\chi))}
    \end{prooftree}
  \]
  \item 
  \[
    \begin{prooftree}
      \hypo{(\psi\land\chi)}
      \infer1[\ce]{\chi}
      \hypo{\phi}
      \infer2[\ci]{(\chi\land\phi)}
    \end{prooftree}
  \]
  \item 
  \[
    \begin{prooftree}
      \hypo{(\phi\land(\psi\land\chi))}
      \infer1[\ce]{(\psi\land\chi)}
      \infer1[\ce]{\chi}
      \hypo{(\phi\land(\psi\land\chi))}
      \infer1[\ce]{\phi}
      \infer2[\ci]{(\chi\land\phi)}
      \hypo{(\phi\land(\psi\land\chi))}
      \infer1[\ce]{(\psi\land\chi)}
      \infer1[\ce]{\psi}
      \infer2[\ci]{((\chi\land\phi)\land\psi)}
    \end{prooftree}
  \]
\end{enumerate}

2.3.2)
  \[
    \begin{prooftree}
      \hypo{D}
      \infer[rule style=no rule]1{\phi}
      \hypo{D'}
      \infer[rule style=no rule]1{\psi}
      \infer2[\ci]{(\phi\land\psi)}
      \infer1[\ce]{\phi}
    \end{prooftree}
  \]
  Just do $D$ to derive $\phi$.

2.3.3) Suppose \(\{\phi_1, \phi_2\} \vdash \psi\). Then:
  \[
    \begin{prooftree}
      \hypo{(\phi_1\land\phi_2)}
      \infer1[\ce]{\phi_1}
      \hypo{(\phi_1\land\phi_2)}
      \infer1[\ce]{\phi_2}
      \infer2[(assumption)]{\psi}
    \end{prooftree}
  \]
Suppose \(\{\phi_1\land\phi_2\} \vdash \psi\). Then:
  \[
    \begin{prooftree}
      \hypo{\phi_1}
      \hypo{\phi_2}
      \infer2[\ci]{(\phi_1\land\phi_2)}
      \infer1[(assumption)]{\psi}
    \end{prooftree}
  \]

2.4.1) 
\begin{enumerate}
  \item \(f\) is differentiable \(\ra\) \(f\) is continuous.
  \item \(x\) is positive \(\ra\) \(x\) has a square root.
  \item \(b \neq 0 \ra \dfrac{ab}{b} = a\)
\end{enumerate}

2.4.2) Proof of \(\vdash ((\phi\land\psi)\ra(\psi\land\phi))\)
\begin{enumerate}
  \item 
  \[
    \begin{prooftree}
      \hypo{\danda{(\phi\land\psi)}{1}}
      \infer1[\ce]{\psi}
      \hypo{\danda{(\phi\land\psi)}{1}}
      \infer1[\ce]{\phi}
      \infer2[\ci]{(\psi\land\phi)}
      \infer[left label=\cir{1}]1[\ii]{((\phi\land\psi)\ra(\psi\land\phi))}
    \end{prooftree}
  \]
  \item Proof of \(\vdash ((\psi\ra\chi)\ra((\phi\ra\psi)\ra(\phi\ra\chi)))\)
  \[
    \begin{prooftree}
      \hypo{\danda{\phi}{1}}
      \hypo{\danda{(\phi\ra\psi)}{2}}
      \infer2[\ie]{\psi}
      \hypo{\danda{(\psi\ra\chi)}{3}}
      \infer2[\ie]{\chi}
      \infer[left label=\cir{1}]1[\ii]{(\phi\ra\chi)}
      \infer[left label=\cir{2}]1[\ii]{((\phi\ra\psi)\ra(\phi\ra\chi))}
      \infer[left label=\cir{3}]1[\ii]{((\psi\ra\chi)\ra((\phi\ra\psi)\ra(\phi\ra\chi)))}
    \end{prooftree}
  \]
\end{enumerate}
2.4.3)
\begin{enumerate}
  \item \(\vdash (\phi \ra (\psi \ra \phi))\)
  \item \(\{\phi\}\vdash (\phi \ra (\psi \ra \phi))\)
  \item \(\{(\phi\land\psi)\} \vdash (\psi \ra (\psi \land \phi))\)
  \item \(\vdash \phi \ra \phi\)
\end{enumerate}

2.4.4)
\begin{enumerate}
  \item 
  \[
    \begin{prooftree}
      \hypo{\danda{\psi}{1}}
      \infer[left label=\cir{1}]1[\ii]{(\psi\ra\psi)}
      \infer1[\ii]{(\phi\ra(\psi\ra\psi))}
    \end{prooftree}
  \]
  \item 
  \[
    \begin{prooftree}
      \hypo{\danda{\phi}{1}}
      \infer[left label=\cir{1}]1[\ii]{(\phi\ra\phi)}
      \hypo{\danda{\psi}{2}}
      \infer[left label=\cir{2}]1[\ii]{(\psi\ra\psi)}
      \infer2[\ci]{((\phi\ra\phi)\land(\psi\ra\psi))}
    \end{prooftree}
  \]
  \item 
  \[
    \begin{prooftree}
      \hypo{\danda{\phi}{1}}
      \hypo{\danda{(\phi\ra(\theta\ra\psi))}{3}}
      \infer2[\ie]{(\theta\ra\psi)}
      \hypo{\danda{\theta}{1}}
      \infer2[\ie]{\psi}
      \infer[left label=\cir{1}]1[\ii]{(\phi\ra\psi)}
      \infer[left label=\cir{2}]1[\ii]{(\theta\ra(\phi\ra\psi))}
      \infer[left label=\cir{3}]1[\ii]{((\phi\ra(\theta\ra\psi))\ra(\theta\ra(\phi\ra\psi)))}
    \end{prooftree}
  \]
  \item 
  \[
    \begin{prooftree}
      \hypo{\danda{\phi}{1}}
      \hypo{(\phi\ra\psi)}
      \infer2[\ie]{\psi}
      \hypo{\danda{\phi}{1}}
      \hypo{(\phi\ra\chi)}
      \infer2[\ie]{\chi}
      \infer2[\ci]{(\psi\land\chi)}
      \infer[left label=\cir{1}]1[\ii]{(\phi\ra(\psi\land\chi))}
    \end{prooftree}
  \]
  \item TODO
  \item 
  \[
    \begin{prooftree}
      \hypo{\danda{(\phi\land\psi)}{1}}
      \infer1[\ce]{\psi}
      \hypo{\danda{(\phi\land\psi)}{1}}
      \infer1[\ce]{\phi}
      \hypo{(\phi\ra(\psi\ra\chi))}
      \infer2[\ie]{(\psi\ra\chi)}
      \infer2[\ie]{\chi}
      \infer[left label=\cir{1}]1[\ii]{((\phi\land\psi)\ra\chi)}
    \end{prooftree}
  \]
  \item 
  \[
    \begin{prooftree}
      \hypo{\danda{\phi}{1}}
      \hypo{\danda{(\phi\ra\psi)}{3}}
      \infer2[\ie]{\psi}
      \hypo{\danda{(\psi\ra\theta)}{2}}
      \infer2[\ie]{\theta}
      \infer[left label=\cir{1}]1[\ii]{(\phi\ra\chi)}
      \infer[left label=\cir{2}]1[\ii]{((\psi\ra\theta)\ra(\phi\ra\chi))}
      \infer[left label=\cir{3}]1[\ii]{((\phi\ra\psi)\ra((\psi\ra\theta)\ra(\phi\ra\chi)))}
    \end{prooftree}
  \]
  \item 
  \[
    \begin{prooftree}
      \hypo{\danda{\phi}{1}}
      \hypo{\danda{(\phi\ra(\psi\land\theta))}{3}}
      \infer2[\ie]{(\psi\land\theta)}
      \infer1[\ce]{\theta}
      \infer[left label=\cir{1}]1[\ii]{(\phi\ra\theta)}
      \hypo{\danda{\phi}{2}}
      \hypo{\danda{(\phi\ra(\psi\ra\theta))}{3}}
      \infer2[\ie]{(\psi\land\theta)}
      \infer1[\ce]{\psi}
      \infer[left label=\cir{2}]1[\ii]{(\phi\ra\psi)}
      \infer2[\ci]{(\psi\ra\theta)\land(\psi\ra\psi)}
      \infer[left label=\cir{3}]1[\ii]{((\phi\ra(\psi\land\theta))\ra((\psi\ra\theta)\land(\psi\ra\psi)))}
    \end{prooftree}
  \]
\end{enumerate}

2.4.5) \((\Rightarrow)\) Suppose \(\{\phi\}\vdash\psi\). Then:
  \[
    \begin{prooftree}
      \hypo{\danda{\phi}{1}}
      \infer1[(assumption)]{\psi}
      \infer[left label=\cir{1}]1[\ii]{(\phi\ra\psi)}
    \end{prooftree}
  \]
\((\Rightarrow)\) Suppose \(\vdash(\phi\ra\psi)\). Then:
  \[
    \begin{prooftree}
      \hypo{\phi}
      \hypo{(\phi\ra\psi)}
      \infer2[\ie]{\psi}
    \end{prooftree}
  \]
  
2.4.6) Suppose \(D_1\) is a derivation whose undischarged assumptions are all in \(\Gamma \cup \{\phi\}\). Take \(D*\) to be a copy of \(D_1\) where each ocurrence of \(\phi\) is replaced by \(\danda{\phi}{1}\)~. Then take the derivation \(D'_1\) to be:
  \[
    \begin{prooftree}
      \hypo{D*}
      \infer[rule style=no rule]1{\psi}
      \infer[left label=\cir{1}]1[\ii]{(\phi\ra\psi)}
    \end{prooftree}
  \]
\((\Rightarrow)\) Now suppose \(D_2\) is a derivation whose undischarged
assumptions are all in \(\Gamma\). Take \(D'_2\) to be:
  \[
    \begin{prooftree}
      \hypo{\phi}
      \hypo{D_1}
      \infer[rule style=no rule]1{(\phi\ra\psi)}
      \infer2[\ii]{\psi}
    \end{prooftree}
  \]

2.5.1) 
\begin{enumerate}
  \item 
  \[
    \begin{prooftree}
      \hypo{\phi}
      \hypo{(\phi\lra\psi)}
      \infer1[\be]{(\phi\ra\psi)}
      \infer2[\ii]{\psi}
    \end{prooftree}
  \]
  \item 
  \[
    \begin{prooftree}
      \hypo{\danda{\phi}{1}}
      \infer[left label=\cir{1}]1[\ii]{(\phi\ra\phi)}
      \hypo{\danda{\phi}{2}}
      \infer[left label=\cir{2}]1[\ii]{(\phi\ra\phi)}
      \infer2[\bi]{(\phi\ra\phi)}
    \end{prooftree}
  \]
  \item 
  \[
    \begin{prooftree}
      \hypo{\danda{\phi}{1}}
      \hypo{(\phi\lra\psi)}
      \infer1[\be]{(\phi\ra\psi)}
      \infer2[\ie]{\psi}
      \hypo{(\psi\lra\chi)}
      \infer1[\be]{(\psi\ra\chi)}
      \infer2[\ie]{\chi}
      \infer[left label=\cir{1}]1[\ii]{(\phi\ra\chi)}

      \hypo{\danda{\chi}{2}}
      \hypo{(\psi\lra\chi)}
      \infer1[\be]{(\chi\ra\psi)}
      \infer2[\ie]{\psi}
      \hypo{(\phi\lra\psi)}
      \infer1[\be]{(\psi\ra\phi)}
      \infer2[\ie]{\phi}
      \infer[left label=\cir{2}]1[\ii]{(\chi\ra\phi)}
      \infer2[\bi]{(\phi\lra\chi)}
    \end{prooftree}
  \]
  \item 
  \[
    \begin{prooftree}
      \hypo{\danda{\psi}{1}}
      \infer1[\ii]{(\phi\ra\psi)}
      \hypo{\danda{\phi}{3}}
      \infer1[\ii]{(\psi\ra\phi)}
      \infer2[\bi]{(\phi\lra\psi)}

      \hypo{((\phi\lra\psi)\lra\chi)}
      \infer1[\be]{((\phi\lra\psi)\ra\chi)}
      \infer2[\ie]{\chi}
      \infer[left label=\cir{1}]1[\ii]{(\psi\ra\chi)}

      \hypo{\danda{\phi}{3}}
      \hypo{\danda{\chi}{2}}
      \hypo{((\phi\lra\psi)\lra\chi)}
      \infer1[\be]{(\chi\ra(\phi\lra\psi))}
      \infer2[\ie]{(\phi\lra\psi)}
      \infer1[\be]{(\phi\ra\psi)}
      \infer2[\ie]{\psi}
      \infer[left label=\cir{2}]1[\be]{(\chi\ra\psi)}
      \infer2[\ie]{(\psi\lra\chi)}
      \infer[left label=\cir{3}]1[\be]{(\phi\ra(\psi\lra\chi))}

    \end{prooftree}
  \]
  \item 
  \[
    \begin{prooftree}
      \hypo{\danda{\psi}{1}}
      \infer[left label=\cir{1}]1[\ii]{(\psi\ra\psi)}
      \hypo{\danda{\psi}{2}}
      \infer[left label=\cir{2}]1[\ii]{(\psi\ra\psi)}
      \infer2[\bi]{(\psi\lra\psi)}

      \hypo{(\phi\lra(\psi\lra\psi))}
      \infer1[\be]{((\psi\lra\psi)\lra\phi)}
      \infer2[\ie]{\phi}
    \end{prooftree}
  \]
\end{enumerate}

2.5.2) We know that \(\forall \phi,\psi \in S, \phi \sim \psi \text{ iff }
\vdash (\phi\lra\phi)\). We have to show that \(\sim\) is an equivalence
relation.
\begin{itemize}
  \item Reflexive: Let \(\phi\) be a formula. By 2.5.1(b), \(\vdash(\phi\lra\phi)\), then \(\phi\sim\phi\).
  \item Symmetric: Let \(\phi,\psi\) be formulas such that \(\phi\sim\psi\).
  Then \(\vdash(\phi\lra\psi\). By example 2.5.1,
  \(\{(\phi\lra\psi)\}\vdash(\psi\lra\phi)\), so \(\psi\sim\phi\).
  In more detail:
  TODO (add derivation as in the transitive case)

  \item Transitive: Let \(\phi,\psi,\chi\) be formulas such that
  \(\phi\sim\psi\) and \(\psi\sim\chi\). Then \(\vdash(\phi\lra\psi)\) and
  \(\vdash(\psi\lra\chi)\). Then by exercise 2.5.1 (c),
  \(\{(\phi\lra\psi),(\psi\lra\chi)\}\vdash (\phi\lra\chi)\), so
  \((\phi\sim\chi)\). In more detail, we have:
  \[
    \begin{prooftree}
      \hypo{D}
      \infer[rule style=no rule]1{(\phi\lra\psi)}
    \end{prooftree}\hspace{6mm}
    \begin{prooftree}
      \hypo{D'}
      \infer[rule style=no rule]1{(\psi\lra\chi)}
    \end{prooftree}\hspace{6mm}
    \begin{prooftree}
      \hypo{D'}
      \infer[rule style=no rule]1{(\psi\lra\chi)}
    \end{prooftree}
  \]
  \[
    \begin{prooftree}
      \hypo{D_c}
      \infer[rule style=no rule]1{(\phi\lra\chi)}
    \end{prooftree}
  \]
  where \(D_c\) is all the derivation we did in 2.5.1(c).
  
\end{itemize}

2.5.3) Suppose we have a derivation \(D\) with no undischarged assumptions. Let \(\phi\) by any statement. Then:
\[
    \begin{prooftree}
      \hypo{D}
      \infer[rule style=no rule]1{\psi}
      \hypo{\danda{(\phi\lra\psi)}{1}}
      \infer1[\be]{(\psi\ra\phi)}
      \infer2[\ie]{\phi}
      \infer[left label=\cir{1}]1[\ii]{((\phi\lra\psi)\ra\phi)}

      \hypo{D}
      \infer[rule style=no rule]1{\psi}
      \infer1[\ii]{(\phi\ra\psi)}

      \hypo{\danda{\phi}{2}}
      \infer1[\ii]{(\psi\ra\phi)}
      \infer2[\bi]{(\phi\lra\psi)}
      \infer[left label=\cir{2}]1[\ii]{(\phi\ra(\psi\ra\phi))}
      \infer2[\ii]{((\phi\lra\psi)\lra\phi)}
    \end{prooftree}
\]

2.5.4) Sequent Rule \(\lra I\): If the sequents \((\Gamma \cup \{\phi\}\vdash
\psi)\) and \((\Delta \cup \{\psi\} \vdash \phi)\) are correct, then so is
\((\Gamma \cup \Delta\vdash (\phi\lra\psi))\).

Sequent Rule \(\lra E\): If the sequent \((\Gamma \vdash (\phi\lra\psi))\) is
correct, then so are \((\Gamma\vdash(\phi\ra\psi))\) and \((\Gamma \vdash
(\psi \ra \phi))\).

2.6.1)
\begin{enumerate}
  \item 
  \[
    \begin{prooftree}
      \hypo{\danda{(\phi\land(\neg \phi))}{1}}
      \infer1[\ce]{\phi}
      \hypo{\danda{(\phi\land(\neg \phi))}{1}}
      \infer1[\ce]{(\neg\phi)}
      \infer2[\nege]{\bot}
      \infer[left label=\cir{1}]1[\negi]{(\neg(\phi\land(\neg \phi)))}
    \end{prooftree}
  \]
  \item 
  \[
    \begin{prooftree}
      \hypo{\danda{(\neg(\phi\ra\psi))}{2}}
      \hypo{\danda{\psi}{1}}
      \infer1[\ii]{(\phi\ra\psi)}
      \infer2[\nege]{\bot}
      \infer[left label=\cir{1}]1[\negi]{(\neg\psi)}
      \infer[left label=\cir{2}]1[\ii]{((\neg(\phi\ra\psi))\ra(\neg\psi))}
    \end{prooftree}
  \]
  \item 
  \[
    \begin{prooftree}
      \hypo{\danda{(\phi\land\psi)}{2}}
      \infer1[\ce]{\psi}

      \hypo{\danda{(\phi\land\psi)}{2}}
      \infer1[\ce]{\phi}
      \hypo{\danda{(\phi\ra(\neg\psi))}{1}}
      \infer2[\ie]{(\neg\psi)}
      \infer2[\nege]{\bot}
      \infer[left label=\cir{1}]1[\negi]{(\neg(\phi\ra(\neg\psi)))}
      \infer[left label=\cir{2}]1[\ii]{((\phi\land\psi)\ra(\neg(\phi\ra(\neg\psi))))}
    \end{prooftree}
  \]
  \item 
  \[
    \begin{prooftree}
      \hypo{((\neg(\phi\land\psi))\land\phi)}
      \infer1[\ce]{(\neg(\phi\land\psi))}

      \hypo{((\neg(\phi\land\psi))\land\phi)}
      \infer1[\ce]{\phi}
      \hypo{\danda{\psi}{1}}
      \infer2[\ie]{(\phi\land\psi)}
      \infer2[\nege]{\bot}
      \infer[left label=\cir{1}]1[\negi]{(\neg\psi)}
    \end{prooftree}
  \]
  \item 
  \[
    \begin{prooftree}
      \hypo{\danda{(\neg\psi)}{2}}
      \hypo{\danda{\phi}{1}}
      \hypo{(\phi\ra\psi)}
      \infer2[\ie]{\psi}
      \infer2[\nege]{\bot}
      \infer[left label=\cir{1}]1[\negi]{(\neg\phi)}
      \infer[left label=\cir{2}]1[\ii]{((\neg\psi)\ra(\neg\phi))}
    \end{prooftree}
  \]
  \item 
  \[
    \begin{prooftree}
      \hypo{\danda{(\phi\land(\neg\psi))}{1}}
      \infer1[\ce]{\phi}

      \hypo{(\phi\ra\psi)}
      \infer2[\ie]{\psi}

      \hypo{\danda{(\phi\land(\neg\psi))}{1}}
      \infer1[\ce]{(\neg\psi)}
      \infer2[\nege]{\bot}
      \infer[left label=\cir{1}]1[\negi]{(\neg(\phi\land(\neg\psi)))}
    \end{prooftree}
  \]
\end{enumerate}

2.6.2)
\begin{enumerate}
  \item 
  \[
    \begin{prooftree}
      \hypo{\danda{\phi}{2}}
      \hypo{\danda{(\neg\psi)}{1}}
      \hypo{((\neg\psi)\ra(\neg\phi))}
      \infer2[\ie]{(\neg\phi)}
      \infer2[\nege]{\bot}
      \infer[left label=\cir{1}]1[\raa]{\psi}
      \infer[left label=\cir{2}]1[\negi]{(\phi\ra\psi)}
    \end{prooftree}
  \]
  \item
  \[
    \begin{prooftree}
      \hypo{\danda{\phi}{1}}
      \hypo{\danda{(\neg\phi)}{2}}
      \infer2[\nege]{\bot}
      \infer1[\raa]{\psi}
      \infer[left label=\cir{1}]1[\ii]{(\phi\ra\psi)}
      \hypo{\danda{(\neg(\phi\ra\psi))}{3}}
      \infer2[\nege]{\bot}
      \infer[left label=\cir{2}]1[\raa]{\phi}
      \infer[left label=\cir{3}]1[\ii]{((\neg(\phi\ra\psi))\ra\phi)}
    \end{prooftree}
  \]
  \item 
  \[
    \begin{prooftree}
      \hypo{\danda{\phi}{2}}
      \hypo{\danda{(\neg\phi)}{1}}
      \infer2[\nege]{\bot}
      \infer1[\raa]{\psi}
      \infer[left label=\cir{1}]1[\ii]{((\neg\phi)\ra\psi)}
      \infer[left label=\cir{2}]1[\ii]{(\phi\ra((\neg\phi)\ra\psi))}
    \end{prooftree}
  \]
  \item 
  \[
    \begin{prooftree}
      \hypo{\danda{\phi}{1}}
      \hypo{\danda{(\neg\phi)}{4}}
      \infer2[\nege]{\bot}
      \infer1[\raa]{\psi}
      \infer[left label=\cir{1}]1[\ii]{(\phi\to\psi)}
      \hypo{\danda{\psi}{2}}
      \hypo{\danda{(\neg\psi)}{3}}
      \infer2[\nege]{\bot}
      \infer1[\raa]{\phi}
      \infer[left label=\cir{2}]1[\ii]{(\psi\to\phi)}
      \infer2[\bi]{(\phi\lra\psi)}
      \hypo{(\neg(\phi\lra\psi))}
      \infer2[\nege]{\bot}
      \infer[left label=\cir{3}]1[\raa]{\psi}
      \infer[left label=\cir{4}]1[\ii]{(\neg(\phi\to\psi))}

      \hypo{(\neg(\phi\lra\psi))}

      \hypo{\danda{\psi}{6}}
      \infer1[\ii]{(\phi\to\psi)}
      \hypo{\danda{\phi}{5}}
      \infer1[\ii]{(\psi\to\phi)}
      \infer2[\bi]{(\phi\lra\psi)}
      \infer2[\nege]{\bot}
      \infer[left label=\cir{5}]1[\negi]{(\neg\phi)}
      \infer[left label=\cir{6}]1[\ii]{(\psi\to(\neg\phi))}
      \infer2[\bi]{((\neg\phi)\lra\psi)}
    \end{prooftree}
  \]
\end{enumerate}

2.7.1)
\begin{enumerate}
  \item
  \[
    \begin{prooftree}
      \hypo{\danda{\phi}{1}}
      \infer1[\di]{(\phi\lor\psi)}
      \infer[left label=\cir{1}]1[\ii]{(\phi\ra(\phi\lor\psi))}
    \end{prooftree}
  \]
  \item
  \[
    \begin{prooftree}
      \hypo{(\neg(\phi\lor\psi))}
      \hypo{\danda{\phi}{1}}
      \infer1[\di]{(\phi\lor\psi)}
      \infer2[\nege]{\bot}
      \infer[left label=\cir{1}]1[\negi]{(\neg\phi)}

      \hypo{(\neg(\phi\lor\psi))}
      \hypo{\danda{\psi}{2}}
      \infer1[\di]{(\phi\lor\psi)}
      \infer2[\nege]{\bot}
      \infer[left label=\cir{2}]1[\negi]{(\neg\psi)}

      \infer2[\di]{((\neg\phi)\land(\neg\psi))}
    \end{prooftree}
  \]
  \item
  \[
    \begin{prooftree}
      \hypo{\danda{(\neg\phi)}{1}}
      \infer1[\di]{((\neg\phi)\lor\psi)}
      \hypo{\danda{(\neg(\neg(\phi\lor\psi)))}{3}}
      \infer2[\nege]{\bot}
      \infer[left label=\cir{1}]1[\raa]{\phi}

      \hypo{\danda{\phi}{2}}
      \hypo{\danda{(\phi\ra\psi)}{4}}
      \infer2[\ie]{\psi}
      \infer1[\di]{((\neg\phi)\lor\psi)}
      \hypo{\danda{(\neg((\neg\phi)\lor\psi))}{3}}
      \infer2[\nege]{\bot}
      \infer[left label=\cir{2}]1[\negi]{(\neg\phi)}
      \infer2[\nege]{\bot}
      \infer[left label=\cir{3}]1[\raa]{((\neg\phi)\lor\psi)}
      \infer[left label=\cir{4}]1[\raa]{((\phi\ra\psi)\ra((\neg\phi)\lor\psi))}
    \end{prooftree}
  \]
\end{enumerate}

2.7.2)
\begin{enumerate}
  \item
  \[
    \begin{prooftree}
      \hypo{(\phi\lor\psi)}
      \hypo{\danda{\phi}{1}}
      \infer1[\di]{(\psi\lor\phi)}
      \hypo{\danda{\psi}{1}}
      \infer1[\di]{(\psi\lor\phi)}
      \infer[left label=\cir{1}]3[\de]{(\psi\lor\phi)}
    \end{prooftree}
  \]
  \item
  \[
    \begin{prooftree}
      \hypo{(\phi\lor\psi)}
      \hypo{\danda{\phi}{1}}
      \hypo{(\phi\ra\chi)}
      \infer2[\ie]{\chi}
      \hypo{\danda{\psi}{1}}
      \hypo{(\psi\ra\chi)}
      \infer2[\ie]{\chi}
      \infer[left label=\cir{1}]3[\de]{\chi}
    \end{prooftree}
  \]
  \item
  \[
    \begin{prooftree}
      \hypo{(\phi\lor\psi)}

      \hypo{\danda{\phi}{1}}
      \hypo{(\neg\phi)}
      \infer2[\nege]{\bot}
      \infer1[\raa]{\psi}
      \hypo{\danda{\psi}{1}}
      \infer[left label=\cir{1}]3[\de]{\psi}
    \end{prooftree}
  \]
  \item
  \[
    \begin{prooftree}
      \hypo{\danda{(\phi\lor\psi)}{2}}

      \hypo{\danda{\phi}{1}}
      \hypo{((\neg\phi)\land(\neg\psi))}
      \infer1[\ce]{(\neg\phi)}
      \infer2[\nege]{\bot}

      \hypo{\danda{\psi}{1}}
      \hypo{(\neg\psi)}
      \infer2[\nege]{\bot}
      \infer[left label=\cir{1}]3[\de]{\bot}
      \infer[left label=\cir{2}]1[\negi]{(\neg(\phi\lor\psi))}

    \end{prooftree}
  \]
  \item
  \[
    \begin{prooftree}
      \hypo{\danda{((\neg\phi)\lor(\neg\psi))}{2}}

      \hypo{(\phi\land\psi)}
      \infer1[\ce]{\phi}
      \hypo{\danda{(\neg\phi)}{1}}
      \infer2[\nege]{\bot}

      \hypo{(\phi\land\psi)}
      \infer1[\ce]{\psi}
      \hypo{\danda{(\neg\psi)}{1}}
      \infer2[\nege]{\bot}
      \infer[left label=\cir{1}]3[\de]{\bot}
      \infer[left label=\cir{2}]1[\negi]{(\neg((\neg\phi)\lor(\neg\psi)))}
    \end{prooftree}
  \]
\end{enumerate}

\newpage 
\bibliography{refs}
\bibliographystyle{unsrt}

\end{document}
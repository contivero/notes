\documentclass{article}

\usepackage[utf8]{inputenc}
\usepackage{caption}
\usepackage{float}
\usepackage{scrextend}
\usepackage{hyperref}
\hypersetup{pdfauthor={Cristian Adrián Ontivero}}

\usepackage{adjustbox}
\usepackage{tikz}
\usetikzlibrary{positioning}
\usetikzlibrary{external}
\tikzexternalize[prefix=tikz/, up to date check=md5]
\usepackage{ebproof}
\usepackage{booktabs}
\usepackage{multicol}
% Set items in enumerate environment to letters.
\usepackage{enumitem}
\setenumerate[0]{label=(\alph*)}
\usepackage{forest}
\forestset{%
  % Tree following the style of Chiswell & Hodges' Mathematical Logic
  chtree/.style={
    baseline,
    for tree={
      inner sep=1.4pt,
      circle,
      draw,
      s sep'+=20pt,
      fit=band,
    },
  },
  c phantom/.style={draw=none, no edge},
}

\usepackage{xspace}
\graphicspath{{imgs/}}

\newlength\tindent%
\setlength{\tindent}{\parindent}
\setlength{\parindent}{0pt}
\renewcommand{\indent}{\hspace*{\tindent}}

%These tell TeX which packages to use.
\usepackage{array, epsfig}
\usepackage{amsmath}
\usepackage{amsfonts}
\usepackage{amssymb}
\usepackage{amsxtra}
\usepackage{amsthm}
\usepackage{mathrsfs}

% Use in the array environment to set a column to text
\newcolumntype{R}{>\(r<\)}
\newcolumntype{L}{>\(l<\)}
\newcolumntype{C}{>\(c<\)}

%Here I define some theorem styles and shortcut commands for symbols I use often
\theoremstyle{definition}

\newcommand{\ra}{\rightarrow}
\newcommand{\surj}{\twoheadrightarrow}
\newcommand{\bb}[1]{\mathbb{#1}}
\newcommand{\Z}{\bb{Z}}
\newcommand{\Q}{\bb{Q}}
\newcommand{\R}{\bb{R}}
\newcommand{\C}{\bb{C}}
\newcommand{\N}{\bb{N}}

\newcommand{\lra}{\leftrightarrow}
\newcommand{\true}{\texttt{true}}
\newcommand{\false}{\texttt{false}}
% Logic rules
\newcommand{\ii}{\footnotesize (\(\rightarrow\)I)}
\newcommand{\ie}{\footnotesize (\(\rightarrow\)E)}
\newcommand{\ci}{\footnotesize (\(\wedge\)I)}
\newcommand{\ce}{\footnotesize (\(\wedge\)E)}
\newcommand{\di}{\footnotesize (\(\vee\)I)}
\newcommand{\de}{\footnotesize (\(\vee\)E)}
\newcommand{\bi}{\footnotesize (\(\leftrightarrow\)I)}
\newcommand{\be}{\footnotesize (\(\leftrightarrow\)E)}
\newcommand{\negi}{\footnotesize (\(\neg\)I)}
\newcommand{\nege}{\footnotesize (\(\neg\)E)}
\newcommand{\raa}{\footnotesize (RAA)}
\newcommand*\cir[1]{\tikz[baseline=(char.base)]{
            \node[shape=circle,draw,inner sep=1pt] (char) {\footnotesize #1};}}

            \newcommand{\danda}[2]{%
    \tikz[baseline=(tocancel.base)]{%
        \node[inner sep=0pt,outer sep=0pt] (tocancel) {\(#1\)};
        \node[inner sep=0pt,outer sep=1pt,above right=0mm of tocancel] (end)
        {\cir{#2}};
        \draw (tocancel.south west) -- (end.260);
    }\hspace{-1.2em}%
}%
%Pagination stuff.
\setlength{\topmargin}{-.3 in}
\setlength{\oddsidemargin}{0in}
\setlength{\evensidemargin}{0in}
\setlength{\textheight}{9.in}
\setlength{\textwidth}{6.5in}
\pagestyle{empty}

\newcommand{\fls}{\texttt{fls}\xspace}
\newcommand{\tru}{\texttt{tru}\xspace}
\renewcommand{\and}{\texttt{and}}
\newcommand{\fst}{\texttt{fst}}
\newcommand{\snd}{\texttt{snd}}
\newcommand{\iszero}{\texttt{iszero}\xspace}
\newcommand{\minus}{\texttt{minus}}
\renewcommand{\l}[1]{\lambda #1.~}
\newcommand{\pair}[1]{\langle #1 \rangle}
\newcommand{\num}[1]{\overline{\texttt{#1}}}

\newcommand{\ttrue}{\small T-True}
\newcommand{\tfalse}{\small T-False}
\newcommand{\tapp}{\small T-App}
\newcommand{\tabs}{\small T-Abs}
\newcommand{\tif}{\small T-If}
\newcommand{\tvar}{\small T-Var}
\newcommand{\Bool}{\texttt{Bool}}
\newcommand{\If}[3]{\texttt{if}~#1~\texttt{then}~#2~\texttt{else}~#3}
% Scott numeral
\newcommand{\snum}[1]{\underline{\texttt{#1}}}
\begin{document}

Useful site: \url{http://www.cburch.com/lambda/}

Booleans
\begin{align*}
  \tru &:= \l{t} \l{f} t \\
  \fls &:= \l{t} \l{f} f
\end{align*}

\(\mathtt{test} := \l{c} \l{t} \l{f} c~t~f \)
\\[.1cm]

Unary boolean functions:

\(\texttt{id} := \l{p} p\)

\(\texttt{constT} := \l{p} \tru\)

\(\texttt{constF} := \l{p} \fls\)

\(\mathtt{not} := \l{p} p~\fls~\tru\)
\\[.1cm]

Binary boolean functions:

\(\and := \l{p} \l{q} p~q~\fls \)

\(\mathtt{or} := \l{p} \l{q} p~\tru~q \)

\(\texttt{implication} := \l{p} \l{q} p~q~\tru\)

\(\texttt{converse\_implication} := \l{p} \l{q} q~p~\tru\)

\(\texttt{nonimplication} := \l{p} \l{q} q~\fls~p\)

\(\texttt{converse\_nonimplication} := \l{p} \l{q} p~\fls~q\)
Or simpler, by \(\eta\)-reduction: \(\l{p} p~\fls\)

\(\texttt{equiv} := \l{p} \l{q} p~q~(q~\fls~\tru)\)

\(\texttt{xor} := \l{p} \l{q} p~(q~\fls~\tru)~q\)

\(\texttt{nand} := \l{p} \l{q} p~(q~\fls~\tru)~\tru\)

\(\texttt{nor} := \l{p} \l{q} p~\fls~(q~\fls~\tru)\)
\\[.1cm]

\newpage

Pairs\\

\(\texttt{pair} := \l{x} \l{y} \l{f} f~x~y\)

\(\texttt{fst} := \l{p} p~\tru\)

\(\texttt{snd} := \l{p} p~\fls\)

For a more concise notation, we will use \(\pair{a, b}\) instead of writing
\((\texttt{pair}~a~b)\) explicitly.\\

Church numerals
\begin{align*}
  \num{0} &:= \l{s} \l{z} z \\
  \num{n+1} &:= \l{s} \l{z} s~\num{n}
\end{align*}

%5.2.2
\(\mathtt{succ} := \l{n} \l{s} \l{z} s~(n~s~z) \)

\(\mathtt{succ} := \l{n} \l{s} \l{z} n~s~(s~z) \)

\(\mathtt{plus} := \l{m} \l{n} \l{s} \l{z} m~s~(n~s~z) \)

\(\mathtt{times} := \l{m} \l{n} m~(\mathtt{plus}~n)~\num{0} \)

\(\mathtt{times} := \l{m} \l{n} \l{s} \l{z} m~(n~s)~z\)

Or \(\eta\)-reducing: \(\mathtt{times} := \l{m} \l{n} \l{s} m~(n~s)\)

\(\mathtt{power} := \l{m} \l{n} m~(\mathtt{times}~n)~\num{1} \)

\(\mathtt{power} := \l{m} \l{n} m~n \)

\(\mathtt{iszero} := \l{n} n~(\l{x} \fls)~\tru\)

\begin{flalign*}
\mathtt{pr}&\mathtt{ed} := \l{n} \mathtt{fst}~(n~\texttt{ss}~\pair{\num{0}, \num{0}}) \\
           &\text{where } \texttt{ss} := (\l{p} \pair{\snd~p, \texttt{succ}~(\snd~p)})
\end{flalign*}

\(\mathtt{leq} := \l{m} \l{n} \mathtt{iszero}~(\minus~m~n)\)

\(\mathtt{geq} := \l{m} \l{n} \mathtt{leq}~n~m\)

\(\mathtt{lt} := \l{m} \l{n} \mathtt{leq}~(\mathtt{succ}~m)~n\)

\(\mathtt{gt} := \l{m} \l{n} \mathtt{lt}~n~m\)

% 5.2.5
\(\minus := \l{n} \l{m} m~\texttt{pred}~n\)

% 5.2.7
\(\mathtt{equals} := \l{m} \l{n} \and~(\mathtt{lt}~m~n))~(\mathtt{lt}~n~m)) \)

\newpage
Lists:
Lets start with \texttt{cons}, which is a function that receives a head
element, the list to be the tail, and appends the first to the second, i.e.\
something of the form \(\l{h} \l{t} \texttt{new\_list}\). We know the lists
fold function takes two arguments (intuitively, the constructor and nil
functions), so \(\texttt{new\_list}\) will take the form \(\l{c} \l{n}
\texttt{something}\). Lastly, based on the example, we need to apply \(c\)
to an element, and a list, so it makes sense to apply
it to \(h\) (the head element), and to \(t\) (the tail function, which needs
to be a list). The tail function \(t\) needs to be applied to \(c\)
and \(n\) to get back the list we require as second argument to the first
\(c\). We thus reach
\[
  \mathtt{cons} := \l{h} \l{t} \l{c} \l{n} c~h~(t~c~n)
\]
Now, \texttt{nil} is a list, which (like always) we represent
with the list's fold function which takes two arguments. Since we need not
apply the constructor function to anything:
\[
  \texttt{nil} := \l{c} \l{n} n
\]
(note that it coincides with \(\num{0}\) or \fls).
For \texttt{isnil}, we follow the reasoning of \iszero: we need a function
that takes a list, and feeds it two arguments. If the list is \texttt{nil},
we know that it will just return the second argument, so we make that
argument \tru. If the list is a \texttt{cons}, it will apply its first
argument to two others, so we make the first argument \(\l{y} \l{z} \fls\),
to get back \fls after applying the function; hence:
\[
  \texttt{isnil} := \l{l} l~(\l{h} \l{t} \fls)~\tru
\]
For \texttt{head}, if it's a \texttt{cons}, it will apply its first argument
to the head and to the tail, so we use as first argument \(\l{h} \l{t} h\).
If it's a \texttt{nil}, \texttt{head} makes no sense (an empty list has no
head), so we can return whatever we want. We can use \fls as a placeholder,
but just using the list itself is easier:
\[
  \texttt{head} := \l{l} l~(\l{h} \l{t} h)~l
\]
Lastly, \texttt{tail} follows a similar reasoning as the \texttt{pred}
function for church numerals. We need to iterate a pair which has the list
itself as second element of the pair, and its tail as first element. We thus
apply the list function to a function \texttt{ff} which works as the
``constructor'' (it must take two arguments), and the pair
\(\pair{\texttt{nil}, \texttt{nil}}\) as second argument (the ``base case'').
\begin{flalign*}
  \texttt{ta}&\texttt{il} := \l{l} \fst~(l~\texttt{ff}~\pair{\texttt{nil}, \texttt{nil}}) \\
           &\text{where } \texttt{ff} := (\l{h} \l{p} \pair{\snd~p, \texttt{cons}~h~(\snd~p)})
\end{flalign*}
Note that anything works as first element of the pair, as long as the second
element is \texttt{nil}.

\newpage
Scott numerals
\begin{align*}
  \snum{0} &:= \l{z} \l{s} z \\
  \snum{n+1} &:= \l{z} \l{s} s~\snum{n}
\end{align*}
When compared to the Church numerals, Scott numerals are distinguished by
their ``linearity'': the bound variables of \(\snum{0}\), \texttt{succ}, and
\texttt{case} occur at most once in the bodies of these functions,
and the predecessor function \(\texttt{pred} := \l{n} n~\snum{0}~(\l{x} x)\)
can be computed trivially\cite{ACP93}.
\\

\(\mathtt{succ} := \l{n} \l{z} \l{s} s~(n~z~s)\)

\(\mathtt{pred} := \l{n} n~\snum{0}~(\l{x} x)\)

\(\mathtt{iszero} := \l{n} n~\tru~(\l{x} \fls)\)

\(\mathtt{add} := \l{n} \l{m} (\l{x} n~m~x)~(\l{y} (\mathtt{add}~y~m) + 1) \)

\(\mathtt{case} := \l{n} \l{a} \l{f} n~a~f\), where \(\mathtt{case}~n~a~f\)
returns \(a\) if \(n\) is 0, and \(f~x\) if \(n\) is the successor of \(x\).

\bibliography{refs}
\bibliographystyle{unsrt}

\end{document}
\documentclass{article}

\usepackage{hyperref}
\hypersetup{pdfauthor={Cristian Adrián Ontivero}}
\usepackage[utf8]{inputenc}
\usepackage{graphicx}
\usepackage{gensymb} % for the degree symbol
\usepackage{caption}
\usepackage{float}
\usepackage{color}
\usepackage{epigraph}
\usepackage[hang,flushmargin]{footmisc} 

% For the commutative diagrams.
\usepackage{tikz-cd}

\usepackage{tikz}
\usetikzlibrary{automata, positioning, arrows,
fit, % for the dashed boxes on Thompson's construction
calc
}
\tikzset{%
  node distance=3cm, % specifies the minimum distance between two nodes. Change if necessary.
  every state/.style={thick}, % sets the properties for each ’state’ node
  double distance=2.5pt,
  shorten >= 2pt, shorten <= 2pt,
  initial text=$ $,
  every edge/.style={%
    draw,->, >=stealth, auto, semithick
  }
}
\graphicspath{{imgs/}}

\definecolor{darkblue}{RGB}{49,130,189}

\newlength\tindent
\setlength{\tindent}{\parindent}
\setlength{\parindent}{0pt}
\renewcommand{\indent}{\hspace*{\tindent}}

%These tell TeX which packages to use.
\usepackage{array,epsfig}
\usepackage{amsmath}
\usepackage{amsfonts}
\usepackage{amssymb}
\usepackage{amsxtra}
\usepackage{amsthm}
\usepackage{mathrsfs}

%Here I define some theorem styles and shortcut commands for symbols I use often
\theoremstyle{definition}

\newcommand{\lra}{\longrightarrow}
\newcommand{\ra}{\rightarrow}
\newcommand{\surj}{\twoheadrightarrow}
\newcommand{\graph}{\mathrm{graph}}
\newcommand{\bb}[1]{\mathbb{#1}}
\newcommand{\Z}{\bb{Z}}
\newcommand{\Q}{\bb{Q}}
\newcommand{\R}{\bb{R}}
\newcommand{\C}{\bb{C}}
\newcommand{\N}{\bb{N}}
\newcommand{\M}{\mathbf{M}}
\newcommand{\m}{\mathbf{m}}
\newcommand{\MM}{\mathscr{M}}
\newcommand{\HH}{\mathscr{H}}
\newcommand{\Om}{\Omega}
\newcommand{\Ho}{\in\HH(\Om)}
\newcommand{\bd}{\partial}
\newcommand{\del}{\partial}
\newcommand{\bardel}{\overline\partial}
\newcommand{\textdf}[1]{\textbf{\textsf{#1}}\index{#1}}
\newcommand{\img}{\mathrm{img}}
\newcommand{\ip}[2]{\left\langle{#1},{#2}\right\rangle}
\newcommand{\inter}[1]{\mathrm{int}{#1}}
\newcommand{\exter}[1]{\mathrm{ext}{#1}}
\newcommand{\cl}[1]{\mathrm{cl}{#1}}
\newcommand{\ds}{\displaystyle}
\newcommand{\vol}{\mathrm{vol}}
\newcommand{\cnt}{\mathrm{ct}}
\newcommand{\osc}{\mathrm{osc}}
\newcommand{\LL}{\mathbf{L}}
\newcommand{\UU}{\mathbf{U}}
\newcommand{\support}{\mathrm{support}}
\newcommand{\AND}{\;\wedge\;}
\newcommand{\OR}{\;\vee\;}
\newcommand{\Oset}{\varnothing}
\newcommand{\st}{\ni}
\newcommand{\wh}{\widehat}

\newcommand{\PS}{\mathcal{P}}
\newcommand{\set}[1]{\left\{#1\right\}}
\newcommand{\bra}[1]{\left[#1\right]}
\newcommand{\abs}[1]{\left|#1\right|}
\newcommand{\ec}[1]{{\left[#1\right]}_{\sim}}
\newcommand{\thra}{\twoheadrightarrow}
\newcommand{\emptystr}{\varepsilon}
\newcommand{\emptylan}{\emptyset}
\newcommand{\pdv}[2]{\partial_{#1} \bigl(#2\bigr)}

%Pagination stuff.
\setlength{\topmargin}{-.3 in}
\setlength{\oddsidemargin}{0in}
\setlength{\evensidemargin}{0in}
\setlength{\textheight}{9.in}
\setlength{\textwidth}{6.5in}
\pagestyle{empty}

% The problem environment is a regular ams theorem environment with "Problem"
% text and some leading space to give some separation between the problems.
\theoremstyle{definition}
\newtheorem{problem-internal}{Problem}[subsection]
\newenvironment{problem}{
  \medskip
  \begin{problem-internal}
}{
  \end{problem-internal}
}

% The solution environment is a proof environment with the "solution" text as
% well as the following adjustments:
% - No indent on paragraphs;
% - A small amount of space between paragraphs.
%
% Note: The negative space at the beginning is to remove the space before the
% first paragraph in the solution.
%\newenvironment{solution}{%
  %\begin{proof}[Solution]
  %\vspace{-8px}
  %\setlength{\parskip}{4px}
  %\setlength{\parindent}{0px}
%}{
  %\end{proof}
%}
\theoremstyle{definition}
\newtheorem{solution-internal}{}[subsection]
\newenvironment{solution}{
  \medskip
  \begin{solution-internal}
}{
  \end{solution-internal}
}

% The chngcntr ("change counter") package is used here so that subsection
% numbers are written without the leading section number. This takes place in
% the subsection headings as well as the theorem environment numbering.
%
% Before:
% 1. Section
% 1.1. Subsection
% Problem 1.1.1. What is 1 + 1?
% Problem 1.1.2. What is 1 + 2?
%
% After:
% 1. Section
% 1. Subsection
% Problem 1.1. What is 1 + 1?
% Problem 1.2. What is 1 + 2?
\usepackage{chngcntr}
\counterwithout{subsection}{section}

% Renewing the \thesection command changes the section numbers to roman
% numerals. This matches the style of the Aluffi textbook.
%
% Before:
% 1. Section
% 1.1. Subsection
%
% After:
% I. Section
% I.1. Subsection
\renewcommand{\thesection}{\Roman{section}} 

\begin{document}

\section*{Chapter 1}
\subsection*{Preliminaries: Set theory and categories}
\setcounter{subsection}{1}
\stepcounter{solution-internal}

\begin{solution}
Suppose $\sim$ is an equivalence relation on a set $S$.
Since $\sim$ is reflexive, $\forall x \in S (x \sim x) $, so $\forall x\in S
(x \in {[x]}_{\sim})$. Thus, no equivalence class is empty, and their union is
the whole set $S$.  It remains to show the disjointness of the equivalence
classes.  Let $x$ and $y$ be two elements of $S$. Suppose ${[x]}_{\sim} \cap
{[y]}_{\sim} \neq \emptyset$. Then there exists $z \in {[x]}_{\sim} \cap
{[y]}_{\sim}$, i.e. $z \in {[x]}_{\sim} \wedge z \in {[y]}_{\sim}$. By
definition of equivalence class, $z \sim x$ and $z \sim y$. By symmetry $x \sim
z$, and by transitivity $x \sim y$. Once more, by symmetry $y \sim x$. Knowing
all this, we proceed to prove that ${[x]}_{\sim} = {[y]}_{\sim}$.
\begin{itemize}
\item[$\subseteq$:] Let $d \in {[x]}_{\sim}$, this means that $d \sim x$. By
transitivity $d \sim y$, so $d \in {[y]}_{\sim}$.
\item[$\supseteq$:] Let $d \in {[y]}_{\sim}$, this means that $d \sim y$. By
transitivity $d \in x$, so $d \in {[x]}_{\sim}$.
\end{itemize}

Thus ${[x]}_{\sim} = {[y]}_{\sim}$.
\end{solution}

\begin{solution}
1.3 Given a partition $\mathcal{P}$ on a set $S$, we define a relation $\sim$ on
$S$ as follows: $x \sim y$ iff $x, y \in X$, for some $X \in \mathcal{P}$.
Let's prove that $\sim$ is an equivalence relation on S.
\begin{itemize}
  \item Reflexive: Let $x \in S$ be arbitrary. $x \in X$ for some $X \in
    \mathcal{P}$, so $x \sim x$.
  \item Symmetric: Let $x, y \in S$ be such that $x \sim y$ and $y \sim z$.
Then by definition of $\sim$ there ixists an $X \in \PS$ such that $x, y \in X$.
This also means that $x, y \in X$, thus $y \sim x$.
  \item Transitive: Let $x, y, z \in S$ be such that $x \sim y$ and $y \sim z$.
Then by definition $x, y \in X_1$ and $y,z \in X_2$, with $X_1, X_2 \in \PS$.
Since $\PS$ is a partition, $y$ is only in one part, thus $X_1 = X_2$. Therefore
$x, z \in X_1$, so $x \sim z$.
\end{itemize}
It now remains to prove that $\PS$ is $\PS_{\sim}$, the set of equivalence classes of $\sim$. Let $X \in \PS$ be a part of the partition. By definition of $\sim$, $\forall x, y \in X (x \sim y)$. Furthermore, given $a \in X$ and $S - X$ we know by definition that $a$ and $b$ are not related. Thus $X$ is an equivalence class of $\sim$. Finally, since every element of $S$ is in exactly one of the parts of $\PS$, we have that there are no equivalence classes besides the ones from the partition. Therefore $\PS$ is the set of equivalence classes of $\sim$.
\end{solution}

\begin{solution}
Five. These are:
\begin{align*}
  \sim_1 &= \set{(1,1), (2,2), (3,3)} \\
  \sim_2 &= \set{(1,1), (1,2), (2,1), (2,2), (3,3)} \\
  \sim_3 &= \set{(1,1), (1,3), (3,1), (3,3), (2,2)} \\
  \sim_4 &= \set{(1,1), (2,2), (2,3), (3,2), (3,3)} \\
  \sim_5 &= \set{(1,1), (1,2), (1,3), (2,2), (2,1), (2,3), (3,1), (3,2), (3,3)}
\end{align*}
This might be easier to think in terms of how many partitions of $\set{1,2,3}$ we can make:
\begin{align*}
  \PS_1 &= \set{\{1\}, \{2\}, \{3\}} \\
  \PS_2 &= \set{\{1, 2\}, \{3\}} \\
  \PS_3 &= \set{\{1, 3\}, \{2\}} \\
  \PS_4 &= \set{\{1\}, \{2, 3\}} \\
  \PS_5 &= \set{\{1, 2, 3\}}
\end{align*}
In general, the number of ways to partition a set of $n$ elements is given by the $n^{\text{th}}$ Bell number.
\end{solution}

\begin{solution}
Take $S = \set{1,2,3}$ and $R = \set{(1,1), (2,2), (3,3), (1,2), (2,1),
(2,3), (3,2)}$. $R$ is reflexive and symmetric, but not transitive. Recall that
an equivalence class of $x$ with respect to $R$ is
\[ \bra{x}_R = \set{y \in S \mid yRx} \]
Thus we have $\bra{1}_R = \set{1,2}$ and $\bra{2}_R = \set{1,2,3}$. Clearly
$\bra{1}_R \neq \bra{2}_R$, but they aren't disjoint ($\bra{1}_R \cap \bra{2}_R
\neq \emptyset$), so they can't form a partition.
\end{solution}

\begin{solution}
$\forall a, b \in \R (a \sim b \iff b - a \in \Z)$.
\begin{itemize}
\item Reflexive: $\forall a \in R, a \sim a \iff a - a = 0 \in Z$
\item Symmetric: Let $a,b \in \R$ by such that $a \sim b$. Then $b - a \in Z$, so its opposite $-(b-a) \in Z$, i.e. $a-b \in Z$, thus $b \sim a$.
\item Transitive: Let $a, b, c \in \R$ by such that $a \sim b$ and $b \sim c$. Then $b - a \in Z$, and $c - b \in Z$. Since $\Z$ is closed under addition, $(b-a) + (c-b) = c - a \in Z$. Thus $a \sim c$.
\end{itemize}
Under $\sim$ all reals with the same decimal expansion belong to the same
equivalence class. In other words $\forall x \in [0,1), \bra{x}_{\sim} = \set{n + x \mid n \in \Z}$.
In particular $\bra{0} = \Z$. Thus $\R/\sim = \R/\Z$ which is isomorphic to
$[0,1)$ and the unit circle (the book until here doesn't get into quotient
groups though!).

\[ \forall (a_1, a_2), (b_1, b_2) \in \R \times \R, (a_1, a_2) \approx (b_1, b_2) \iff b_1 - a_1, b_2 - a_2 \in Z \]

\begin{itemize}
\item Reflexive: Let $(a_1, a_2) \in \R^2$, $(a_1, a_2) \approx (a_1, a_2) \iff a_1 - a_1 = 0 \in \Z$ and $a_2 - a_2 = 0 \in \Z$.
\item Symmetric: Let $(a_1, a_2), (b_1, b_2) \in \R^2$ be such that $(a_1, a_2) \approx (b_1, b_2)$.
Then $b_1 - a_1 \in \Z \AND -(b_2 - a_1) \in \Z$, i.e.  $a_1 -b_1 \in \Z \AND a_2 - b_2 \in \Z$.
Therefore $(b_1, b_2) \approx (a_1, a_2)$.
\item Transitive: Let $(a_1, a_2), (b_1, b_2), (c_1, c_2) \in \R^2$ be such that $(a_1, a_2) \approx (b_1, b_2)$ and $(b_1, b_2) \approx (c_1, c_2)$. Then $(b_1 - a_1), (b_2 - a_2), (c_1 - b_1), (c_2 - b_2) \in \Z$, so $(b_1 - a_1) + (c_1 - b_1) = c_1 - a_1 \in \Z$ and $(b_2 - a_2) + (c_2 - b_2) = c_2 - a_2 \in \Z$, because $\Z$ is closed under addition. Therefore $(a_1, a_2) \approx (c_1, c_2)$.
\end{itemize}
\end{solution}


\section*{Chapter 2}
\subsection*{TODO complete title}
\setcounter{subsection}{2}
\setcounter{solution-internal}{0}

Check from \S 2.5.
Let $f \colon A \to B$ be a bijection, and define $g \colon B \to A$ by $g(b)=a$ whenever $f(a)=b$. We need to show that $g$, the flip of $\Gamma_f$ is a
function according to the definition given, which is: 
\[ (\forall b \in B)(\exists! a \in A) (b,a) \in \Gamma_g \]
We prove it by contradiction. Suppose $\neg (\forall b \in B)(\exists! a \in A)
(b,a) \in \Gamma_g$. This is equivalent to:
\[ \exists b \in B \neg (\exists! a \in A) (b,a) \in \Gamma_g \]
Take this $b$. Either there is no element in $A$ such that $(b,a) \in \Gamma_g$, or there is more than one.
\begin{itemize}
\item Suppose $\neg \exists a \in A$ such that $(b,a) \in \Gamma_g$: then
$g(b)$ isn't defined, which by the way $g$ is defined implies that $f$ isn't
surjective, which is absurd.
\item Suppose there is more than one: then there are at least two different ones. Call them $c$ and $d$. Then $g(b) = c \AND g(b) = d \AND c \neq d$. By how $g$ is defined, this implies that $f(c) = b$, $f(d) = b$, and $c \neq d$, meaning that $f$ isn't injective, which is absurd.
\end{itemize}
Thus $(\forall b \in B)(\exists! a \in A) (b,a) \in \Gamma_g$.

Corollary 2.2: See Proposition 4.2

Canonical decomposition (2.8) (p.15)
\begin{itemize}
\item Reflexive: Let $a \in A$, $f(a) = f(a)$, so $a \sim a$.
\item Symmetric: Let $a, b \in A$ be such that $a \sim b$. Then $f(a) = f(b)$. Since $=$ is symmetric, $f(b) = f(a)$, so $b \sim a$.
\item Transitive: Let $a, b, c \in A$ be such that $a \sim b$ and $b \sim c$. Then $f(a) = f(b)$ and $f(b) = f(c)$. Since $=$ is transitive, $f(a) = f(c)$, so $a \sim c$.
\end{itemize}


% Problem 2.1
\begin{solution}
There are $n!$ different bijections from a set of $n$ elements to itself, which
amount to the different possible permutations of the set. Take the first
element. You can map it to any of the n elements. Once that has been chosen,
map a second element to the remaining $(n-1)$, and so on.
\end{solution}

% Problem 2.2
\begin{solution}
Assume $A \neq \emptyset$ and let $f \colon A \to B$ be a function. To be proved: $f$ has a right-inverse iff it is surjective.
\begin{itemize}
  \item[$(\Rightarrow)$:] Suppose $f$ has a right-inverse, then there exists $g \colon B \to A$ such that $f \circ g = id_B$. We need to show that $f$ is surjective. Let $b \in B$. Then $\exists a \in A$, namely $g(b)$, such that $f(a) = b$. Hence $f$ is surjective.
\item[$(\Leftarrow)$:] Suppose $f$ is surjective. For each $b \in B$ construct a set
\[ \Lambda_b = \set{(a,b) \mid a \in A, f(a) = b} \]
Every $\Lambda_b$ is nonempty due to $f$ begin surjective, so we can choose a
pair from each $\Lambda_b$ to define $g \colon B \to A$ by $g(b) = a$, where
$(a,b) \in \Lambda_b$. Since $b = f(a) = f(g(b)) = (f \circ g)(b)$, $g$ is a
right-inverse of $f$.
\end{itemize}
\end{solution}

% Problem 2.3
\begin{solution}
Suppose $f$ is a bijection, and $f^{-1}$ its inverse. Then $f^{-1}$ is a bijection too, since it has $f$ as its inverse: 
\[f \circ f^{-1} = id_B \AND f^{-1} \circ f = id_A \]
where $f \colon A \to B$ and $f^{-1} \colon B \to A$.
\end{solution}
Let $f\colon A \to B$ and $g\colon B \to C$ be two bijections, with $f^{-1}$
and $g^{-1}$ their inverses, respectively. Then $g \circ f \colon A \to C$ is
also a bijection, with its inverse $(g\circ f)^{-1}\colon C \to A$ being $f^{-1} \circ g^{-1}$. To see this:

\[(g \circ f)\circ {(g\circ f)}^{-1} = (g\circ f) \circ (f^{-1}\circ g^{-1}) = g \circ id_B \circ g^{-1} = id_C \]
\[{(g\circ f)}^{-1}\circ (g \circ f) = (f^{-1}\circ g^{-1}) \circ (g\circ f) = f^{-1} \circ id_B \circ f = id_A \]

% Problem 2.4
\begin{solution}
Let $A,B,C$ be sets. We need to show that the relation ``is isomorphic
to'', denoted $\cong$, is an equivalence relation.
\begin{itemize}
\item Reflexive: $id_A\colon A \to A$ is an isomorphism, so $A\cong A$.
\item Symmetric: Suppose $A \cong B$. Then there exists an isomorphism $f\colon A \to B$, and it has another isomorphism $f^{-1}\colon B \to A$ as its inverse. Thus $B\cong A$.
\item Transitive: Suppose $A\cong B$ and $B\cong C$. There there exists $f\colon A \to B$ and $g\colon B \to C$, both isomorphisms. From the previous problem we know that $(g \circ f)\colon A \to C$ is an isomorphism, so $A \cong C$.
\end{itemize}
\end{solution}

% Problem 2.5
\begin{solution}
A function $f\colon A \to B$ is an epimorphism (or epi) if the following holds:
for all sets $Z$ and all functions $\alpha', \alpha''\colon B \to Z (\alpha' \circ f = \alpha'' \circ f) \implies \alpha' = \alpha'')$.

Proposition: a function is surjective iff it is an epimorphism.
Proof
\begin{itemize}
\item[$(\Rightarrow)$:] Suppose $f\colon A \to B$ is surjective. By proposition 2.1 there is a function $g\colon B \to A$ which is its right-inverse. Assume $\alpha',\alpha''$ are arbitrary functions from $B$ to another set $Z$, and that $\alpha' \circ f = \alpha'' \circ f$.
Composing on the right by $g$, and using the associtivity of composition:
\begin{align*}
(\alpha' \circ f)\circ g &= (\alpha''\circ f)\circ g \\
\alpha' \circ (f\circ g) &= \alpha''\circ (f\circ g) \\
\alpha' \circ (id_B) &= \alpha''\circ (id_B) \\
\alpha' &= \alpha''
\end{align*}
Thus $f$ is an epimorphism.
\item[$(\Leftarrow)$:] We prove by contradiction. Suppose $f$ is not
surjective. Then there is some $b \in B$ such that $\forall a \in A (f(a) \neq
b)$.Take two functions $\alpha', \alpha''\colon B \to Z$ where $Z =
\set{z_1,z_2,\dots}$ has at least two elements. $\alpha'(b) = z_1$,
$\alpha''(b) = z_2$, and they agree on the rest of the points (say $\forall b' \in B - \set{b}, \alpha'(b') = \alpha''(b') = z_1)$.
Suppose $\alpha \circ f = \alpha'' \circ f$. Since $\alpha' \neq \alpha''$, $f$
cannot be epic. Therefore $f$ is surjective.
\end{itemize}
\end{solution}

% Problem 2.6
\begin{solution}
Let $A, B$ be sets. Given a function $f\colon A\to B$, we can define a function $s\colon A \to A \times B$ as $s(a) = (a, f(a))$. Then:

\begin{center}
\begin{tikzcd}
A \arrow[rr, two heads, "\pi_A"'] &  & B \arrow[ll, "f"', bend right]
\end{tikzcd}
\end{center}

i.e. $(\pi_A \circ g)(a) = \pi_A(g(a)) = \pi_A(a, f(a)) = a$.
\end{solution}

% Problem 2.7
\begin{solution}
To prove $\Gamma_A \cong A$ we need to find some isomorphism $g\colon \Gamma_f \to A$. Define $g = \pi_A|_{\Gamma_f}$ (i.e. the natural projection restricted to the subset $\Gamma_f$ of $A \times B$). This $g$ has a section $s$ as previously proved, namely $s(a) = (a, f(a))$. Thus we have:
\begin{align}
  (g\circ s)(a) = g(s(a)) = g(a, f(a)) = a \\
  (s\circ g)(a, f(a)) = s(g(a, f(a)) = s(a) = (a, f(a))
\end{align}
i.e. $g$ is the inverse of $s$, so it is an isomorphism.
\end{solution}

% Problem 2.8
\begin{solution}
The canonical decomposition of $f$ (theorem 2.7) is given by:

\begin{center}
\begin{tikzcd}
\R \arrow[r, two heads] \arrow[rrr, bend left] & \R/\sim \arrow[r, "\tilde f"', "\sim"] & im f \arrow[r, hook] & \C
\end{tikzcd}
\end{center}

where $f\colon \R \to \C$ is defined as $f(r) = e^{2\pi i r}$. Thus, $f$ can be decomposed into three functions:
\begin{enumerate}
\item The canonical projection $\R \thra \R/\sim$, which sends every $r \in R$
to its equivalence class $\ec{r}$. Since $f$ is periodic with period 1, the
relation $a \sim b$ on $\R$ means that $a-b$ is an integer (as in exercise
1.6). The space $\R/\sim$ is the set of all equivalence classes, so for example $\frac{1}{8}$ (i.e. 45 degrees) becomes the same as $\frac{9}{8}$ (a $360\degree + 45\degree = 405\degree$ angle).
\item A bijection $\tilde f$ defined by $\tilde f(\ec{r}) = f(r)$. $im f$ is
the unit circle $\set{z \in \C : \abs{z} = 1}$. $\tilde f$ takes the equivalene
class to which $r$ belongs, to the point $e^{2\pi i r}$. For instance, if $r = \frac{1}{12}$, then:
\[ 
\textstyle
\tilde f(\ec{\frac{1}{12}}) = e^{2\pi i (\frac{1}{12})} = \cos(\frac{\pi}{6}) + i\sin(\frac{\pi}{6}) = \frac{\sqrt{3}}{2} + i\frac{1}{2}
\]
that point is on the circle $\abs{z} = 1$. To see $\tilde f$ is iso, note that
it is mono because it takes different equivalence classes to different points,
and is epic because every point on the circle is the image of some equivalence
class.

\item A function taking a point in $im f$ to itself in $\C$, which is clearly mono.
\end{enumerate}
\end{solution}

% Problem 2.9
\begin{solution}
Let $A', A'', B', B''$ be as stated. Then we have two bijections $f\colon A' \to A''$ and $g\colon B' \to B''$. Define a function $h\colon A' \cup B' \to A'' \cup B''$ as:
\[ h(x) = 
\left\{
  \begin{array}{ll}
    f(x), & \text{if $x \in A'$} \\
    g(x), & \text{otherwise}
  \end{array}\right.
\]
We need to that $h$ is a bijection. Let $d \in A'' \cup B''$. Since $A'' \cap B'' = \emptyset$, $d \in A''$ or $d \in B''$.
\begin{itemize}
\item Suppose $d \in A''$, then there is some $x_0 \in A'$ such that $f(x_0) = d$, since $f$ is surjective.
\item Analogous reasoning applies if $d \in B''$.
\end{itemize}
Thus $h$ is surjective. Now let $d_1, d_2 \in A' \cup B'$ be such that $d_1 \neq d_2$. We have four possibilities (since $A' \cap B' = \emptyset$):
\begin{itemize}
\item Case $d_1, d_2 \in A'$: then $h(d_1) = f(d_1)$ and $h(d_2) = f(d_2)$. Since $f$ is injective, $f(d_1) \neq f(d_2)$, i.e. $h(d_1) \neq h(d_2)$.
\item Case $d_1, d_2 \in B'$: analogous to the previous case.
\item Case $d_1 \in A', d_2 \in B'$: then $h(d_1) = f(d_1) \neq g(d_2) = h(d_2)$.
\item Case $d_1 \in B', d_2 \in A'$: analogous to the previous case.
\end{itemize}
Thus $h$ is injective. So $h$ is bijective.
Since $A', A''$ correspond to ``copies'' of $A$, and $B', B''$ correspond to copies of $B$ for use in the disjoint set operation, the disjoint union is well-defined \textit{up to isomorphism}.

\end{solution}


\end{document}

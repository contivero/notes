\documentclass{article}

\usepackage[utf8]{inputenc}
\usepackage{babel}
\usepackage{graphicx}
\usepackage{caption}
\usepackage{float}
\usepackage{color}
\usepackage{epigraph}
\usepackage[hang,flushmargin]{footmisc} 
\usepackage{tikz}
\usepackage{hyperref}
\hypersetup{pdfauthor={Cristian Adrián Ontivero}}
\usetikzlibrary{automata, positioning, arrows,
fit, % for the dashed boxes on Thompson's construction
calc
}
\tikzset{%
  node distance=3cm, % specifies the minimum distance between two nodes. Change if necessary.
  every state/.style={thick}, % sets the properties for each ’state’ node
  double distance=2.5pt,
  shorten >= 2pt, shorten <= 2pt,
  initial text=$ $,
  every edge/.style={%
    draw,->, >=stealth, auto, semithick
  }
}
\graphicspath{{imgs/}}

\definecolor{darkblue}{RGB}{49,130,189}

\newlength\tindent
\setlength{\tindent}{\parindent}
\setlength{\parindent}{0pt}
\renewcommand{\indent}{\hspace*{\tindent}}

%These tell TeX which packages to use.
\usepackage{array,epsfig}
\usepackage{amsmath}
\usepackage{amsfonts}
\usepackage{amssymb}
\usepackage{amsxtra}
\usepackage{amsthm}
\usepackage{mathrsfs}
\usepackage{color}

%Here I define some theorem styles and shortcut commands for symbols I use often
\theoremstyle{definition}
\newtheorem{defn}{Definition}
\newtheorem{thm}{Teorema}
\newtheorem{cor}{Corolario}
\newtheorem*{rmk}{Remark}
\newtheorem{lem}{Lema}
\newtheorem*{joke}{Joke}

\newtheorem{ex}{Example}
\newcommand{\exautorefname}{Ejemplo}

\newtheorem{exercise}{Ejercicio}
\newcommand{\exerciseautorefname}{Ejercicio}

\newtheorem{soln}{Solución}
\newtheorem{prop}{Proposición}

\newcommand{\lra}{\longrightarrow}
\newcommand{\ra}{\rightarrow}
\newcommand{\surj}{\twoheadrightarrow}
\newcommand{\graph}{\mathrm{graph}}
\newcommand{\bb}[1]{\mathbb{#1}}
\newcommand{\Z}{\bb{Z}}
\newcommand{\Q}{\bb{Q}}
\newcommand{\R}{\bb{R}}
\newcommand{\C}{\bb{C}}
\newcommand{\N}{\bb{N}}
\newcommand{\M}{\mathbf{M}}
\newcommand{\m}{\mathbf{m}}
\newcommand{\MM}{\mathscr{M}}
\newcommand{\HH}{\mathscr{H}}
\newcommand{\Om}{\Omega}
\newcommand{\Ho}{\in\HH(\Om)}
\newcommand{\bd}{\partial}
\newcommand{\del}{\partial}
\newcommand{\bardel}{\overline\partial}
\newcommand{\textdf}[1]{\textbf{\textsf{#1}}\index{#1}}
\newcommand{\img}{\mathrm{img}}
\newcommand{\ip}[2]{\left\langle{#1},{#2}\right\rangle}
\newcommand{\inter}[1]{\mathrm{int}{#1}}
\newcommand{\exter}[1]{\mathrm{ext}{#1}}
\newcommand{\cl}[1]{\mathrm{cl}{#1}}
\newcommand{\ds}{\displaystyle}
\newcommand{\vol}{\mathrm{vol}}
\newcommand{\cnt}{\mathrm{ct}}
\newcommand{\osc}{\mathrm{osc}}
\newcommand{\LL}{\mathbf{L}}
\newcommand{\UU}{\mathbf{U}}
\newcommand{\support}{\mathrm{support}}
\newcommand{\AND}{\;\wedge\;}
\newcommand{\OR}{\;\vee\;}
\newcommand{\Oset}{\varnothing}
\newcommand{\st}{\ni}
\newcommand{\wh}{\widehat}

\newcommand{\emptystr}{\varepsilon}
\newcommand{\emptylan}{\emptyset}
\newcommand{\pdv}[2]{\partial_{#1} \bigl(#2\bigr)}

%Pagination stuff.
\setlength{\topmargin}{-.3 in}
\setlength{\oddsidemargin}{0in}
\setlength{\evensidemargin}{0in}
\setlength{\textheight}{9.in}
\setlength{\textwidth}{6.5in}
\pagestyle{empty}

% Print today's date in ISO format (YYYY-MM-DD).
\def\isodate{\leavevmode\hbox{\the\year-\twodigits\month-\twodigits\day}}
\def\twodigits#1{\ifnum#1<10 0\fi\the#1}

\begin{document}
\begin{center}
  {\LARGE Down the Regular Expression Rabbit Hole}\\[.2cm]
  Cristian Adrián Ontivero \\[.05cm]%
  \isodate
\end{center}

1.2 Suppose $\sim$ is an equivalence relation on a set S.
Since $\sim$ is reflexive, $x \sim x \forall x \in S$, so $x \in {[x]}_{\sim}
\forall x \in S$. Thus, no equivalence class is empty, and their union is the
whole set $S$.
It remains to show the disjointness of the equivalence classes.
Let $x$ and $y$ be two elements of $S$. Suppose ${[x]}_{\sim} \cap {[y]}_{\sim} \neq
\emptyset$. Then there exists $z \in {[x]}_{\sim} \cap {[y]}_{\sim}$, i.e. $z \in
{[x]}_{\sim} \wedge z \in {[y]}_{\sim}$. So, by definition of equivalence class,
$z \sim x$ and $z \sim y$. By symmetry $x \sim z$, and by transitivity $x \sim
y$. Once more, by symmetry $y \sim x$. Knowing all this, we proceed to prove
that ${[x]}_{\sim} = {[y]}_{\sim}$.

$\subseteq$: Let $d \in {[x]}_{\sim}$, this means that $d \sim x$. By
transitivity $d \sim y$, so $d \in {[y]}_{\sim}$.

$\supseteq$: Let $d \in {[y]}_{\sim}$, this means that $d \in y$. By
transitivity $d \in x$, so $d \in {[x]}_{\sim}$.

Thus ${[x]}_{\sim} = {[y]}_{\sim}$.

1.3 Given a partition \mathcal{P} on a set $S$, we define a relation $\sim$ on
$S$ as follows: $x \sim y$ iff $x, y \in X$, for some $X \in \mathcal{P}$.
Let's prove that $\sim$ is an equivalence relation on S.
\begin{itemize}
  \item Reflexive: Let $x \in S$ be arbitrary. $x \in X$ for some $X \in
    \mathcal{P}$, so $x \sim x$.
  \item Symmetric: Let $x, y \in S$
  \item Transitive: Let $x \in S$
\end{itemize}


\newpage 
\bibliography{refs}
\bibliographystyle{unsrt}

\end{document}



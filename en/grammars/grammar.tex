\documentclass{article}

\usepackage[utf8]{inputenc}
\usepackage{caption}
\usepackage{float}

\usepackage{tikz}
\usetikzlibrary{automata, positioning, arrows,
fit, % for the dashed boxes on Thompson's construction
calc
}
\tikzset{%
  node distance=3cm, % specifies the minimum distance between two nodes. Change if necessary.
  every state/.style={thick}, % sets the properties for each ’state’ node
  double distance=2.5pt,
  shorten >= 2pt, shorten <= 2pt,
  initial text=$ $,
  every edge/.style={%
    draw,->, >=stealth, auto, semithick, font=\tt
  }
}

\newcommand{\emptystr}{\varepsilon}

\usepackage{hyperref}
\hypersetup{pdfauthor={Cristian Adrián Ontivero}}
\graphicspath{{imgs/}}

\newlength\tindent%
\setlength{\tindent}{\parindent}
\setlength{\parindent}{0pt}
\renewcommand{\indent}{\hspace*{\tindent}}

\renewcommand*{\tableautorefname}{Tabla}
\renewcommand*{\figureautorefname}{Figura}

%These tell TeX which packages to use.
\usepackage{array,epsfig}
\usepackage{amsmath}
\usepackage{amsfonts}
\usepackage{amssymb}
\usepackage{amsxtra}
\usepackage{amsthm}
\usepackage{mathrsfs}

\newcommand{\lra}{\longrightarrow}
\newcommand{\ra}{\rightarrow}
\newcommand{\surj}{\twoheadrightarrow}
\newcommand{\graph}{\mathrm{graph}}
\newcommand{\bb}[1]{\mathbb{#1}}
\newcommand{\Z}{\bb{Z}}
\newcommand{\Q}{\bb{Q}}
\newcommand{\R}{\bb{R}}
\newcommand{\C}{\bb{C}}
\newcommand{\N}{\bb{N}}
\newcommand{\M}{\mathbf{M}}
\newcommand{\m}{\mathbf{m}}
\newcommand{\MM}{\mathscr{M}}
\newcommand{\HH}{\mathscr{H}}
\newcommand{\Om}{\Omega}
\newcommand{\Ho}{\in\HH(\Om)}
\newcommand{\bd}{\partial}
\newcommand{\del}{\partial}
\newcommand{\bardel}{\overline\partial}
\newcommand{\textdf}[1]{\textbf{\textsf{#1}}\index{#1}}
\newcommand{\img}{\mathrm{img}}
\newcommand{\ip}[2]{\left\langle{#1},{#2}\right\rangle}
\newcommand{\inter}[1]{\mathrm{int}{#1}}
\newcommand{\exter}[1]{\mathrm{ext}{#1}}
\newcommand{\cl}[1]{\mathrm{cl}{#1}}
\newcommand{\ds}{\displaystyle}
\newcommand{\vol}{\mathrm{vol}}
\newcommand{\cnt}{\mathrm{ct}}
\newcommand{\osc}{\mathrm{osc}}
\newcommand{\LL}{\mathbf{L}}
\newcommand{\UU}{\mathbf{U}}
\newcommand{\support}{\mathrm{support}}
\newcommand{\AND}{\;\wedge\;}
\newcommand{\OR}{\;\vee\;}
\newcommand{\Oset}{\varnothing}
\newcommand{\st}{\ni}
\newcommand{\wh}{\widehat}

%Pagination stuff.
\setlength{\topmargin}{-.3 in}
\setlength{\oddsidemargin}{0in}
\setlength{\evensidemargin}{0in}
\setlength{\textheight}{9.in}
\setlength{\textwidth}{6.5in}
\pagestyle{empty}

\begin{document}

Interesting grammars:

Lambda Calculus:

\[
  \begin{array}{crl}
    E & ::= & x \\
      & \mid& (\lambda x.~E) \\
      & \mid& (E~E) \\
  \end{array}
\]

Non-Deterministic Lambda Calculus \(\Lambda^*\) \cite{AZ13}:
\[
  \begin{array}{lrrl}
   \text{Constants} & \mathtt{c}    & ::= & \mathtt{It^*} \mid \mathtt{if^*} \mid \mathtt{c_0} \mid \mathtt{c_1} \mid \dots \\
   \text{Terms}     & t, u & ::= & x \mid (\lambda x.~t) \mid (t)~u \mid \langle t, u \rangle \mid [t]\pi_0 \mid [t]\pi_1 \mid \mathtt{c}
  \end{array}
\]

CFG grammar for Brainfuck

\newcommand\Prog{\mathtt{Prog}}
\newcommand\Instr{\mathtt{Instr}}
\[
  \begin{array}{crl}
    \Prog  & \to & \Instr~\Prog \mid \varepsilon \\
    \Instr & \to & \mathtt{+} \mid \mathtt{-} \mid \mathtt{>} \mid \mathtt{<}
    \mid \mathtt{,} \mid \mathtt{.} \mid \mathtt{[}~\Prog~\mathtt{]}
\end{array}
\]

CFG for the functional SKI combinator calculus:

\[
  \begin{array}{crl}
  P & \to & E \\
  E & \to & \mathbf{S}~E~E~E \mid \mathbf{K}~E~E \mid \mathbf{I} \mid (~E~)
\end{array}
\]

% From Mathematical Logic for Computer Science by Ben-Ari
CFG for Propositional Logic

\begin{align*}
  F &::= p \mid \neg F \mid F O F \\
  O &::= \land \mid \lor \mid \to \mid \leftrightarrow \mid \oplus
\end{align*}

A formula \(F\) is in \textit{conjunctive normal form} (CNF) if it is a
conjunction of clauses, where each clause \(C\) is a disjunction of literals:
\begin{align*}
  F &::= C \mid C \land F \\
  C &::= L \mid L \lor C \\
  L &::= p \mid \neg p
\end{align*}

Horn Formula
\begin{align*}
  P &::= \bot \mid \top \mid p \\
  A &::= P \mid P \land A \\
  C &::= A \to P \\
  H &::= C \mid C \land H
\end{align*}

A propositional logic notation that doesn't require parenthesis is the so
called \textit{Polish} (or \textit{head-initial}) notation, due to
the polish mathematician Jan Łukasiewicz\footnote{Most letters used as
connectives come from their corresponding polish names: \textit{negacja} (\(N\)),
\textit{koniunkcja} (\(K\)), \textit{alternatywa} (\(A\)),
\textit{implikacja} (\(C\)), and \textit{ekwiwalencja} (\(E\)).}:

\[
  \phi ::= p \mid N \phi \mid K \phi \phi \mid A \phi \phi \mid C \phi \phi \mid E \phi \phi
\]




Bryant, Seger and Beatty~\cite{BS90, BSB91} developed and algorithm based on
symbolic simulation for model checking in a restricted linear temporal logic.
Trajectory Formula:

Let \(\mathcal{V}\) be a set of symbolic Boolean variables and
\(\mathcal{N}\) a set of nodes.
A \textit{step-level} symbolic trajectory formula is defined recursively:
\begin{itemize}
  \item Constants: TRUE is a trajectory formula
  \item Atomic propositions: for \(n_i \in N\) both \((n_i = 1)\) and \((n_i = 0)\) are trajectory formulas.
  \item Conjunction: \((F_1 \land F_2)\) is a trajectory formula if \(F_1\) and \(F_2\) are trajectory formulas.
  \item Domain restriction: \((E \to F)\) is a trajectory formula if \(E\) is
  a Boolean expression over \(\mathcal{V}\) and \(F\) is a trajectory
  fomrula.
  \item Next time: \((X_s F)\) is a trajectory formula if \(F\) is a trajectory formula.
\end{itemize}

Basic modal logic

\[
  \phi ::= \bot \mid \top \mid p \mid (\neg \phi) \mid (\phi \land \phi) \mid (\phi \lor \phi) \mid (\phi \to \phi) \mid (\phi \leftrightarrow \phi) \mid (\square \phi) \mid (\lozenge \phi)
\]


% Logic in Computer Science (Huth & Ryan) 2nd ed., page 390.
Relational \(\mu\)-calculus:
  \begin{align*}
    f & ::= 0 \mid 1 \mid v \mid \overline{f} \mid f_1 + f_2 \mid f_1 \cdot f_2 \mid f_1 \oplus f_2 \mid \exists x.~f \mid \forall x.~f \mid \mu Z.~f \mid \nu Z.~f \mid f[\hat{x} := \hat{x}'] \\
    v & ::= x \mid Z
  \end{align*}
where \(x\) and \(Z\) are boolean variables, and \(\hat{x}\) is a tuple of variables.

A formula in the multi-modal logic KT\(45^n\) is defined as:
\[
  \phi ::= \bot \mid \top \mid p \mid (\neg \phi) \mid (\phi \land \phi) \mid (\phi \lor \phi) \mid (\phi \to \phi) \mid (\phi \leftrightarrow \phi) \mid (K_i~\phi) \mid (E_G~\phi) \mid (C_G~\phi) \mid (D_G~\phi)
\]
Equivalent, optimized grammar % Hecho por mi, no sacado de libro.
  \begin{align*}
    F & ::= \bot \mid \top \mid p \mid P \\
    P & ::= (I) \\
    I & ::= U F \mid F B F \\
    U & ::= \neg \mid K_i \mid E_G \mid C_G \mid D_G \\
    B & ::= \land \mid \lor \mid \:\to \: \mid \:\leftrightarrow
  \end{align*}

Others:

WHILE programming language (a pedagogical language. TODO Make reference to first definition, by Uwe
Schöning in his Theoretische Informatik - kurz gefasst)

LTL

CTL

CTL*

FOL

Łukasiewicz logic

Classical linear logic (CLL)


\newpage
\bibliography{refs}
\bibliographystyle{unsrt}

\end{document}

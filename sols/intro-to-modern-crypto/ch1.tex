
English Freqs | Ciphertext Freqs
e   12.7      | F  37   e
t   9.1       | Q  26   t
a   8.2       | W  21   s
o   7.5       | G  19   r
i   7.0       | L  17   n
n   6.7       | O  16   o
s   6.3       | V  15   h
h   6.1       | H  14   a
r   6.0       | B  12
d   4.3       | P  10
l   4.0       | J  9
u   2.8       | I  9
c   2.8       | Z  7
w   2.4       | R  7
m   2.4       | M  4
f   2.2       | E  4
y   2.0       | Y  3
g   2.0       | K  3
p   1.9       | C  3
b   1.5       | A  3
v   1.0       | S  2
k   0.8       | D  2
x   0.2       | X  1
j   0.2
z   0.1
q   0.1

Ordenamos. De entrada es muy probable que F = e, y asumimos también que Q = t.
Para el resto de los caracteres, es menos claro. Resolverlo solo con la
frecuencia de letras requiere mucha prueba y error. Nos ayudamos con la
frecuencia de digrafos \footnote{Cortesía de Norvig:
http://norvig.com/mayzner.html}:

TH  (3.56%)  ("QV",9) -- th
HE  (3.07%)  ("FP",8) -- 
IN  (2.43%)  ("VF",7) -- he
ER  (2.05%)  ("GF",7) -- re
AN  (1.99%)  ("QO",6) -- to
RE  (1.85%)  ("PF",5)
ON  (1.76%)  ("OL",5) 
AT  (1.49%)  ("FW",5) -- es
EN  (1.45%)  ("FG",5) -- er
ND  (1.35%)  ("WQ",4) -- st
TI  (1.34%)  ("LF",4)
ES  (1.34%)  ("JV",4)
OR  (1.28%)  ("GQ",4)
TE  (1.20%)  ("WJ",3)
OF  (1.17%)  ("WB",3)
ED  (1.17%)  ("VH",3) 
IS  (1.13%)  ("QW",3)
IT  (1.12%)  ("QG",3)
AL  (1.09%)  ("OG",3) -- or
AR  (1.07%)  ("OC",3) -- of
ST  (1.05%)  ("HQ",3) -- at
TO  (1.04%)  ("HL",3) -- an
NT  (1.04%)  ("HG",3) -- ar

Partiendo del supuesto que Q = t, y F = e, es lógico asumir que V = h (TH y HE
son muy frequentes, al igual que QV y VF). También vemos que ER y RE son
frecuentes en inglés, y tenemos GF y FG con frecuencia en el texto cifrado, con
lo que sumamos G = r a nuesta lista de supuestos (también sería posible usar ese
razonamiento con FP y PF, pero en este caso resulta que P no es r, y para
simplificar el camino... etc.).

Si con estas cuatro equivalencias probamos decifrar, vemos algunas ocurrencias
de "the", lo cual parecería indicar que vamos en buen camino. También vemos
"Other", con lo que parecería que O = o, y "thHt", lo cual asumimos que es
"that", es decir H = a. Agregamos estas nuevas equivalencias y probamos denuevo.

Vemos que "eAer" se repite, suponemos que A = v, y repetimos.
Vemos "hoSever", suponemos S = w, y repetimos.
Volvemos a la tabla de digrafos, y sabiendo que O = o, asumimos L = n, ya que
"ON" es el digrama más frecuente que empieza con O.
Vemos "thananRother", asumimos R = y.
Vemos "nevertheZeWW", pareciera ser "nevertheless", asumimos Z = l, W = s.
Vemos "systeP", "seeP", "theP", asumimos P = m.
Vemos "eDtremely", D = x.
Vemos "neverthelessEorsomereason", E = f.
Vemos "manynonexMerts", M = p.
Vemos "JryptoYraph", asumimos J = c, Y = g.
Vemos "cryptographBc", asumimos B = i.
Vemos "KifficIlt", asumimos K = d, I = u.
Vemos "toCuild", asumimos C = b.
Vemos "trivialtobreaX", X = k

